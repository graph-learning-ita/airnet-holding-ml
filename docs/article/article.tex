%%
%% Copyright 2007-2020 Elsevier Ltd
%%
%% This file is part of the 'Elsarticle Bundle'.
%% ---------------------------------------------
%%
%% It may be distributed under the conditions of the LaTeX Project Public
%% License, either version 1.3 of this license or (at your option) any
%% later version.  The latest version of this license is in
%%    http://www.latex-project.org/lppl.txt
%% and version 1.3 or later is part of all distributions of LaTeX
%% version 1999/12/01 or later.
%%
%% The list of all files belonging to the 'Elsarticle Bundle' is
%% given in the file `manifest.txt'.
%%
%% Template article for Elsevier's document class `elsarticle'
%% with harvard style bibliographic references

\documentclass[preprint,12pt,authoryear]{elsarticle}

%% Use the option review to obtain double line spacing
%% \documentclass[authoryear,preprint,review,12pt]{elsarticle}

%% Use the options 1p,twocolumn; 3p; 3p,twocolumn; 5p; or 5p,twocolumn
%% for a journal layout:
%% \documentclass[final,1p,times,authoryear]{elsarticle}
%% \documentclass[final,1p,times,twocolumn,authoryear]{elsarticle}
%% \documentclass[final,3p,times,authoryear]{elsarticle}
%% \documentclass[final,3p,times,twocolumn,authoryear]{elsarticle}
%% \documentclass[final,5p,times,authoryear]{elsarticle}
%% \documentclass[final,5p,times,twocolumn,authoryear]{elsarticle}

%% For including figures, graphicx.sty has been loaded in
%% elsarticle.cls. If you prefer to use the old commands
%% please give \usepackage{epsfig}

\usepackage[T1]{fontenc}		% Seleção de códigos de fonte.
\usepackage[utf8]{inputenc}		% Codificação do documento (conversão automática dos acentos)
\usepackage{lmodern}			% Usa a fonte Latin Modern
% Para utilizar a fonte Times New Roman, inclua uma % no início do comando acima  "\usepackage{lmodern}"
% Abaixo, tire a % antes do comando  \usepackage{times}
%\usepackage{times}		    	% Usa a fonte Times New Roman
% Para usar a fonte , lembre-se de tirar a % do comando %\renewcommand{\ABNTEXchapterfont}{\rmfamily}, localizado mais abaixo, logo após "Outras opções para nota de rodapé no Sistema Numérico"
\usepackage{lastpage}			% Usado pela Ficha catalográfica
\usepackage{indentfirst}		% Indenta o primeiro parágrafo de cada seção.
\usepackage{color}				% Controle das cores
\usepackage{graphicx}			% Inclusão de gráficos
\usepackage{float} 				% Fixa tabelas e figuras no local exato
\usepackage{chemfig,chemmacros} % Para escrever reações químicas
\usepackage{tikz}				% Para escrever reações químicas e outros
\usetikzlibrary{positioning}
\usepackage{microtype} 			% para melhorias de justificação
\usepackage{pdfpages}
\usepackage{makeidx}            % para gerar índice remissivo
\usepackage{hyphenat}          % Pacote para retirar a hifenizacao do texto
\usepackage[absolute]{textpos} % Pacote permite o posicionamento do texto
\usepackage{eso-pic}           % Pacote para incluir imagem de fundo
\usepackage{makebox}           % Pacote para criar caixa de texto

\usepackage{graphicx}
\usepackage{graphbox}
\usepackage{tasks}
\usepackage{amssymb}
\usepackage{amsmath}
\usepackage{amsthm}
\usepackage{fontawesome5}

% TODO List
\usepackage{todonotes}

\usepackage{svg} % ADDED JORGE FOR SVG

\usepackage[ruled]{algorithm2e}

\usepackage{tikz}
\usepackage{aircraftshapes}

\usetikzlibrary{positioning}

\newcommand{\destaq}[1]{\textcolor{BlueViolet}{\textbf{#1}}}
\renewcommand\todo[1]{\colorbox{green}{#1}}

\definecolor{echodrk}{HTML}{0099cc}
\definecolor{olivegreen}{rgb}{0,0.6,0}
\definecolor{camdrk}{RGB}{0,62,114}

\usetikzlibrary{arrows,shapes}

\definecolor{mygreen}{rgb}{0,0.6,0}

\definecolor{mymauve}{rgb}{0.58,0,0.82}

\usetikzlibrary{arrows,shapes, decorations.pathmorphing,backgrounds,positioning}

\pgfdeclarelayer{background}
\pgfsetlayers{background,main}

\tikzstyle{vertex}=[circle,fill=black!25,minimum size=20pt,inner sep=0pt]
\tikzstyle{selected vertex} = [vertex, fill=red!24]
\tikzstyle{select vertex} = [vertex, fill=blue!24]
\tikzstyle{selectx vertex} = [vertex, fill=green!24]
\tikzstyle{edge} = [draw,thick,-]
\tikzstyle{selected edge} = [draw,line width=5pt,-,red!50]

\usepackage{minted}
\usemintedstyle{vs}

\usepackage{enumitem}
\newlist{todolist}{itemize}{2}
\setlist[todolist]{label=$\square$}
\usepackage{pifont}
\newcommand{\cmark}{\ding{51}}%
\newcommand{\xmark}{\ding{55}}%
\newcommand{\done}{\rlap{$\square$}{\raisebox{2pt}{\large\hspace{1pt}\cmark}}%
\hspace{-2.5pt}}
\newcommand{\wontfix}{\rlap{$\square$}{\large\hspace{1pt}\xmark}}

%% MANOEL: CORREÇÕES PARA ADAPTAÇÃO PARA ARTIGO
\let\chapter\section
\let\section\subsection
\let\citeonline\cite
\renewcommand{\refname}{\chapter{References}}

\newtheorem{theorem}{Theorem}[section]
\newtheorem{lemma}[theorem]{Lemma}
\newtheorem{property}[theorem]{Property}

\renewcommand\qedsymbol{$\blacksquare$}

\usepackage{lipsum}				% para geração de dummy text
% ---

% pacotes de tabelas
\usepackage{multicol}	% Suporte a mesclagens em colunas
\usepackage{multirow}	% Suporte a mesclagens em linhas
\usepackage{longtable}	% Tabelas com várias páginas
\usepackage{threeparttablex}    % notas no longtable
\usepackage{array}

\usepackage{subfig}


%% The lineno packages adds line numbers. Start line numbering with
%% \begin{linenumbers}, end it with \end{linenumbers}. Or switch it on
%% for the whole article with \linenumbers.
%% \usepackage{lineno}

%% MANOEL: REMOVE PREPRINT NOTE TO THE FRONTPAGE
\makeatletter
\def\ps@pprintTitle{%
  \let\@oddhead\@empty
  \let\@evenhead\@empty
  \def\@oddfoot{\reset@font\hfil\thepage\hfil}
  \let\@evenfoot\@oddfoot
}
\makeatother

% ---
% compila o sumário e índice
\makeindex
% ---



\begin{document}

\begin{frontmatter}

%% Title, authors and addresses

%% use the tnoteref command within \title for footnotes;
%% use the tnotetext command for theassociated footnote;
%% use the fnref command within \author or \affiliation for footnotes;
%% use the fntext command for theassociated footnote;
%% use the corref command within \author for corresponding author footnotes;
%% use the cortext command for theassociated footnote;
%% use the ead command for the email address,
%% and the form \ead[url] for the home page:
%% \title{Title\tnoteref{label1}}
%% \tnotetext[label1]{}
%% \author{Name\corref{cor1}\fnref{label2}}
%% \ead{email address}
%% \ead[url]{home page}
%% \fntext[label2]{}
%% \cortext[cor1]{}
%% \affiliation{organization={},
%%            addressline={},
%%            city={},
%%            postcode={},
%%            state={},
%%            country={}}
%% \fntext[label3]{}

\title{Graph machine learning for flight delay prediction due to
    holding manouver} %% Article title

\affiliation[label1]{organization={Instituto Tecnológico de Aeronáutica},
               addressline={Praça Marechal Eduardo Gomes, 50 - Vila das Acácias},
               city={São José dos Campos},
               postcode={12228-900},
               state={São Paulo},
               country={Brazil}}

\author[label1]{Manoel Vilela Machado Neto}
\author[label1]{Jorge Luiz Franco}
\author[label1]{Felipe Alves Neto Verri}


% Abstract
\begin{abstract}
%% Text of abstract
This project models the prediction of flight delays due to holding
maneuvers as a graph problem, leveraging advanced Graph Machine
Learning (Graph ML) techniques to capture complex interdependencies in
air traffic networks. Holding maneuvers, while crucial for safety,
cause increased fuel usage, emissions, and passenger dissatisfaction,
making accurate prediction essential for operational
efficiency. Traditional machine learning models, typically using
tabular data, often overlook spatial-temporal relations within air
traffic data. To address this, we model the problem of predicting
holding as edge feature prediction in a directed (multi)graph where we
apply both CatBoost, enriched with graph features capturing network
centrality and connectivity, and Graph Attention Networks (GATs),
which excel in relational data contexts. Our results indicate that
CatBoost outperforms GAT in this imbalanced dataset, effectively
predicting holding events and offering interpretability through
graph-based feature importance. Additionally, a web-based tool,
Airdelay, allows users to simulate real-time delay predictions,
demonstrating the model's potential operational impact. This research
underscores the viability of graph-based approaches for predictive
analysis in aviation, with implications for enhancing fuel efficiency,
reducing delays, and improving passenger experience.


\end{abstract}

%%Graphical abstract
%%\begin{graphicalabstract}
%\includegraphics{grabs}
%\end{graphicalabstract}

%%Research highlights
%\begin{highlights}
%\item Research highlight 1
%\item Research highlight 2
%\end{highlights}

%% Keywords
\begin{keyword}
%% keywords here, in the form: keyword \sep keyword

%% PACS codes here, in the form: \PACS code \sep code

%% MSC codes here, in the form: \MSC code \sep code
%% or \MSC[2008] code \sep code (2000 is the default)
   Graph Neural Networks \sep Graphs \sep Machine Learning \sep Complex Networks

\end{keyword}

\end{frontmatter}
% Seleciona o idioma do documento (conforme pacotes do babel)
%\selectlanguage{brazil}
% Se o idioma do texto for inglês, inclua uma % antes do
%      comando \selectlanguage{brazil} e
%      retire a % antes do comando abaixo

% Retira espaço extra obsoleto entre as frases.
\frenchspacing

% ---
% ----------------------------------------------------------
% ELEMENTOS TEXTUAIS
% ----------------------------------------------------------
% Os capítulos são inseridos como arquivos externos

% Capítulo 1 - Introdução
% ---
%% USPSC-Introducao.tex

% ----------------------------------------------------------
% Introdução (exemplo de capítulo sem numeração, mas presente no Sumário)
% ----------------------------------------------------------
\chapter[Introduction]{Introduction}
\label{Introdução}

Efficient last-mile delivery logistics is vital for accurate and timely delivery of goods, which influences customer satisfaction and overall business success \cite{boysen2021lastmile}. The use of drones, also called Unmanned Aerial Vehicles (UAVs), in Last-Mile Delivery improves efficiency by overcoming traffic constraints, reducing delivery times, and lowering operational costs.  For this reason, the literature on Delivery Drones increased significantly in the last few years, as analyzed in \citeonline{DUKKANCI2023}.

The literature related to Last-Mile Delivery Drones is highly heterogeneous, encompassing a diverse array of techniques and problem formulations. This diversity spans from studies involving both drones and trucks in delivery scenarios, linear integer modeling for logistic problem-solving, applications of fuzzy logic to address uncertainties, multi-objective optimization to consider multiple criteria simultaneously, to exclusive investigations focusing solely on drones without the presence of other vehicles, including decentralized models.

In \citeonline{FREITAS2023100094} it is studied the Truck-Drone Delivery Problem, which can be addressed as a variation of TSP (Traveling Salesman Problem) for two agents, using a Mixed Integer Programming (MIP) formulation and a heuristic based on a TSP solver and two metaheuristics: General Variable Neighborhood Search (GVNS) and Tabu Search (TS). At the same time, \citeonline{MOSHREFJAVADI2020290} addresses a near problem, but with multiple drones, using Mixed Integer Linear Programming (MILP) and a heuristic based on Adaptive Large Neighborhood Search (ALNS) and Simulated Annealing (SA). A similar MILP modeling was used in \citeonline{drones7070407} and two hybrid genetic algorithms were evaluated.



When addressing multiple criteria simultaneously in collaborative Truck-Drone Delivery, researchers have employed multi-objective optimization techniques.  \citeonline{9580555} consider flexible time windows and diverse objectives, including minimizing the routing costs of drones and trucks and maximizing customer satisfaction. They improved NSGA-II(Non-dominated Sorting Genetic Algorithm) and used a posterior heuristic for Pareto Local Search(PLS).   Complementing this, \citeonline{ZHANG2022108679} introduces a novel multi-objective optimization model for drone delivery, optimizing economic and environmental goals concurrently. The model addresses dynamic drone flight endurance based on loading rates, using an extended NSGA-II algorithm that proves to be effective in generating high-quality solutions, marking advancements in this field.

%Now, article fuzzy, sentiment and energy.

In another branch, works such as \citeonline{farjana2020last} introduce a systematic multi-criterion, multipersonnel decision-making approach called interval-valued inferential fuzzy TOPSIS, handling fuzziness in decision-making and providing quality drone selection decisions. Another study \citeonline{9672856} considers sustainability in optimizing the vehicle routing problem with drones, utilizing sentiment analysis based on Twitter to determine customer sentiments on environmental protection. The net promoter score index calculated from the sentiment analysis is considered as a coefficient in the penalty function added to the base model, aiming to help logistics companies improve the sustainability of the supply chain. Addressing the aspect of energy efficiency in drone-based last mile delivery, another study \citeonline{10049551} focuses on tactical decisions about the selection of shared fulfillment centers used as launch and recovery locations for drones, fleet size plans, and operational drone route decisions. 

%até aqui sem collision

A new perspective in Last Mile Delivery Drones was introduced in \citeonline{Verri}. The study addressed the last-mile delivery problem from a complex system viewpoint, where the collective performance of the drones is investigated. The delivery system incorporates a \textit{tradable permit model} \cite{AKAMATSU2017178}, based on a financial decentralized market-inspired approach, for airspace use, requiring drones to compete for airspace permits in a distributed manner. The simulation evaluates how different parameters, such as arrival rate and airspace dimensions, impact system behavior in terms of cost, time required for drones to acquire flight permits and airspace utilization. Although the simulation employs a simplified model with naïve agents and disregards drone flight dynamics, it captures interesting properties in the agents' collective behavior, demonstrating satisfactory system performance even under high traffic conditions. Simultaneously, \citeonline{lee2022last} contributes a novel combinatorial double auction bi-objective winner determination problem for last-mile drone delivery. 

As we can see, the Last Mile Delivery Drone problem is very hard and complex. In \citeonline{faical}, the authors characterize the problem through a cyber-physical lens, highlighting challenges related to the physical aspects of airspace in the context of drone operations for last-mile delivery. Their analysis delves into considerations such as air traffic management systems, focusing on the evolving landscape of airspace control. This review underscores the need for optimal usage of airspace, then ensuring the need for collision avoidance. However, collision avoidance is a problem that has not been addressed by most of the cited papers until now. \citeonline{Verri} addresses this systematically within the decentralized approach. Indeed, \citeonline{DUKKANCI2023} highlights collision avoidance and air traffic management as a crucial aspect that needs attention in Last-mile delivery drone problems, as much of the literature tends to overlook this when modeling the problem. 


With this problem of collision avoidance and efficient airspace control in mind, the main idea of our work is solve the Last Mile Delivery Drone problem using a MAPF (Multi-Agent Path Finding) approach, thus not ignoring spatial characteristics natural from the airspace. The problem addressed in this work can easily be seen as multi-objective drone path planning for search and rescue \cite{hayat2020multi}, thus the Last Mile Delivery Drone is per se a MAPF.  As stated in \citeonline{lavalle} and proved in \citeonline{Nebel_2020} the MAPF problem is NP-hard. Then our approach is to use elements of prioritized planning \cite{7138650} and conflict-based search \cite{SHARON201540} to face the computational complexity inherited from MAPF while satisfying the constraints of the reduced problem. Although our algorithm does not guarantee global optimum, it always converges in a finite and bounded polynomial time, which is an improvement compared to the previous algorithms described ( MILPs, MOEAs, Metaheuristics).



Centralized control for airspace management, particularly under the Unmanned Aircraft System Traffic Management (UTM) framework developed by the Federal Aviation Administration (FAA) and NASA, exemplifies the necessity of organized, legislative-backed airspace control \cite{nasa}. The UTM facilitates multiple drone operations beyond visual line of sight (BVLOS) by enabling cooperative interaction between drone operators and the FAA, determining and communicating real-time airspace status. While decentralized models, such as those based on blockchain technology \cite{Verri}, offer novel approaches, they introduce complexities in scalability, regulatory compliance, and operational efficiency that can hinder real-time airspace control. The centralized UTM system, supported by regulatory compliance and legislative backing, ensures effective airspace management, safety, and reliability, fostering the integration of UAV deliveries within existing air traffic systems. The FAA's UTM Implementation Plan \cite{9256745} and ongoing evaluations highlight the continual refinement of this centralized approach, emphasizing its critical role in future UAV operations. 


In this study, we propose a novel approach to tackle the Last Mile Delivery Drone problem by employing a Multi-Agent Path Finding (MAPF) strategy. Leveraging elements of prioritized planning and conflict-based search, our algorithm aims to address the computational complexity inherent in the MAPF problem. Unlike previous algorithms such as MILPs and MOEAs, our heuristic guarantees convergence in finite and bounded polynomial time, representing a significant advancement. Our graph-based approach, employing Breadth-First Search (BFS) on temporal grids, enhances the efficiency of our solution. Also, we propose a hybrid method, that combines the heuristic and the MILP. We further present experimental results, comparing our approach with a MILP-based solution and evaluating its performance in the hybrid approach that combines the time found by the heuristic and the MILP solution. Finally, a qualitative comparison between our centralized approach and the decentralized, proposed in \citeonline{Verri}, is presented. The use of a graph-based heuristic offers distinctive advantages, and our experiments shed light on its efficacy in solving the complex Last Mile Delivery Drones problem.

Therefore, in what follows, we describe the main contributions
of this work:
\begin{itemize}
    \item We develop a MILP model to solve the LMDD problem exactly. This model offers a novel method to adapt the MAPF solution to the LMDD context, allowing for drone take-offs, waiting in airspace, and correct landings.
    \item We propose a heuristic algorithm to address the LMDD problem. The primary advantage of our heuristic lies in its graph-based structure, which can be adapted to higher dimensions and weighted sets in the LMDD context. Additionally, our heuristic is polynomially bounded with a low computational cost, representing a novel contribution to the current literature on drone delivery solutions.
\end{itemize}




This monograph is organized as follows. In Section 2 we present the problem definition and the new property. The proposed solution approach is described in Section 3, while the results of the computational experiments are presented in Section 4. Finally, in Section 5 conclusions are drawn.


% ---

% ---
% Capítulo 2

\chapter[Theoretical Framework]{Theoretical Framework and Related Works}
\label{TheoreticalFramework}

Graph machine learning can be tracked backwards to the problem of `learning' on data that is inherently a graph \cite{silva2016machine, JMLR:Perozzi} or can be modeled as a graph \cite{verri2013,grape2020}. This field encompasses a variety of tasks, including node/edge classification, network construction, link prediction, graph classification, graph cut/partitioning, network embeddings, graph coarsening/reduction, which rely on learning representations from graph-structured data. Over the last decades, researchers have developed numerous approaches to tackle these challenges, initially these techniques were most developed by complex networks researchers. However, in the last decade with the advancements in deep learning, the field has seen a significant shift towards the merging of three main communities: graph signal processing, deep learning and complex nets.

As described, defining the field of graph machine learning is not straightforward, as it encompasses a broad range of methods and applications. The tasks mentioned above are just a few examples of the many challenges that can be addressed through graph-based learning techniques. For clarity, these tasks can be categorized into three main learning paradigms: supervised, unsupervised, and semi-supervised learning. In this study, we are interested on the (semi-)supervised learning paradigm, which encompasses a variety of techniques designed to leverage learning to (partially-)labeled data \cite{verri2018advantages,amanciof}. But we can refine even more, in fact, this work will focus in the subset of graph elements prediction(classification/regression) methods.

In this chapter, we provide an overview of the theoretical framework of graph machine learning for node/edge prediction. Here we consider the division of the field into \texttt{classical} graph learning and \texttt{deep} graph learning, where here `classical' refers to the machine learning techniques applied to graphs before the advent of graph neural networks, where standard ML algorithms were applied to graph data and the topological information measures were encoded as features together with the tabular data  \cite{costa2007characterization, silva2016machine}. This bipartition is what will pave the way of our explanation, since the last decade has seen a complex interplay between these two approaches. The field's evolution can be traced back to when \citeonline{bruna2013spectral} introduced one of the first GNN architectures leaned on the theory of graph signal processing. Concurrently, researchers were developing node embedding techniques like DeepWalk \cite{perozzi2014deepwalk} and node2vec \cite{grover2016node2vec}, which bridged classical and deep approaches while remaining using complex networks concepts. The subsequent years saw a surge in GNN architectures, including Graph Convolutional Networks \cite{kipf2016semi} and GraphSAGE \cite{hamilton2017inductive}, marking a shift towards more sophisticated deep learning approaches for graphs and the unification of the field.  

In the following sections, we explain each subset, their theory and applications, and how they have evolved over time. We also discuss the challenges and limitations of these methods.

\section{Classical graph learning}



These early efforts focused on shallow learning techniques such as feature engineering, graph traversal algorithms, and spectral methods, which laid the foundation for understanding graph structure and dynamics. Methods like community detection, centrality measures, and link prediction became key tools for analyzing large-scale networks in areas such as social science, biology, and infrastructure systems. By modeling relationships as graphs, these approaches enabled researchers to capture both local and global properties, leading to significant insights in network theory and real-world applications.

\section{Deep graph learning}

The rise of deep learning has revolutionized the field of graph machine learning, enabling the development of more powerful and scalable models for graph data. Graph neural networks can be divide in two main categories: spectral-based and spatial-based. Here is a trick thing, the GCN architecture \cite{kipf2016semi} is commonly divulgated as a spatial-based method, since it is more intuitive talking about the convolution operation in the spatial domain, where we simply aggregate information from the immediate neighbors. However, the GCN is a spectral-based method, in fact, it can be thought as a simplification of the first spectral GNN \cite{bruna2013spectral} proposed and that builds the math behind GCNs. That said, first we introduce the spectral-based GNNs and then the spatial-based ones.

\subsection{Spectral-based GNNs}

Spectral methods are rooted in graph signal processing. The core idea is that a signal on a graph can be represented as node features, where each feature vector at a node corresponds to a `signal' defined over the graph. In this context, the graph Laplacian $L = D - A$, where $D$ is the degree matrix and $A$ is the adjacency matrix, plays a crucial role. It captures the structure of the graph and can be used to perform operations analogous to Fourier transforms in classical signal processing. Spectral methods can be categorized into two types: eigenvalue-based, where the focus is on creating a graph filter in the Fourier domain, and eigenvector-based, where the goal is to use a spectral basis to decompose the signal \cite{bo2023surveyspectralgraphneural}.

As we already stated, \citeonline{bruna2013spectral} was the first spectral GNN proposed and the ideia was to translate the ideias from the standard CNN for images to graphs, thats why it's called Spectral CNN (SCNN). The ideia of the SCNN is to use the spectral decomposition of the graph Laplacian $L= U \Lambda U^T $ to define a filter convolution operation in the Fourier domain, where the graph fourier transform in a signal $f$ become $\hat{f}= U^T f$. The convolution operation ($\star$) is defined as $g_{\theta} \star f = U g_{\theta} U^T x$, where $g_{\theta}$ is a learnable filter parameterized by $\theta$, finally the learning in the . The SCNN is a powerful method, but it needs to calculate the entire spectrum of the graph $\mathcal{O}(n^3)$ for a graph, which can be computationally expensive for large graphs.

% ---

%\include{6_mathematical_model}

%\include{7_heuristic}



%\include{5_materialsAndMethods}

% ---
% Capítulo 3
% ---
\chapter[Materials and Methods]{Materials and Methods}
\label{Materials}

Our data is a lot imbalanced, but there is little work on imbalanced data in the literature for graph machine learning and most of the work is based on oversampling (\cite{GraphSmote} GRAPH SMOTE), however current literature shows that oversampling can lead to overfitting and poor generalization. In this work, we will deal with class imbalancing using undersampling for the GAT.  


% In this chapter, we present in the first two sections the methodologies of both methods we used to solve the LMDD problem. Further, we present the hybrid method, that unifies both approaches.

% \section[Mathematical Model]{Mathematical Model}
\label{Mathematical_Model}

We propose a Mixed-Integer Linear Programming (MILP) model to solve the problem. First, in Section~\ref{graph-modeling}, we illustrate in detail how to construct a graph representing the spatio-temporal permits shared by the drones.

\subsection{Graph Modeling} \label{graph-modeling}

Following the flow network problem paradigm, we model our spatio-temporal permits as a directed graph (digraph) $G = (V, A)$, where:

\begin{itemize}
  \item $V$ is the set of nodes representing the airspace and virtual drone locations.
  \item $A$ is the set of directed arcs representing the permitted transitions between nodes.
\end{itemize}

The temporal component is omitted here as we are primarily interested in representing the problem's spatial topology.

\begin{figure}[H]
  \centering
  \includegraphics[width=0.8\textwidth]{img/graph_model.pdf}
  \caption[graph modelling]{Graph modelling. Source: The authors.}
  \label{fig:graph_model}
\end{figure}



Figure~\ref{fig_graph} depicts a simplified 2D grid-shaped example of a shared airspace (without altitude) for illustrative purposes. The dotted rectangular area represents the physical 2D airspace.

\subsubsection{Virtual Nodes for Drones}

In addition to the shared airspace, we introduce two virtual nodes for each drone $k$:

\begin{itemize}
  \item A source node $b_k$: This represents the starting point of drone $k$'s mission.
  \item A sink node $e_k$: This represents the ending point of drone $k$'s mission.
\end{itemize}


Each drone $k$ must initiate its mission at $b_k$ and conclude it at $e_k$. These virtual nodes streamline the model by circumventing the need to explicitly handle individual drone start and end positions.

Here's a breakdown of the virtual node connections:

\begin{itemize}
  \item Drone $k$ can only remain in $b_k$ or exit it to the shared airspace.
  \item Once exited, no drone (including $k$ itself) can re-enter $b_k$.
  \item A priori, any drone $k$ can enter its designated sink node $e_k$. However, constraints will be formulated in the mathematical programming section to ensure only drone $k$ enters its corresponding $e_k$.
\end{itemize}

\subsubsection{Drone Loitering}

Since we're dealing with rotary-wing vehicles, the model allows drones to remain (loiter) at a node for a certain time. To accommodate this, we include a loop arc for every node in the graph, encompassing both physical and virtual source/sink nodes. For clarity, these loops are implicitly considered in the example of Figure~\ref{fig_graph} and not explicitly depicted.

Including loops simplifies the mathematical programming model by avoiding the need to address potential exceptions.

\subsection{Parameters}

Below are listed the parameters that are constants of the problem:

\begin{itemize}
\item $T$: maximum time allowed for the mission (limits drone movement);
\item $b_k$: initial virtual vertex representing the initial position (begin) of drone $k$;
\item $e_k$: final virtual vertex representing the final position (end) of drone $k$;
\item $\mathcal{R}$: set of drones;
\item $\mathcal{G}$: digraph $(\mathcal{V}, \mathcal{A})$ representing the airspace;
\item $\mathcal{V}$: set of vertices of $\mathcal{G}$ representing each cell of the airspace or virtual vertex;
\item $\mathcal{B} \subset \mathcal{V}$: set of initial virtual vertices $b_k$;
\item $\mathcal{E} \subset \mathcal{V}$: set of final virtual vertices $e_k$;
\item $\mathcal{S}$: set $\mathcal{V} \setminus (\mathcal{B} \cup \mathcal{E})$ of vertices exclusively representing the shared airspace, i.e., eliminating virtual vertices;
\item $\mathcal{A}$: set of arcs $(i,j) \in \mathcal{A}$ of $\mathcal{G}$ connecting airspace cells to each other or a virtual vertex to the airspace.
\end{itemize}

\subsection{Indices}

In the equations, we use the following iteration indices:

\begin{itemize}
\item $k$ : drone $\implies$ $k \in \mathcal{R}$;
\item $t$ : time $\implies$ $1 \leq t \leq T $;
\item $i,j,l$ : vertices $\implies$ $i, j, l \in \mathcal{V}$.
\end{itemize}

\subsection{Decision Variables}

Below are listed the decision variables of the model:

\begin{itemize}
\item $x_{i,j,t}^k = 1 \iff$ drone $k$ jumps from $i$ to $j$ at time $t$.
\end{itemize}


\subsection{Objective Function}

We choose an objective function that minimizes the total sum of the number of drone movements,
\begin{equation}\label{eq:fobj}
\min
\sum_{k \in \mathcal{R}}
\sum_{t=1}^T
\sum_{ \; (i,j) \in \mathcal{A}:\ j \notin (\mathcal{E} \cup \mathcal{B})} x_{i,j,t}^{k}\text{,}
\end{equation}
counting $n-1$ jumps for each drone that performs $n$ jumps, since the jump to the final virtual vertex cannot be counted as a vertex with cost for the drone mission. Loops at initial virtual vertices are also disregarded.

\subsection{Constraints}

The set of constraints are described in the following.

\subsubsection{Movement Start}

We require that each drone starts its mission, \begin{equation}\label{eq:restr_inicio}
\sum_{t=1}^{T}
\sum_{j \in \mathcal{S}}
x_{b_k,j,t}^k = 1, \quad \forall k \in \mathcal{R}
\end{equation},  by jumping from its virtual initial node $b_k$ to the shared airspace.

\subsubsection{Flow Conservation}

 The flow conservation is also addressed,  \begin{equation}\label{eq:restr_fluxo}
\sum_{j \in \mathcal{V}} x_{i,j,t-1}^{k} =
\sum_{l \in \mathcal{V}} x_{j,l,t}^{k}, \quad
\forall j \in \mathcal{V}, \forall k \in \mathcal{R}, \forall t \in \{2, \ldots, T\} \; \text{,}
\end{equation} ensuring the consistency of allowed movements defined by the graph $\mathcal{G}$

\subsubsection{Border Condition}

Initially, under the border condition at time $t=0$, we consider there is a loop flow of drones at the virtual initial vertices.
This flow is redirected into the problem by the flow conservation constraint. \begin{equation}\label{eq:restr_borda}
    x_{i,j,0}^k = \left\{
    \begin{matrix}
        1, & \text{if}\ i=b_k \land j=b_k,\\
        0, & \text{otherwise}.
    \end{matrix}
    \right.
    \quad \forall k \in \mathcal{R}, \forall (i,j) \in A  \; \text{.}
\end{equation} For all other variables at time $t=0$, we consider a flow equal to zero.

\subsubsection{Mutual exclusion of vertex occupation}

We ensure that no drones occupy the same position,

\begin{equation}\label{eq:restr_ocupacao}
\sum_{k \in \mathcal{R}}
\sum_{j \in \mathcal{V}}
x_{i,j,t}^{k} \leq 1, \quad \forall j \in \mathcal{V}, \forall t \in \{1, \ldots, T\} \; \text{,}
\end{equation} that is, each vertex $j \in \mathcal{V}$ is occupied by at most one drone.

\subsubsection{Mission Accomplishment}

Every drone $k$ needs to complete its mission, 

\begin{equation}\label{eq:restr_fim}
\sum_{t=1}^{T}
\sum_{i \in \mathcal{S}}
x_{i,e_k,t}^k \geq 1, \quad \forall k \in \mathcal{R} \; \text{,}
\end{equation} i.e., reaching its final virtual vertex $e_k$ is required.


% \section[Heuristic]{Heuristic}
\label{Heuristic}

In this section, we present our heuristic approach for tackling the problem, focusing on the utilization of a distance measure as a heuristic metric to organize drones in ascending order(prioritized planning) based on their start and end points. This prioritized order is then employed in an iterative Breadth-First Search (BFS) on the temporal graph. The BFS process dynamically updates the constraints of occupied positions(conflict-based search), creating a sequential refinement of the temporal graph with each newly generated minimum path. By combining heuristic sorting and iterative BFS, our approach aims to efficiently navigate the solution space, providing a balance between effective path planning and adaptability to evolving constraints in the context of drone delivery.


MAPF algorithms are recognized for their versatility with various distance metrics \cite{Weise2023}. In our specific scenario, we chose the Euclidean Distance as our metric of choice. This decision stems from its efficacy as a tiebreaker for drones that traverse the same number of cells in the minimum path, as observed with the Manhattan Distance. Notably, drones may share the same Manhattan Distance but diverge in Euclidean Distance, particularly when paths involve changes in direction. For instance, drones following a straight-line trajectory in the minimum path encounter fewer potential movement positions, while paths with changes in axis may introduce additional possibilities, making Euclidean Distance a valuable discriminant in such cases. This strategic use of the Euclidean Distance enhances the overall performance and adaptability of our algorithm in navigating complex drone delivery scenarios, as exemplified in \ref{ssec:example}, especially in scenarios with high traffic congestion where having more possibilities is crucial for finding local optimal solutions.

While conflict-based search has exponential complexity \cite{DBLP:conf/socs/GordonFS21}, our algorithm has polynomial complexity in function of size of the grid and number of drones. However, we cannot guarantee optimality, and this is expected since, even in the 2D grid case, MAPF where each agent has three or more possible directions of movement is also NP-hard\cite{geft2023finegrained}.


\begin{table}[H]
  \centering
  \caption{Notation used in the Algorithm.}
  \label{tab:notation}
  \begin{tabular}{ll}
    \toprule
    Notation & Definition \\
    \midrule
    $\mathcal{V}$ &Set of vertices in the graph $: (i,j,t)$ \\
    $\mathcal{E}$ & Set of edges  \\
    $\mathcal{D}$ & Set of drones \\
    $\mathcal{S}$ & Set of already scheduled vertices \\
    $\mathcal{P}_d \subseteq \mathcal{V} $ & Path of drone $d$ \\
    $\mathcal{G} = (\mathcal{V},\mathcal{E}) $ & Temporal Graph \\
    \bottomrule
  \end{tabular}
\end{table}

Formally, using the notation in Table \ref{tab:notation}, we can create a temporal graph $\mathcal{G}$ that naturally represents our problem. Since the set of vertex $\mathcal{V}:= (i,j,t)$ represents every possible position $(i,j)$ at any given time $t$ in our problem.  This modeling approach streamlines the search for optimal solutions in an efficient manner. The representation allows us to capture the possibility of a drone waiting at its position at a specific time $t_{wait}$ through the graph edges. We achieve this by defining the set of outgoing edges for a vertex $v^*=(i^*,j^*,t^*$ as $\mathcal{E}_{v^*} = \{ (i^*+1,j^*,t^*+1), (i^*,j^*+1,t^*+1), (i^*-1,j^*,t^*+1), (i^*,j^* -1,t^*+1), (i^*,j^*,t^*+1) \}$. The edge $(i^*,j^*,t^*+1)$ means that the drone chooses a path in which it decides to wait a time step. We define our algorithm sequentially as follows:

\begin{enumerate}

   \item \textbf{Drones Sorting}: We begin by ascending sorting the drones using the Euclidean Distance as a heuristic, given by \( \sqrt{(x_{\text{begin}} - x_{\text{end}})^2 + (y_{\text{begin}} - y_{\text{end}})^2} \), \begin{equation}
        \mathcal{D}_{\text{sorted}} = \text{sort}(\mathcal{D}, \text{heuristic}) \; \text{.}
    \end{equation}
    
\item \textbf{Path for Each Drone}: For each drone \(d\) in the sorted order, we compute the path \(P_d\) using Breadth-First Search (BFS) on the graph \(\mathcal{G}\). The BFS is performed with the constraints imposed by the set \(\mathcal{S}\), \begin{equation}
        \forall d \in \mathcal{D}_{\text{sorted}}: \quad \mathcal{P}_d = \text{BFS}(\mathcal{G}, d) \; \text{.}
    \end{equation}
    
    \item \textbf{Constraints Update}: After determining paths for the respective drone, we update the set of already scheduled vertices \(\mathcal{S}\) immediately following the execution of each BFS sequentially. This update involves incorporating the vertices covered by the path into the existing set of constraints (previously scheduled paths), \begin{equation}
        \mathcal{S} = \mathcal{S} \cup \bigcup_{d \in \mathcal{D}_{\text{sorted}}} \mathcal{P}_d \; \text{.}
    \end{equation}

   Indeed, the \textbf{constraints} ensure that no two drones occupy the same vertex at the same time, \begin{equation}
    \forall d, d' \in \mathcal{D}, d \neq d', \forall (i, j, t) \in \mathcal{P}_d, (i, j, t) \notin \mathcal{P}_{d'} \; \text{.}
\end{equation}

    
\end{enumerate}

In the computational aspect, our implementation follows the standard implementation of BFS in 3D grids, since the time $t$ can be simply thought as a third dimension. The main difference of common implementations is the addition of a \textit{map}, that is a Red Black Tree, to manage the set of scheduled positions. This addition adds $\mathcal{O}(\log n)$ in the complexity of our algorithm. The pseudo algorithm of the heuristic is described in Algorithm \ref{alg:grid-bfs}.


\begin{algorithm}[H]
    \label{alg:grid-bfs}
    \DontPrintSemicolon
    \KwData{Grid dimensions $N$ (rows) and $M$ (columns), Number of drones $K$, List of drones \textit{drones}}
    \KwResult{Paths for each drone considering constraints}

    \BlankLine
    \ForEach{\textit{drone} in \textit{drones}}{
        Initialize data structures for BFS: \textit{bfs\_queue}, \textit{parent}, \textit{visited}\;
        Set \textit{flight\_time} of \textit{drone} to -1\;
        \While{No valid path found for \textit{drone}}{
            Increment \textit{flight\_time} of \textit{drone}\;
            Enqueue \textit{drone\_begin} with \textit{flight\_time} into \textit{bfs\_queue}\;
            \While{\textit{bfs\_queue} is not empty}{
                Dequeue a position and time \;
                \If{Position is the destination of \textit{drone}}{
                    Reconstruct and schedule path for \textit{drone}\;
                    \Return path\;
                }
                \If{Position is scheduled}{
                \textit{\textbf{Continue}}
                    \;
                }
                \For{Neighbor positions}{
                    \If{Position is valid and not visited}{
                        \If{Position is scheduled and no schedule discovered in neighbors yet}{
                            
                            Enqueue current position and \textit{flight\_time + 1} \;
                            
                            \textit{\textbf{Continue}} \;
                        }
                        Mark position as visited\;
                        Store parent information\;
                        Enqueue the neighbor position into \textit{bfs\_queue}\;
                    }
                }
            }
        }
    }
    \caption{Path Planning for Drones using BFS}
\end{algorithm}


\subsection{Example} \label{ssec:example}

\begin{figure}[h]
    \centering
    \begin{tikzpicture}[scale=.8, every node/.style={minimum size=1cm}]

        % Define quadcopter style
        \tikzset{
            quadcopter/.style={
                quadcopter side,
                fill=white,
                draw=gray,
                minimum width=1cm,
                below=of second-scope, 
            },
        }
        
        % Add another drone
        \node (quadcopterGreen) [quadcopter side, fill=white, draw=greenMW, minimum width=1cm, rotate=0] at (5,1) {};

        \node (quadcopterBrown) [quadcopter side, fill=white, draw=brown, minimum width=1cm, rotate=0] at (5,-11.3) {};

        \node (quadcopterOrange) [quadcopter side, fill=white, draw=orange, minimum width=1cm, rotate=0] at (4.4,-14.3) {};

        \node (quadcopterBlue) [quadcopter side, fill=white, draw=blue, minimum width=1cm, rotate=0] at (-5,-8.2) {};
        
        % Último Escopo (t=6)
        \begin{scope}[yshift=-423, every node/.append style={yslant=0.5, xslant=-1}, yslant=0.5, xslant=-1]

            

            \fill[white, fill opacity=0.9] (0,0) rectangle (5,5);
            
            \draw[step=4mm, black] (0,0) grid (5,5);
            \draw[black, thick] (0,0) rectangle (5,5); % Borders
            \fill[orange] (2.05,2.05) rectangle (2.35,2.35); % center pixel
            \node (centerBegint2) at (2.2,2.2) {B}; % place 'B' in the center
            % \fill[greenMW] (1.65,2.05) rectangle (1.95,2.35); % left
            % \fill[greenMW] (2.45,2.05) rectangle (2.75,2.35); % right
            \fill[orange] (2.05,1.95) rectangle (2.35,1.65); % bottom
            \fill[orange,yshift=-11.2] (2.05,1.95) rectangle (2.35,1.65); % bottom
            \fill[orange,yshift=-22.5] (2.05,1.95) rectangle (2.35,1.65); % bottom
            \fill[orange,yshift=-33.5] (2.05,1.95) rectangle (2.35,1.65); % bottom
            
            %\fill[greenMW] (2.05,2.45) rectangle (2.35,2.75); % top
            % 8 -pixel setting
           
            % \fill[greenMW] (2.75,1.95) rectangle (2.45,1.65); % bottom-right
            % \fill[greenMW] (1.65,1.95) rectangle (1.95,1.65); % bottom-left
            
            \node (landedOrange) at (1.5,0.8) {};

            \node(pathdestinyt6) [black, font=\bfseries] at ([yshift=-45.3]centerBegint2) {E};
        \end{scope}



        % Penúltimo Escopo (t=5)
        \begin{scope}[yshift=-340, every node/.append style={yslant=0.5, xslant=-1}, yslant=0.5, xslant=-1]
            \fill[white, fill] (0,0) rectangle (5,5);
            \draw[step=4mm, black] (0,0) grid (5,5);
            \draw[black, thick] (0,0) rectangle (5,5); % Borders
            \fill[orange!55] (2.05,2.05) rectangle (2.35,2.35); % center pixel
            \node (centerBegint3) at (2.2,2.2) {B}; % place 'B' in the center

            \fill[orange!55,xshift=-22.6] (2.45,2.05) rectangle (2.75,2.35); % most left
            

            \node (landedBrown) at (1.5,0.6) {};

            %\fill[blue] (1.65,2.05) rectangle (1.95,2.35); % left
            % left X
            \draw[orange] (1.65,2.05) -- (1.95,2.35);
            \draw[orange] (1.65,2.35) -- (1.95,2.05);
            % left bottom X
            \draw[orange, yshift=-11.3] (1.65,2.05) -- (1.95,2.35);
            \draw[orange, yshift=-11.3] (1.65,2.35) -- (1.95,2.05);   
            % most left X
            \draw[orange, xshift=-11.3] (1.65,2.05) -- (1.95,2.35);
            \draw[orange, xshift= -11.3] (1.65,2.35) -- (1.95,2.05);   

            \fill[orange!55,xshift=11.3] (2.45,2.05) rectangle (2.75,2.35); % most right
            % most right X
            \draw[orange, xshift=34.4] (1.65,2.05) -- (1.95,2.35);
            \draw[orange, xshift=34.4] (1.65,2.35) -- (1.95,2.05);

            % bottom center right X
            \draw[orange, xshift=34.4, yshift= -11.3] (1.65,2.05) -- (1.95,2.35);
            \draw[orange, xshift=34.4, yshift= -11.3] (1.65,2.35) -- (1.95,2.05);

            %  top center right X
            \draw[orange, xshift=34.4, yshift= 11.3] (1.65,2.05) -- (1.95,2.35);
            \draw[orange, xshift=34.4, yshift= 11.3] (1.65,2.35) -- (1.95,2.05);
            
            % most most right X
            \draw[orange, xshift=34.4 + 11.3] (1.65,2.05) -- (1.95,2.35);
            \draw[orange, xshift=34.4 + 11.3] (1.65,2.35) -- (1.95,2.05);

            

            \fill[orange!55] (2.45,2.05) rectangle (2.75,2.35); % right
            
            \fill[orange!55] (2.05,2.45) rectangle (2.35,2.75); % top
            \fill[orange!55, xshift=11.3] (2.05,2.45) rectangle (2.35,2.75); % top right
            \fill[orange!55, xshift=-11.3] (2.05,2.45) rectangle (2.35,2.75); % top left
            \fill[orange!55, yshift=11.3] (2.05,2.45) rectangle (2.35,2.75); % most top
            \fill[orange!55] (2.05,1.95) rectangle (2.35,1.65); % bottom

            \fill[orange!55, xshift= 11.3] (2.05,1.95) rectangle (2.35,1.65); % bottom right

            \fill[orange!55,yshift=-11.3] (2.05,1.95) rectangle (2.35,1.65); % most bottom
            
            %\fill[greenMW] (1.65,2.45) rectangle (1.95,2.75); % top-left 
            % top left X
            \draw[orange] (1.65,2.75) -- (1.95,2.45);
            \draw[orange] (1.65,2.45) -- (1.95,2.75);
            
            % top most left X
            \draw[orange, xshift=-11.3] (1.65,2.75) -- (1.95,2.45);
            \draw[orange, xshift=-11.3] (1.65,2.45) -- (1.95,2.75);
            
            % \draw[orange,xshift=-11.3] (1.65,2.75) -- (1.95,2.45);
            % \draw[orange,xshift=-11.3] (1.65,2.45) -- (1.95,2.75);   

            %\fill[greenMW] (2.45,2.45) rectangle (2.75,2.75); % top-right
            \draw[orange] (2.45,2.45) -- (2.75,2.75);
            \draw[orange] (2.45,2.75) -- (2.75,2.45);

            %\fill[greenMW] (2.75,1.95) rectangle (2.45,1.65); % bottom-right
            \draw[orange] (2.45,1.95) -- (2.75,1.65);
            \draw[orange] (2.45,1.65) -- (2.75,1.95);

            % top most X 
            \draw[orange,xshift= -34 +45.4,yshift=11] (1.65,2.75) -- (1.95,2.45);
            \draw[orange,xshift=-34 +45.4,yshift=11] (1.65,2.45) -- (1.95,2.75);

            % top most right X 
            \draw[orange,xshift= -34 +45.4 + 11.2,yshift=11] (1.65,2.75) -- (1.95,2.45);
            \draw[orange,xshift=-34 +45.4 + 11.2,yshift=11] (1.65,2.45) -- (1.95,2.75);

            % top most left X 
            \draw[orange,xshift= -34 +45.4 - 11.2,yshift=11] (1.65,2.75) -- (1.95,2.45);
            \draw[orange,xshift=-34 +45.4 - 11.2,yshift=11] (1.65,2.45) -- (1.95,2.75);

            % top most last X 
            \draw[orange,xshift= -34 +45.4 ,yshift=22.2] (1.65,2.75) -- (1.95,2.45);
            \draw[orange,xshift=-34 +45.4 ,yshift=22.2] (1.65,2.45) -- (1.95,2.75);

            % bottom most X
            \draw[orange,xshift= -34 +45.4,yshift=-34.4] (1.65,2.75) -- (1.95,2.45);
            \draw[orange,xshift=-34 +45.4,yshift=-34.4] (1.65,2.45) -- (1.95,2.75);

            % bottom most right X
            \draw[orange,xshift= -34 +45.4 +11.2 ,yshift=-34.4] (1.65,2.75) -- (1.95,2.45);
            \draw[orange,xshift=-34 +45.4 + 11.2,yshift=-34.4] (1.65,2.45) -- (1.95,2.75);

            % bottom most left X
            \draw[orange,xshift= -34 +45.4 -11.2 ,yshift=-34.4] (1.65,2.75) -- (1.95,2.45);
            \draw[orange,xshift=-34 +45.4  -11.2,yshift=-34.4] (1.65,2.45) -- (1.95,2.75);

            % bottom most last X
            \draw[orange,xshift= -34 +45.4,yshift=-45.8] (1.65,2.75) -- (1.95,2.45);
            \draw[orange,xshift=-34 +45.4,yshift=-45.8] (1.65,2.45) -- (1.95,2.75);

                    

            %\fill[blue] (1.65,1.95) rectangle (1.95,1.65); % bottom-left
            % 2. ring
            %\fill[greenMW] (1.25,1.55) rectangle (1.55,1.25); % bottom-left
            %\fill[greenMW] (0.85,1.55) rectangle (1.15,1.25); % bottom-left
            %\fill[greenMW] (0.85,1.15) rectangle (1.15,0.85); % bottom-left

            \fill[brown, xshift=22.7, yshift=-11.2] (1.25,0.75) rectangle (1.55,0.45); % bottom-left

            \node(pathdestinyt3) [black, font=\bfseries] at ([yshift=-45.3]centerBegint3) {E};
        \end{scope}


        % Escopo (t=4)
        \begin{scope}[yshift=-250, every node/.append style={yslant=0.5, xslant=-1}, yslant=0.5, xslant=-1]
            \fill[white, fill] (0,0) rectangle (5,5);
            \draw[step=4mm, black] (0,0) grid (5,5);
            \draw[black, thick] (0,0) rectangle (5,5); % Borders
            \fill[orange!55] (2.05,2.05) rectangle (2.35,2.35); % center pixel
            \node (centerBegint3) at (2.2,2.2) {B}; % place 'B' in the center
            
            %\fill[blue] (1.65,2.05) rectangle (1.95,2.35); % left
            \draw[orange] (1.65,2.05) -- (1.95,2.35);
            \draw[orange] (1.65,2.35) -- (1.95,2.05);   

            \node (landedBlue) at (1.1,2.3) {};

            % most right X
            \draw[orange, xshift=34.4] (1.65,2.05) -- (1.95,2.35);
            \draw[orange, xshift=34.4] (1.65,2.35) -- (1.95,2.05);  

            \fill[orange!55] (2.45,2.05) rectangle (2.75,2.35); % right
            \fill[orange!55] (2.05,2.45) rectangle (2.35,2.75); % top
            \fill[orange!55] (2.05,1.95) rectangle (2.35,1.65); % bottom
            
            %\fill[greenMW] (1.65,2.45) rectangle (1.95,2.75); % top-left
            \draw[orange] (1.65,2.75) -- (1.95,2.45);
            \draw[orange] (1.65,2.45) -- (1.95,2.75);   
            
            % \draw[orange,xshift=-11.3] (1.65,2.75) -- (1.95,2.45);
            % \draw[orange,xshift=-11.3] (1.65,2.45) -- (1.95,2.75);   

            %\fill[greenMW] (2.45,2.45) rectangle (2.75,2.75); % top-right
            \draw[orange] (2.45,2.45) -- (2.75,2.75);
            \draw[orange] (2.45,2.75) -- (2.75,2.45);

            %\fill[greenMW] (2.75,1.95) rectangle (2.45,1.65); % bottom-right
            \draw[orange] (2.45,1.95) -- (2.75,1.65);
            \draw[orange] (2.45,1.65) -- (2.75,1.95);

            % top most X 
            \draw[orange,xshift= -34 +45.4,yshift=11] (1.65,2.75) -- (1.95,2.45);
            \draw[orange,xshift=-34 +45.4,yshift=11] (1.65,2.45) -- (1.95,2.75);

            % bottom most X
            \draw[orange,xshift= -34 +45.4,yshift=-34.4] (1.65,2.75) -- (1.95,2.45);
            \draw[orange,xshift=-34 +45.4,yshift=-34.4] (1.65,2.45) -- (1.95,2.75);

            \fill[blue] (1.65,1.95) rectangle (1.95,1.65); % bottom-left
            % 2. ring
            %\fill[greenMW] (1.25,1.55) rectangle (1.55,1.25); % bottom-left
            %\fill[greenMW] (0.85,1.55) rectangle (1.15,1.25); % bottom-left
            %\fill[greenMW] (0.85,1.15) rectangle (1.15,0.85); % bottom-left

            \fill[brown, xshift=22.6] (1.25,0.75) rectangle (1.55,0.45); % bottom-left

            \node(pathdestinyt3) [black, font=\bfseries] at ([yshift=-45.3]centerBegint3) {E};
        \end{scope}

        % Terceiro Escopo (t=3)
        \begin{scope}[yshift=-166, every node/.append style={yslant=0.5, xslant=-1}, yslant=0.5, xslant=-1]
            \fill[white, fill opacity=0.9] (0,0) rectangle (5,5);
            \draw[step=4mm, black] (0,0) grid (5,5);
            \draw[black, thick] (0,0) rectangle (5,5); % Borders
            \fill[orange!55] (2.05,2.05) rectangle (2.35,2.35); % lighter orange for center pixel
            \node (centerBegint3) at (2.2,2.2) {B}; % place 'B' in the center
            \fill[blue] (1.65,2.05) rectangle (1.95,2.35); % left
            \draw[orange] (2.45,2.05) -- (2.75,2.35);
            \draw[orange] (2.45,2.35) -- (2.75,2.05);
            %\fill[greenMW] (2.05,2.45) rectangle (2.35,2.75); % top
            \draw[orange] (2.05,2.45) -- (2.35,2.75);
            \draw[orange] (2.05,2.75) -- (2.35,2.45);

            %\fill[greenMW] (2.05,1.95) rectangle (2.35,1.65); % bottom
            \draw[orange] (2.05,1.95) -- (2.35,1.65);
            \draw[orange] (2.05,1.65) -- (2.35,1.95);

            
            %\fill[greenMW] (1.65,2.45) rectangle (1.95,2.75); % top-left
            %\fill[greenMW] (2.45,2.45) rectangle (2.75,2.75); % top-right
            %\fill[greenMW] (2.75,1.95) rectangle (2.45,1.65); % bottom-right
            %\fill[greenMW] (1.65,1.95) rectangle (1.95,1.65); % bottom-left
            % 2. ring
            %\fill[greenMW] (1.25,1.55) rectangle (1.55,1.25); % bottom-left
            %\fill[greenMW] (0.85,1.55) rectangle (1.15,1.25); % bottom-left
            %\fill[greenMW] (0.85,1.15) rectangle (1.15,0.85); % bottom-left

            \fill[brown, xshift=11.3] (1.25,0.75) rectangle (1.55,0.45); % bottom-left

            \node(pathdestinyt3) [black, font=\bfseries] at ([yshift=-45.3]centerBegint3) {E};

            \node(midpointX) at (2.6, 2.2){};
            
        \end{scope}

        % Escopo do meio (t=2)
        \begin{scope}[yshift=-83, every node/.append style={yslant=0.5, xslant=-1}, yslant=0.5, xslant=-1]

            \node [quadcopter side,fill=white,draw=orange,minimum width=1cm,rotate=10] (quadcoptert2) at (1,7) {};

            \fill[white, fill opacity=0.9] (0,0) rectangle (5,5);

            \node (arrivalOrange) at (1.8,2.1) {};
            
            \draw[step=4mm, black] (0,0) grid (5,5);
            \draw[black, thick] (0,0) rectangle (5,5); % Borders
            \fill[orange] (2.05,2.05) rectangle (2.35,2.35); % center pixel
            \node (centerBegint2) at (2.2,2.2) {B}; % place 'B' in the center
            % \fill[greenMW] (1.65,2.05) rectangle (1.95,2.35); % left
            % \fill[greenMW] (2.45,2.05) rectangle (2.75,2.35); % right
            \fill[greenMW] (2.05,1.95) rectangle (2.35,1.65); % bottom
            \node (landedGreen) at (1.8,1.9) {};
            %\fill[greenMW] (2.05,2.45) rectangle (2.35,2.75); % top
            % 8 -pixel setting
            \fill[blue] (1.65,2.45) rectangle (1.95,2.75); % top-left
            % \fill[greenMW] (2.45,2.45) rectangle (2.75,2.75); % top-right
            % \fill[greenMW] (2.75,1.95) rectangle (2.45,1.65); % bottom-right
            % \fill[greenMW] (1.65,1.95) rectangle (1.95,1.65); % bottom-left
            % 2. ring
            % \fill[greenMW] (1.25,1.55) rectangle (1.55,1.25); % bottom-left
            % \fill[greenMW] (0.85,1.55) rectangle (1.15,1.25); % bottom-left
            % \fill[greenMW] (0.85,1.15) rectangle (1.15,0.85); % bottom-left
            \fill[brown] (1.25,0.75) rectangle (1.55,0.45); % bottom-left
            \node(pathdestinyt2) [black, font=\bfseries] at ([yshift=-45.3]centerBegint2) {E};
        \end{scope}

        % Primeiro Escopo (t=1)
        \begin{scope}[yshift=0, every node/.append style={yslant=0.5, xslant=-1}, yslant=0.5, xslant=-1]

           \node [quadcopter side,fill=white,draw=orange,minimum width=1cm,rotate=10] (quadcoptert1) at (3,7.6) {};
            
           \node (orangeTakeOff) at (2.5,1.8) {};
        
            \fill[white, fill opacity=0.9] (0,0) rectangle (5,5);
            \draw[step=4mm, black] (0,0) grid (5,5); % Grid definition
            \draw[black, thick] (0,0) rectangle (5,5); % Borders
            \fill[greenMW] (2.05,2.05) rectangle (2.35,2.35); % center pixel
            \draw[red, thin] (2.05,2.05) rectangle (2.35,2.35); % border
            \node (centerBegint1) at (2.2,2.2) {B}; % place 'B' in the center
            \fill[blue] (2.05,2.45) rectangle (2.35,2.75); % top
            % \fill[blue] (1.65,2.45) rectangle (1.95,2.75); % top-left
            % \fill[blue] (2.45,2.45) rectangle (2.75,2.75); % top-right
            % \fill[blue] (2.75,1.95) rectangle (2.45,1.65); % bottom-right
            % \fill[blue] (1.65,1.95) rectangle (1.95,1.65); % bottom-left
            % % 2. ring
            % \fill[blue] (1.25,1.55) rectangle (1.55,1.25); % bottom-left
            % \fill[blue] (0.85,1.55) rectangle (1.15,1.25); % bottom-left
            % \fill[blue] (0.85,1.15) rectangle (1.15,0.85); % bottom-left
            % \fill[blue] (1.25,0.75) rectangle (1.55,0.45); % bottom-left

            \fill[brown, xshift= -11.3] (1.25,0.75) rectangle (1.55,0.45); % bottom-left
            
            % Insert 'E' three squares down the center
            \node(pathdestinyt1) [black, font=\bfseries] at ([yshift=-45.3]centerBegint1) {E};
            
        \end{scope}

        % Draw annotations
        \draw[-latex, thick, gray, dashed] (-5.4,-3.5) node[left] {occupied} to[out=0, in=200] (-0.5,-4.1);

        \draw[-latex, thick, red] (quadcoptert1) to[out=35, in=90] node[midway, above] {failed takeoff} (orangeTakeOff);

        \draw[-latex, thick, orange] (quadcoptert2) to[out=0, in=10] node[very near start, above] {arrival} (arrivalOrange);

        %\draw[-latex, thick, orange] ([xshift=-10]centerBegint3) to[out=0, in=10] (midpointX);

        \draw[thick, green!60!black, dashed] (landedGreen) to[out=0, in=10] node[above, pos=0.45 , font=\footnotesize] {landed} (quadcopterGreen);

        \draw[thick, brown!, dashed] (landedBrown) to[out=0, in=10] node[ above, pos=1 , font=\footnotesize] {landed} (quadcopterBrown);

        \draw[thick, orange!, dashed] (landedOrange) to[out=0, in=10] node[ above, pos=0.75 , font=\footnotesize] {path found} (quadcopterOrange);

        \draw[thick, blue!60!white, dashed] (landedBlue) to[out=0, in=13] node[ above, pos=0.85 , font=\footnotesize] {landed} (quadcopterBlue);

        % Annotations
        \draw[thick, gray!60!black] (6,4) node {t=1};
        \draw[thick, gray!60!black] (6,0) node {t=2};
        \draw[thick, gray!60!black] (6,-3) node {t=3};
        \draw[thick, gray!60!black] (6,-6) node {t=4};
        \draw[thick, gray!60!black] (6,-9) node {t=5};
        \draw[thick, gray!60!black] (6.2,-12.4) node {t=6};
    \end{tikzpicture}
    \caption{Algorithm visualization. Source: The authors.}
    \label{fig:visualgo}
\end{figure}


In Figure \ref{fig:visualgo}, the algorithm visualization shows its ability to address corner cases, including take-off in occupied positions and collision avoidance.

The depicted scenario involves determining the route for the orange drone within a grid where blue, brown, and green drones have already established their scheduled routes. These pre-existing paths serve as constraints in the optimization problem.

Upon closer inspection of Figure \ref{fig:visualgo}, it is evident that both the orange and brown drones cover an equal number of positions in the grid. However, an important factor influencing the algorithm's decision-making is a heuristic chosen for prioritization. In this case, the heuristic involves considering the shorter Euclidean distance covered by the brown drone. Consequently, during the sorting process, the algorithm prioritizes the brown drone based on this chosen heuristic, irrespective of the equality in the number of positions traversed.



\subsection{Complexity Analysis and Boundedness}
\label{secc:complexity_analysis}

Given $K$ the number of drones and
$N, M$ the grid sizes, the number of rows, and the number of columns, respectively. We can analyze our algorithm regarding these inputs using asymptotic computational complexity.

The complexity of our algorithm is the worst case of a BFS in the temporal graph, considering the $\mathcal{O}(\log S)$ from the scheduled structure, where $S= N M T$ is the size of the search space. It can be seen that this would be $\mathcal{O}(N M T)\mathcal{O}(\log N M T)$, where $T$ is the maximum time a drone lands. However, $T$ is surely bounded by $\mathcal{O}(K N M)$. In addition, it can be shown that $T$ is bounded by $\mathcal{O}((N+M) K)$. An intuitive view of this is shown in \ref{fig:worst_path}, where we have three drones on a $3\times 4$ grid with the same initial position($s_i$) and final position($g_i$), where $i=1\dots3$. In this case, the shorter path for each drone is $n+m$,$n=3;m=4$, and in the worst case, we will wait for each drone to complete its path, thus $k(n+m)$ is an upper bound for our heuristic since our heuristic always chooses a shorter path than $k(n+m)$ because we do not wait for each drone to complete the path.   Then the complexity is bounded by $$\mathcal{O}((N+M) K N M \log((N+M) K N M))= \mathcal{O}(\max{(N,M)} N M K  \log(\max{(N,M)} N M K)) $$  $\approx \mathcal{O}(N^3  K \log(N^3 K) = \mathcal{O}(N^3  K \max{(\log N , \log K))} $, if the grid is square. 



\begin{figure}[H]
    \centering
    \includegraphics[width=0.8\textwidth]{img/worst_path.drawio.pdf}
    \caption{Worst case path.}
    \label{fig:worst_path}
\end{figure}



\subsection{Adaptability for 3D}

A notable difference from our method to others in MAPF is that our algorithm remains consistent across any grid dimension. That is, the algorithm for a 2D grid obtains solutions in the same way as for a 3D or 4D grid.

This fact becomes clear when we see that both our methods, the heuristic and the exact model, are based solely on the topology of the graph. There is no spatial dependency. When we increase the dimension to 3D, we are simply increasing the neighborhood of each node. In some sense, each vertex $v$ in 3D space will have two additional edges in its neighborhood (up and down).

Again, using the notation in Table \ref{tab:notation}, we can create a temporal graph $\mathcal{G}$ that naturally represents our problem in three dimensional space. The set of vertices $\mathcal{V} := (i,j,k,t)$ represents every possible position $(i,j,k)$ at any given time $t$ in our problem. Our definition is almost the same, the difference is we add one more index $k$ and consequently the possibility of the drone going up($k+1$) and down($k-1$). We achieve this by defining the set of outgoing edges for a vertex $v^* = (i^*,j^*,k^*,t^*)$ as $\mathcal{E}_{v^*} = \{ 
(i^*+1,j^*,k^*,t^*+1), 
(i^*,j^*+1,k^*,t^*+1), 
(i^*-1,j^*,k^*,t^*+1), 
(i^*,j^*-1,k^*,t^*+1), 
(i^*,j^*,k^*+1,t^*+1),
(i^*,j^*,k^*-1,t^*+1),
(i^*,j^*,k^*,t^*+1)
\}$.





% \section{Hybrid Methodology}
% \label{Hybrid_Methodology}





\chapter{Results}
\label{Results}

This chapter presents a comprehensive evaluation of the classification performance for the models applied in this study, with a focus on comparing the CatBoost and Graph Attention Network (GAT) models under varying layer configurations. Given the dataset's imbalance, multiple performance metrics are reported to provide a more nuanced assessment than accuracy alone.

\section{GAT vs. CatBoost Classification Performance}
\label{sec:gat_vs_catboost}

The performance of each model in predicting flight delays due to holding maneuvers was assessed using the following metrics:

\begin{itemize}
    \item \textbf{Accuracy}: Measures the overall proportion of correct predictions across all instances. While accuracy is a useful baseline metric, it can be misleading with imbalanced datasets, as it may remain high even if the model fails to correctly predict instances of the minority class, such as delayed flights.

    \item \textbf{Precision}: Represents the ratio of true positive predictions (correctly predicted delays) to all positive predictions made by the model. Precision provides insight into the model's ability to accurately identify actual delays, where higher precision indicates fewer false positives.

    \item \textbf{Recall}: The ratio of true positive predictions to the total number of actual positive instances (delays) in the dataset. A high recall indicates the model’s effectiveness in capturing most delayed flights, crucial in applications where missing delay predictions could impact operations.

    \item \textbf{F1-Score}: The harmonic mean of precision and recall, which balances these two metrics into a single score. The F1-score is particularly useful for imbalanced datasets, as it considers both false positives and false negatives, providing a balanced measure of performance.
\end{itemize}

The performance of each model configuration across these metrics is shown in Table \ref{tab:gat_catboost_metrics}, highlighting the trade-offs between model complexity (number of GAT layers) and classification effectiveness. Notably, the CatBoost model achieves balanced performance, which is advantageous given the imbalanced dataset.

\begin{table}[!htbp]
	\centering
	\caption{Performance metrics for various GAT layer configurations and CatBoost with graph features.}
	\label{tab:gat_catboost_metrics}
	\begin{tabular}{ccccc}
		\toprule
        \textbf{Model} & \textbf{Test Accuracy} & \textbf{Precision} & \textbf{Recall} & \textbf{F1-Score} \\
		\midrule
        \textbf{CatBoost} & 0.91 & \textbf{0.10} & 0.56 & \textbf{0.17} \\
        \textbf{1 GAT Layer} & \textbf{0.95} & 0.03 & 0.06 & 0.04 \\
        \textbf{3 GAT Layers} & 0.52 & 0.01 & 0.40 & 0.03 \\
        \textbf{5 GAT Layers} & 0.57 & 0.01 & 0.30 & 0.02 \\
        \textbf{10 GAT Layers} & 0.91 & 0.02 & 0.08 & 0.03 \\
        \textbf{30 GAT Layers} & 0.02 & 0.02 & \textbf{0.99} & 0.03 \\
		\bottomrule
	\end{tabular}%
\end{table}


As we can see, Catboost outperforms GAT in terms of precision and F1-Score, while GAT models with more layers tend to have higher recall and with a single layer the higher accuracy. However, the 30-layer GAT model exhibits a significant drop in accuracy, indicating overfitting. The CatBoost model, with a balanced performance across all metrics, is better suited for the imbalanced dataset, as it captures both delayed and non-delayed flights effectively.

Since we are dealing with an unbalanced setting, it was already expected that the CatBoost model would perform better, as it is a decision tree ensemble model that is known for its robustness in handling class imbalance. That is why it had the higher F1-score that is in our problem the most reliable metric, since we want to predict holding and it's better to be wrong some times than always predict `no holding'. The GAT model, on the other hand, as most of GNNs have trouble in dealing with class imbalance, as they tend more to overfitting.

\section{Predictive Regression Analysis and Interpretability}

Beyond classification, we extended our investigation to evaluate the CatBoost model's capacity for regression on continuous delay values and its interpretability. The objective was to assess how effectively CatBoost could predict the extent of flight delays while providing insights into which features were most influential in the model's decision-making. This section explores both aspects—interpretability and regression accuracy—with a focus on the model’s performance and its potential operational impact.

\subsection{Interpretability of the CatBoost Model}

One of the major advantages of the CatBoost model is its transparency, particularly when compared to more complex neural network architectures. By utilizing graph-based features, CatBoost not only achieved favorable classification performance but also provided a clearer view of feature importance. Unlike black-box models, CatBoost can be examined through Explainable AI (XAI) techniques, which help interpret how specific features contribute to predictions.

Figure \ref{fig:catboost_feature_importance} presents the feature importance for the CatBoost model, identifying which graph-based features most significantly influence the prediction of flight delays due to holding maneuvers. This insight confirms the utility of graph-based features in providing a robust predictive foundation and highlights their relative impact on the delay predictions.

\begin{figure}[!htbp]
    \centering
    \includegraphics[width=0.8\textwidth]{img/feature_importance_plot.pdf}
    \caption{Feature importance for the CatBoost model on the airport network dataset, indicating the relevance of graph-based features.}
    \label{fig:catboost_feature_importance}
\end{figure}

\subsection{Regression Performance}

To further assess the model's robustness, we applied CatBoost to a regression task, predicting continuous delay values rather than binary classifications. Figures \ref{fig:y_pred} and \ref{fig:y_test} display the distribution of predicted ($y_\text{pred}$) and actual ($y_\text{test}$) delay values, respectively. By comparing these distributions, we gain insights into CatBoost’s ability to capture trends and variability within the dataset.

\begin{figure}[!htbp]
    \centering
    \includegraphics[width=0.8\textwidth]{img/regression_catboost_ypred.pdf}
    \caption{Predicted delay values distribution ($y_\text{pred}$) for the regression task using CatBoost.}
    \label{fig:y_pred}
\end{figure}

\begin{figure}[!htbp]
    \centering
    \includegraphics[width=0.8\textwidth]{img/regression_catboost_ytest.pdf}
    \caption{Actual delay values distribution ($y_\text{test}$) for the regression task in the test set.}
    \label{fig:y_test}
\end{figure}

The comparison between predicted and actual distributions reveals that CatBoost captures core trends in the data, though some deviation at extreme delay values suggests potential areas for improvement. Overall, CatBoost’s performance in regression further validates its flexibility and predictive power, offering a promising model for continuous delay predictions in airport network analysis.

\section{Deployment and Implementation}

The source code for this project is available on GitHub at \href{https://github.com/graph-learning-ita/airnet-holding-ml/}{https://github.com/graph-learning-ita/airnet-holding-ml/}.

We have also developed a web-based simulation tool using Folium and Streamlit, which visualizes flight delays as predicted by the CatBoost model. Users can specify a simulation period, during which the model's predictions guide the flight paths and indicate holding maneuvers in real time, enabling stakeholders to observe potential delay scenarios. This interactive platform enhances the practical application of the model by providing a visual, user-friendly interface to interpret predictions dynamically.

We called the application `Airdelay' and it is available at \url{https://airdelay.manoel.dev}. Figure \ref{fig:airdelay} shows a screenshot of the Airdelay tool, illustrating the predicted flight delays due to holding maneuvers. Users can interact with the map, explore different scenarios, and observe the model's predictions in real time, enhancing their understanding of the model's performance and potential operational impact.

\begin{figure}
    \centering
    \includegraphics[width=0.8\textwidth]{img/airdelay.png}
    \caption{Airdelay web-based simulation tool, showing predicted flight delays due to holding maneuvers.}
    \label{fig:airdelay}
\end{figure}

\section{Discussion, Limitations and Future Works}

The results presented in this chapter provide insights into the trade-offs between the CatBoost model and various GAT configurations. CatBoost, with its interpretable, tabular-focused approach, offers a more reliable model for imbalanced data, outperforming GAT in both precision and recall. GAT models with additional layers did not significantly improve performance and, in some cases, led to overfitting or instability due to the network's complexity and the dataset's limitations.

By employing graph features in a gradient boosting framework, we have shown that structured data representations, combined with graph-theoretic metrics, can in some cases outperform GNNs in predictive tasks. Revealing that traditional graph machine learning continues important even in the area of graph deep learning, because of the possibility of integrating graph measures into tree-based models.

However, our work is limited due to the lack of time for more extensive hyperparameter tuning and model optimization. Future work should focus on refining the GAT model architecture, exploring additional graph-based features, and addressing the dataset's class imbalance more effectively with undersampling and/or oversampling techniques. Additionally, other ML models, such as SVM, could be tested.

Furthermore, there are new GNN architectures that could be explored. A recent work by \citeonline{egressy2024provably} introduced a provably powerful graph neural network for directed multigraphs, that could improve minority-class F1 score by up to 30\% in their tests. In our scenario this could be a game changer, since the problem with the GAT model was the class imbalance.

% ---
\FloatBarrier

% -----------------------------------------------------------
% Referências bibliográficas
% ----------------------------------------------------------

\newpage

%---------------------------------------------------------------------

\bibliographystyle{plainnat}
\bibliography{8_references}

\end{document}
