\section{Spatial GNNs}


\begin{frame}{Spatial-based GNNs Overview}
    \begin{itemize}
        \item Spatial-based GNNs perform convolutions in the spatial domain by directly leveraging the graph structure.
        \item These methods aggregate features from local neighborhoods, similar to CNNs on image data.
        \item Each node's features are updated iteratively through message passing with its neighboring nodes.
        \item Suitable for large-scale graphs due to their scalability and flexibility.
    \end{itemize}
\end{frame}

\begin{frame}{Message Passing Framework for Spatial GNNs}
    \begin{itemize}
        \item At layer \( t \), each node \( i \) aggregates features from its neighbors \( \mathcal{N}(i) \).
        \[
        \mathbf{m}_i^{(t+1)} = \text{AGGREGATE}^{(t)} \left( \left\{ \mathbf{h}_j^{(t)} : j \in \mathcal{N}(i) \right\} \right)
        \]
        \item Node \( i \) then updates its own features based on the aggregated message:
        \[
        \mathbf{h}_i^{(t+1)} = \text{UPDATE}^{(t)} \left( \mathbf{h}_i^{(t)}, \mathbf{m}_i^{(t+1)} \right)
        \]
        \item Flexible, adaptable to dynamic graphs, as they only require local neighborhood information.
    \end{itemize}
\end{frame}

\begin{frame}{Message Passing Diagram}

    \begin{figure}[h]
  \centering
  \begin{columns}

    \column{.25\linewidth}

    \caption{Message Passing GNN}

    \label{fig:visualgo}

    \column{.75\linewidth}
    \resizebox{0.825\columnwidth}{0.625\linewidth}{%

      \begin{tikzpicture}[scale=.8, every node/.style={minimum size=1cm}]
        \node[circle, gray, draw, very thick] (1) {$\vec{h}^\ell_1$};
        \node[circle, echodrk, draw, below right=2em and 3em of 1, very thick] (2) {$\vec{h}^\ell_2$};
        \node[circle, draw, echodrk, above right=3em and 4em of 1, very thick] (3) {$\vec{h}^\ell_3$};
        \node[circle, draw, olivegreen, right=7em of 1, ultra thick] (4) {$\vec{h}^{\ell+1}_4$};
        \node[circle, echodrk, draw, above right=3.5em and 4em of 4, very thick] (5) {$\vec{h}^\ell_5$};
        \node[circle, echodrk, draw, below right=3em and 3em of 4, very thick] (6) {$\vec{h}^\ell_6$};

        \draw[gray, very thick] (1) -- (2);
        \draw[gray, very thick] (2) -- (3);
        \draw[red, ultra thick, -stealth] (2) -- node[below, xshift=0.9em] (ll) {$\vec{h}_{2\rightarrow 4}^\ell$} (4);
        \draw[red, ultra thick, -stealth] (3) -- node[above,xshift=1em, inner sep=0em] (l1) {$\vec{h}_{3\rightarrow 4}^\ell$} (4);
        \draw[red,ultra thick, -stealth] (5) -- node[right, yshift=-0.5em] (lr) {$\vec{h}_{5\rightarrow 4}^\ell$} (4);
        \draw[red,ultra thick, -stealth] (6) -- node[right, xshift=0.1em] (lw) {$\vec{h}_{6\rightarrow 4}^\ell$} (4);

        \node[right=5.5em of 6, echodrk] (31) {$\vec{h}^\ell_3$};
        \node[right=1em of 31, echodrk] (41) {$\vec{h}^\ell_4$};

        \node[rectangle, draw, camdrk, very thick, above right=3em and -0.5em of 31] (F) {$f_e^\ell$};

        \node[above=3em of F, red] (34) {$\vec{h}^\ell_{3\rightarrow 4}$};

        \draw[very thick, camdrk, -stealth] (31) -- (F);
        \draw[very thick, camdrk, -stealth] (41) -- (F);
        \draw[very thick, camdrk, -stealth] (F) -- (34);

        \draw[very thick, -stealth, dashed, echodrk] (3) edge[bend left=30] (31);
        \draw[very thick, -stealth, dashed, echodrk] (4) edge[bend right=65] (41);
        \draw[very thick, -stealth, dashed, red] (34) edge[bend right=40] (l1);

        \node[right= of 4, camdrk, rectangle, draw, very thick] (G) {$f_v^\ell$};
        \node[right=4 em of G, olivegreen] (l11) {$\vec{h}^{\ell+1}_4$};

        \draw[-stealth, camdrk, very thick] (4) -- (G);
        \draw[-stealth, camdrk, very thick] (G) -- (l11);

        \draw[very thick, -stealth, dashed, olivegreen] (l11) edge[bend left=25] (4);
      \end{tikzpicture}

    }
  \end{columns}
\end{figure}


\end{frame}


\begin{frame}{GraphSAGE: Sampling and Aggregation}
    \begin{itemize}
        \item GraphSAGE \cite{hamilton2017inductive} computes node embeddings by sampling and aggregating neighbor features.
        \item Various aggregation functions (e.g., mean, LSTM-based, pooling) allow flexibility for large graphs.
        \item Supports inductive learning by enabling embeddings for unseen nodes.
        \item Advantages:
            \begin{itemize}
                \item Inductive capability for new nodes.
                \item More scalable than spectral-based GNNs for large graphs.
            \end{itemize}
    \end{itemize}
\end{frame}

\begin{frame}{Graph Attention Networks (GAT)}
    \begin{itemize}
        \item GAT \cite{velickovic2017graph} introduces attention mechanisms to learn different weights for neighbors.
        \item Attention coefficient between nodes \( i \) and \( j \):
        \[
        c_{ij} = \phi_1 \left( \mathbf{a}^T \left[ W \mathbf{h}_i \, || \, W \mathbf{h}_j \right] \right)
        \]
        \item Attention coefficients normalized using softmax:
        \[
        \alpha_{ij} = \frac{\exp(c_{ij})}{\sum_{k \in \mathcal{N}(i)} \exp(c_{ik})}
        \]
        \item Final feature update with weighted aggregation:
        \[
        \mathbf{h}_i' = \phi_2 \left( \sum_{j \in \mathcal{N}(i)} \alpha_{ij} \mathbf{W} \mathbf{h}_j \right)
        \]
    \end{itemize}
\end{frame}

\begin{frame}{GAT Diagram}
    \begin{figure}[h]
    \centering
   
	
    \caption{GAT Layer with multi-head attention.}

    \label{fig:gat_layer}

   

\begin{tikzpicture}

	\node[circle, draw, thick] (h1) {$\vec{h}_1$};
	\node[circle, draw, thick, above left=of h1] (h4) {$\vec{h}_2$};
	\node[circle, draw, thick, left=5em of h1] (h5) {$\vec{h}_3$};
	\node[circle, draw, thick, below left=of h1] (h6) {$\vec{h}_4$};
	\node[circle, draw, thick, below=5em of h1] (h7) {$\vec{h}_5$};
	\node[circle, draw, thick, below right=of h1] (h8) {$\vec{h}_6$};
	
	\draw[-stealth, mymauve, thick,decoration={snake, pre length=0.01mm, segment length=2mm, amplitude=0.3mm, post length=1.5mm}, decorate] (h8.120) -- node[sloped, above, black] {$\vec{\alpha}_{16}$} (h1.-30);
	\draw[-stealth, blue, thick] (h8.135) -- (h1.-45);
	\draw[-stealth, mygreen, thick, decoration={zigzag, pre length=0.01mm, segment length=2mm, amplitude=0.3mm, post length=1.5mm}, decorate] (h8.150) -- (h1.-60);
	
	\draw[-stealth, mymauve, thick,decoration={snake, pre length=0.01mm, segment length=2mm, amplitude=0.3mm, post length=1.5mm}, decorate] (h1.30) to[looseness=7] node[sloped, above, black] {$\vec{\alpha}_{11}$}(h1.105);
	\draw[-stealth, blue, thick] (h1.45) to[looseness=9] (h1.90);
	\draw[-stealth, mygreen, thick, decoration={zigzag, pre length=0.01mm, segment length=2mm, amplitude=0.3mm, post length=1.5mm}, decorate] (h1.60) to[looseness=20] (h1.75);
	
	\draw[-stealth, mymauve, thick,decoration={snake, pre length=0.01mm, segment length=2mm, amplitude=0.3mm, post length=1.5mm}, decorate] (h4.285) -- node[sloped, below, black] {$\vec{\alpha}_{12}$}(h1.150);
	\draw[-stealth, blue, thick] (h4.300) -- (h1.135);
	\draw[-stealth, mygreen, thick, decoration={zigzag, pre length=0.01mm, segment length=2mm, amplitude=0.3mm, post length=1.5mm}, decorate] (h4.315) -- (h1.120);
	
	\draw[-stealth, mymauve, thick,decoration={snake, pre length=0.01mm, segment length=2mm, amplitude=0.3mm, post length=1.5mm}, decorate] (h5.-15) -- node[sloped, below, black] {$\vec{\alpha}_{13}$}(h1.195);
	\draw[-stealth, blue, thick] (h5.0) -- (h1.180);
	\draw[-stealth, mygreen, thick, decoration={zigzag, pre length=0.01mm, segment length=2mm, amplitude=0.3mm, post length=1.5mm}, decorate] (h5.15) -- (h1.165);
	
		\draw[-stealth, mymauve, thick,decoration={snake, pre length=0.01mm, segment length=2mm, amplitude=0.3mm, post length=1.5mm}, decorate] (h6.15) -- node[sloped, below, black] {$\vec{\alpha}_{14}$}(h1.240);
	\draw[-stealth, blue, thick] (h6.30) -- (h1.225);
	\draw[-stealth, mygreen, thick, decoration={zigzag, pre length=0.01mm, segment length=2mm, amplitude=0.3mm, post length=1.5mm}, decorate] (h6.45) -- (h1.210);
	
	\draw[-stealth, mymauve, thick,decoration={snake, pre length=0.01mm, segment length=2mm, amplitude=0.3mm, post length=1.5mm}, decorate] (h7.75) -- node[sloped, below, black] {$\vec{\alpha}_{15}$}(h1.-75);
	\draw[-stealth, blue, thick] (h7.90) -- (h1.-90);
	\draw[-stealth, mygreen, thick, decoration={zigzag, pre length=0.01mm, segment length=2mm, amplitude=0.3mm, post length=1.5mm}, decorate] (h7.105) -- (h1.-105);
	
	\node[circle, draw, thick, right=10em of h1, opacity=0.8] (hp) {$\vec{h}_1'$};
	
	\coordinate[right=5em of h1] (A);
	
	\draw[-stealth,  mymauve, opacity=0.5, ultra thick,decoration={snake, pre length=0.01mm, segment length=2mm, amplitude=0.3mm, post length=1.5mm}, decorate] (h1.20) -- (A) -- (hp);
	\draw[-stealth, mygreen, opacity=0.5, ultra thick,decoration={zigzag, pre length=0.01mm, segment length=2mm, amplitude=0.3mm, post length=1.5mm}, decorate] (h1.-20) -- (A) -- (hp);
	\draw[-stealth, blue, opacity=0.5, ultra thick] (h1.0) -- (A) -- node[black, above, opacity=1.0] {concat/avg} (hp);

\end{tikzpicture}




\end{figure}

\end{frame}

\begin{frame}{Benefits of Spatial-based GNNs}
    \begin{itemize}
        \item Localized computation makes spatial GNNs efficient for large, dynamic graphs.
        \item No dependency on the global graph structure, unlike spectral methods.
        \item Attention mechanisms allow fine-tuned aggregation, useful for complex relational data.
        \item Applications in social networks, recommendation systems, and knowledge graphs.
    \end{itemize}
\end{frame}

\begin{frame}{Limitations of Spatial-based GNNs}
    \begin{itemize}
        \item Aggregation methods can oversimplify node information, leading to potential loss of unique node features.
        \item Spatial-based GNNs struggle with long-range dependencies due to limited neighborhood scope.
        \item May suffer from over-smoothing, where repeated aggregations make node representations indistinguishable.
        \item High memory and computation costs when aggregating large numbers of neighbors, especially for high-degree nodes.
        \item Lack of global context can limit performance on tasks requiring global structural information.
    \end{itemize}
\end{frame}