% ---
% Conclusão
% ---
\chapter{Conclusion}
\label{Conclusion}

\section{Project Contributions to the Student}

This project was an excellent way to reinforce the knowledge I gained
in my undergraduate program, particularly in areas like algorithms and
graph applications. Working at ICMC was an invaluable opportunity to
engage with experts in graph machine learning and complex
networks. Through this project, I gained insight into both fields—a
feat that would have been challenging without the foundation provided
by courses like Complex Networks and Artificial Intelligence, which
taught me the basics of each. Additionally, this experience provided
an ideal introduction to studying \emph{graph neural networks}, a
field I plan to pursue further.

\section{Relationship between the Undergraduate Course and the Project}

The undergraduate curriculum strongly aligned with this project. Key
subjects included Algorithms and Data Structures, Computational
Modeling in Graphs, Artificial Intelligence, High-Performance
Computing, Linear Algebra, Calculus, Stochastic Processes, Statistics,
and Advanced Algorithms and Applications. Together, these courses
provided a comprehensive foundation in computer science, with ample
focus on graph theory, machine learning, and algorithm design. The
curriculum thus offered a robust theoretical framework that enriched
my understanding of complex networks and graph-based models.

Another crucial aspect of my program was the opportunity to
participate in teaching and research. This experience not only
deepened my technical knowledge but also honed my communication
skills, enabling me to articulate ideas and solutions more
effectively.

\section{Reflections on the Undergraduate Course}

The Bachelor of Computer Science at the Instituto de Ciências
Matemáticas e de Computação, Universidade de São Paulo, has been
instrumental in providing me with knowledge, friendships,
determination, and joy. I am deeply grateful for the supportive
structure at ICMC, which offered everything necessary for personal and
academic growth. Being surrounded by dedicated and inquisitive peers
and professors has been the highlight of my time here, creating a
stimulating environment that constantly pushed me to expand my
horizons as a person, a scholar, and a future professional.

That said, I do have one critique regarding the course
curriculum. Recently, it feels as though the Bachelor of Computer
Science program has shifted its focus more toward the job market than
foundational science. While I understand the need to prepare students
for industry, I believe that we must preserve the course's scientific
roots. For instance, `Research Methodology' is not a required course,
yet it is fundamental for academic development, while other subjects
more geared toward job readiness are mandatory. The removal of
`Calculus 4` from the curriculum further illustrates this shift away
from theoretical depth, despite its relevance in areas like machine
learning.

To address this, I would suggest offering two distinct tracks within
the program: an Academic Track and a Professional Track. This
structure would allow students interested in research and theoretical
science to focus on these areas, while those aiming for immediate
industry applications could focus on practical skills. This change
might also encourage more students to pursue advanced degrees at ICMC
by fostering a stronger scientific foundation during their
undergraduate studies.

Despite these critiques, I am confident that choosing this program was
one of the best decisions I have made. The course has opened doors to
opportunities I had not even dreamed of achieving.
