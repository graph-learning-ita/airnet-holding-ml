\section[Graph Attention Network]{Graph Attention Network}
\label{Graph Attention Network}

As we previously described, the GAT model in section
\ref{spatial-based} has a large range of applications, from drug
discovery to fake news detection \cite{keywordsCaravanti}. The GAT
model leverages the underlying graph structure but does not rely on
explicitly computed graph-derived features like the CatBoost model
does. Instead, it learns node representations in an end-to-end manner,
enabling the model to capture the relationships between airports and
flights directly from the data.

The modelling of a GNN for our problem is a challenging task, as we
have to adapt the model to predict edge features, since `holding' is
an edge feature in our setting. In section \ref{spectral-based} we
detailed why the spectral-based GNNs are not suitable for our setting,
as they are not able to handle edge features and direction, due to
their `node-centric' approach based on the adjacency matrix. Although
spatial-based GNNs can handle direction in their majority, they are
not able to handle edge features in general, since they need to create
a way to aggregate the edge features with the neighbors' features.

The GAT model is so used because it is highly adaptable in pratically
any graph setting. As we will show, the attention mechanism detailed
in section \ref{spatial-based} can be generalized to handle edge
features, and the model can be adapted to predict edge features. In
fact, a simple concatenation ($ || $) in the attention formula already
gives us this power ,

$$ \alpha_{ij} = \sigma(\phi_1( \mathbf{a}^T [ W h_i || W h_j || W_2 e_{ij} ])) \; \; \text{,}$$

  where $e_{ij}$ are the edge features, $h_i$ and $h_j$ are the node
  features, and $W$ and $W_2$ are the weight matrices. This formula
  allows the model to focus on the relevant neighboring nodes, making
  it ideal for relational data. In our case, the edge features are the
  tabular data features with holding being part of them, which is the
  target we want to predict. This mechanism is demonstrated in Figure
  \ref{fig:gat_layer}.


\input{img/gat_layer.tex}

Furthermore, the directed multigraph setting we described in section
\ref{sec:catboost_model} is not a problem for the GAT model, since it
can handle multiple edges between the same pair of nodes, as we will
show in the following sections. We show how we model the GAT to be a
directed multigraph representing the flights and their features in
Figure \ref{fig:multigraph_layer}.


\begin{figure}[h]
  \centering
  \begin{tikzpicture}
    % Define colors for different attention heads
    \definecolor{color1}{RGB}{60, 120, 216}  % Blue
    \definecolor{color2}{RGB}{34, 139, 34}   % Green
    \definecolor{color3}{RGB}{255, 99, 71}   % Red

    % Define nodes
    \node[circle, draw, thick] (node1) at (0,0) {$\vec{h}_{\text{SP}}$};
    \node[circle, draw, thick] (node2) at (5,0) {$\vec{h}_{\text{RJ}}$};

    % Flight 1: alternating colors for attention heads
    \draw[-stealth, thick, color1!150] (node1) .. controls +(1,2) and +(-1,2) .. node[midway, above, black] {$\vec{\alpha}_{\text{SP},\text{RJ}}^{(\text{\faPlane}_1)}$} (node2);
    \draw[-stealth, thick, color2!150, dashed] (node1) .. controls +(1,1.5) and +(-1,1.5) .. node[midway, above, color2] {} (node2);
    \draw[-stealth, thick, color3!150, dotted] (node1) .. controls +(1,1) and +(-1,1) .. node[midway, above, color3] {} (node2);

    % Flight 2: same color pattern but lighter shades
    \draw[-stealth, thick, color1!100] (node1) .. controls +(1,-1) and +(-1,-1) .. node[midway, below, color1!70] {} (node2);
    \draw[-stealth, thick, color2!100, dashed] (node1) .. controls +(1,-1.5) and +(-1,-1.5) .. node[midway, below, black] {$\vec{\alpha}_{\text{SP},\text{RJ}}^{(\text{\faPlane}_2)}$} (node2);
    \draw[-stealth, thick, color3!100, dotted] (node1) .. controls +(1,-2) and +(-1,-2) .. node[midway, below, color3!70] { } (node2);

    % Flight 3: same color pattern but even lighter shades
    \draw[-stealth, thick, color1!50] (node1) .. controls +(1,-3) and +(-1,-3) .. node[midway, below, color1!50] { } (node2);
    \draw[-stealth, thick, color2!50, dashed] (node1) .. controls +(1,-3.5) and +(-1,-3.5) .. node[midway, below, color2!50] { } (node2);
    \draw[-stealth, thick, color3!50, dotted] (node1) .. controls +(1,-4) and +(-1,-4) .. node[midway, below, black] {$\vec{\alpha}_{\text{SP},\text{RJ}}^{(\text{\faPlane}_3)}$} (node2);

    % Output node
    \node[circle, draw, thick, right=7em of node2, opacity=0.8] (output) {$\vec{h}_{\text{RJ}}'$};

    % Aggregation line with label
    % \draw[-stealth, thick, color1!70] (node2) -- ++(1.5,0) -- ++(1.5,0) node[midway, above, black] {concat/avg} -- (output);

    \draw[-stealth,  color1, opacity=0.5, ultra thick,decoration={snake, pre length=0.01mm, segment length=2mm, amplitude=0.3mm, post length=1.5mm}, decorate] (node2.20)  -- (output);
    \draw[-stealth, color3, opacity=0.5, ultra thick,decoration={zigzag, pre length=0.01mm, segment length=2mm, amplitude=0.3mm, post length=1.5mm}, decorate] (node2.-20)  -- (output);
    \draw[-stealth, color2, opacity=0.5, ultra thick] (node2.0)  -- node[black, above, opacity=1.0] {concat/avg} (output);

  \end{tikzpicture}
  \caption{Airport multigraph GAT Layer with multi-head attention for three different flights between nodes (SP,RJ), with alternating colors opacity for each flight.}
  \label{fig:multigraph_layer}
\end{figure}


Finally, the last layer of our predictor would be to pass the learned
node embeddings $h_i$ and $h_j$ with the edge feature $e_{ij}^{(k)}$
of the flight $k$ to a fully connected layer (MLP) to predict the
holding of the flight $k$.  That is, we simply concatenate them, and
after the MLP layer, we have a sigmoid $\sigma$ activation function that
outputs the prediction $\hat{y}_k$ of holding,

$$ \hat{y}_k = \sigma (\text{MLP}(h_i || h_j || e_{ij}^{(k)})) \; \; \; \text{.} $$
