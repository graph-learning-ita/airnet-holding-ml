%% USPSC-pre-textual-ICMC.tex
%% Camandos para definição do tipo de documento (tese ou dissertação ou monografia), área de concentração, opção, preâmbulo, titulação 
%% referentes ao Programa de Pós-Graduação o ICMC
\instituicao{Instituto de Ci\^encias Matem\'aticas e de Computa\c{c}\~ao, Universidade de S\~ao Paulo}
\unidade{INSTITUTO DE CI\^ENCIAS MATEM\'ATICAS E DE COMPUTA\c{C}\~AO}
\unidademin{Instituto de Ci\^encias Matem\'aticas e de Computa\c{c}\~ao}
\universidademin{Universidade de S\~ao Paulo}
\setorpos{SERVI\c{C}O DE P\'OS-GRADUA\c{C}\~AO DO ICMC-USP}

\notafolharosto{Vers\~ao original}
\notafolharostoadic{Original version}
%Para versão original em inglês, comente os comandos/declarações acima (inclua % antes do comando acima) 
% e tire a % dos comandos/declarações abaixo no idioma do texto
%\notafolharosto{Original version}
%\notafolharostoadic{Vers\~ao original}
 
%Para versão revisada, comente os comandos/declarações acima (inclua % antes do comando acima) 
% e tire a % dos comandos/declarações abaixo, em conformidade com o idioma do texto
% Se o Idioma do texto for português: 
%\notafolharosto{Vers\~ao revisada}
%\notafolharosto{Final version}
% Se o Idioma do texto for Inglês: 
%\notafolharosto{Final version}
%\notafolharosto{Vers\~ao revisada}
% ---
% dados complementares para CAPA e FOLHA DE ROSTO
% ---
\universidade{UNIVERSIDADE DE S\~AO PAULO}

% Idioma do texto em PORTUGUÊS
\titulo{Modelo para teses e disserta\c{c}\~oes em \LaTeX\ utilizando o Pacote USPSC para o ICMC} 
\titleabstract{Model for thesis and dissertations in \LaTeX\ using the USPSC Package to the ICMC}
\tituloadic{Model for thesis and dissertations in \LaTeX\ using the USPSC Package to the ICMC}
\tituloresumo{Modelo para teses e disserta\c{c}\~oes em LaTeX utilizando o Pacote USPSC para o ICMC}

% Idioma do texto em INGLÊS
% 22/02/2017 - Título para página de rosto adicional
% para a versão em inglês, utilize os comandos abaixo, e inclua % no início dos 4 comandos logo acima,  cada comando acima que são referentes ao texto do Trabalho Acadêmico em português
%\titulo{Model for thesis and dissertations in LaTeX using the USPSC Package to the ICMC}
%\titleabstract{Model for thesis and dissertations in LaTeX using the USPSC Package to the ICMC}
%\tituloadic{Modelo para teses e disserta\c{c}\~oes em LaTeX utilizando o Pacote USPSC para o ICMC}
%\tituloresumo{Modelo para teses e disserta\c{c}\~oes em LaTeX utilizando o Pacote USPSC para o ICMC}

\autor{Jos\'e da Silva}
\autorficha{Silva, Jos\'e da}
\autorabr{SILVA, J.}

\cutter{S856m}
% Para gerar a ficha catalográfica sem o Código Cutter, basta 
% incluir uma % na linha acima e tirar a % da linha abaixo
%\cutter{ }

\local{S\~ao Carlos}
\data{2021}

% Para o idioma português:
\renewcommand{\orientadorname}{Orientadora:}
\orientador{Profa. Dra. Elisa Gon\c{c}alves Rodrigues}
\orientadoradic{Advisor: Profa. Dra. Elisa Gon\c{c}alves Rodrigues}
\orientadorcorpoficha{orientadora Elisa Gon\c{c}alves Rodrigues}
\orientadorficha{Rodrigues, Elisa Gon\c{c}alves, orient}
%Para incluir o nome do(a) coorientados(a), inclua % nos 2 comandos acima e retire a % dos 2 comandos abaixo
%\orientadorcorpoficha{orientadora Elisa Gon\c{c}alves Rodrigues ;  co-orientador Jo\~ao Alves Serqueira}
%\orientadorficha{Rodrigues, Elisa Gon\c{c}alves, orient. II. Serqueira, Jo\~ao Alves, co-orient}


%Se o idoma for o inglês, inclua % nos comandos acima e exclua dos comandos abaixo
%\renewcommand{\orientadorname}{Advisor:}
%\orientador{Profa. Dra. Elisa Gon\c{c}alves Rodrigues}
%\orientadoradic{Orientadora: Profa. Dra. Elisa Gon\c{c}alves Rodrigues}
%\orientadorcorpoficha{orientadora Elisa Gon\c{c}alves Rodrigues}
%\orientadorficha{Rodrigues, Elisa Gon\c{c}alves, orient}
%Para incluir o nome do(a) coorientados(a), inclua % nos 2 comandos acima e retire a % dos 2 comandos abaixo
%\orientadorcorpoficha{orientadora Elisa Gon\c{c}alves Rodrigues ;  co-orientador Jo\~ao Alves Serqueira}
%\orientadorficha{Rodrigues, Elisa Gon\c{c}alves, orient. II. Serqueira, Jo\~ao Alves, co-orient}

% Quando houver Coorientador(a): 
% Para o idioma português:
% basta retirar  % antes de um dos comandos abaixo
%\newcommand{\coorientadorname}{Coorientador:}
%\newcommand{\coorientadorname}{Coorientadora:}
% Para o idoma inglês:
% basta retirar  % antes do comando abaixo
%\newcommand{\coorientadorname}{Coorientador:}

% Quando houver Coorientador(a), basta tirar a % utilizar o comando abaixo
%\newcommand{\coorientadorname}{Coadvisor:}
%Se houver co-orientador, inclua % antes das duas linhas (antes dos comandos \orientadorcorpoficha e \orientadorficha) 
%          e tire a % antes dos 3 comandos abaixo
%\coorientador{Prof. Dr. Jo\~ao Alves Serqueira}
%\coorientadoradic{ Co-orientador: Prof. Dr. Jo\~ao Alves Serqueira}
%\orientadorcorpoficha{orientadora Elisa Gon\c{c}alves Rodrigues ;  co-orientador Jo\~ao Alves Serqueira}
%\orientadorficha{Rodrigues, Elisa Gon\c{c}alves, orient. II. Serqueira, Jo\~ao Alves, co-orient}

%Para o idioma Inglês, retire a % antes da linha abaixo
%\renewcommand{\areaname}{Concentration area: }

		
\notaautorizacao{AUTORIZO A REPRODU\c{C}\~AO E DIVULGA\c{C}\~AO TOTAL OU PARCIAL DESTE TRABALHO, POR QUALQUER MEIO CONVENCIONAL OU ELETR\^ONICO PARA FINS DE ESTUDO E PESQUISA, DESDE QUE CITADA A FONTE.}
% Se o idioma for o inglês, inclua a % antes do campo \notaautorizacao acima e retire a % da linha abaixo
%\notaautorizacao{I AUTORIZE THE REPRODUCTION AND DISSEMINATION OF TOTAL OR PARTIAL COPIES OF THIS DOCUMENT, BY CONVENCIONAL OR ELECTRONIC MEDIA FOR STUDY OR RESEARCH PURPOSE, SINCE IT IS REFERENCED.}

\notabib{Ficha catalogr\'afica elaborada pela Biblioteca Prof. Achille Bassi, ICMC/USP, com os dados fornecidos pelo(a) autor(a)}

\newcommand{\programa}[1]{
% MPMp ==========================================================================
\ifthenelse{\equal{#1}{MPMp}}{
	\tipotrabalho{Disserta\c{c}\~ao (Mestrado em Ci\^encias)}
	\tipotrabalhoabs{Dissertation (Master in Science)}
	\area{Matem\'atica em Rede Nacional}
	\areaadic{Concentration area: Mathematics in National Network}
	%\opcao{Nome da Opção em português}
	%\opcaoadic{Nome da Opção em inglês}
	% O preambulo deve conter o tipo do trabalho, o objetivo, 
	% o nome da instituição, a área de concentração e opção quando houver
	\preambulo{Disserta\c{c}\~ao apresentada ao Instituto de Ci\^encias Matem\'aticas e de Computa\c{c}\~ao, Universidade de S\~ao Paulo - ICMC/USP, como parte dos requisitos para obten\c{c}\~ao do t\'itulo de Mestre em Ci\^encias - Mestrado Profissional em Matem\'atica em Rede Nacional.}
	\preambuloadic{Dissertation submitted to the Instituto de Ci\^encias Matem\'aticas e de Computa\c{c}\~ao, Universidade de S\~ao Paulo - ICMC/USP, in partial fulfillment of the requirements for the degree of the Master in Science - Professional Master in Mathematics in National Network.}
	\notaficha{Disserta\c{c}\~ao (Mestrado - Programa de Mestrado Profissional em Matem\'atica em Rede Nacional)}
	\notacapaicmc{Disserta\c{c}\~ao de Mestrado do Programa de Mestrado Profissional em \\Matem\'atica em Rede Nacional (PROFMAT)}
    }{
% MPMe ==========================================================================
\ifthenelse{\equal{#1}{MPMe}}{
	\renewcommand{\areaname}{Concentration area:}
	\tipotrabalho{Disserta\c{c}\~ao (Mestrado em Ci\^encias)}
	\tipotrabalhoabs{Dissertation (Master in Science)}
	\area{Mathematics in National Network}
	\areaadic{\'Area de concentra\c{c}\~ao: Matem\'atica em Rede Nacional}
	%\opcao{Nome da Opção em inglês}
	%\opcaoadic{Nome da Opção em português}
	% O preambulo deve conter o tipo do trabalho, o objetivo, 
	% o nome da instituição, a área de concentração e opção quando houver
	\preambulo{Dissertation submitted to the Instituto de Ci\^encias Matem\'aticas e de Computa\c{c}\~ao, Universidade de S\~ao Paulo - ICMC/USP, in partial fulfillment of the requirements for the degree of the Master in Science - Professional Master in Mathematics in National Network.}		
	\preambuloadic{Disserta\c{c}\~ao apresentada ao Instituto de Ci\^encias Matem\'aticas e de Computa\c{c}\~ao, Universidade de S\~ao Paulo - ICMC/USP, como parte dos requisitos para obten\c{c}\~ao do t\'itulo de Mestre em Ci\^encias - Mestrado Profissional em Matem\'atica em Rede Nacional.}
	\notaficha{Dissertation (Master - Professional Master\'{}s Program in Mathematics on the National Network)}
	\notacapaicmc{Master\'{}s Dissertation of the Professional Master\'{}s Program in \\Mathematics on the National Network (PROFMAT)}
    }{   
% MPMECAIp ==========================================================================
\ifthenelse{\equal{#1}{MPMECAIp}}{
	\tipotrabalho{Disserta\c{c}\~ao (Mestrado em Ci\^encias)}
	\tipotrabalhoabs{Dissertation (Master in Science)}
	\area{Matem\'atica, Estat\'istica e Computa\c{c}\~ao}
	\areaadic{Concentration area: Mathematics, Statistics and Computing}
	%\opcao{Nome da Opção em português}
	%\opcaoadic{Nome da Opção em inglês}
	% O preambulo deve conter o tipo do trabalho, o objetivo, 
	% o nome da instituição, a área de concentração e opção quando houver
	\preambulo{Disserta\c{c}\~ao apresentada ao Instituto de Ci\^encias Matem\'aticas e de Computa\c{c}\~ao, Universidade de S\~ao Paulo - ICMC/USP, como parte dos requisitos para obten\c{c}\~ao do t\'itulo de Mestre em Ci\^encias - Mestrado Profissional em Matem\'atica, Estat\'istica e Computa\c{c}\~ao Aplicadas \`a Ind\'ustria.}
	\preambuloadic{Dissertation submitted to the Instituto de Ci\^encias Matem\'aticas e de Computa\c{c}\~ao, Universidade de S\~ao Paulo - ICMC/USP, in partial fulfillment of the requirements for the degree of the Master in Science - Professional Masters in Mathematics, Statistics and Computing Applied to Industry.}
	\notaficha{Disserta\c{c}\~ao (Mestrado - Programa de Mestrado Profissional em Matem\'atica, Estat\'istica e Computa\c{c}\~ao Aplicadas \`a Ind\'ustria)}
	\notacapaicmc{Disserta\c{c}\~ao de Mestrado do Programa de Mestrado Profissional em \\Matem\'atica, Estat\'istica e Computa\c{c}\~ao Aplicadas \`a Ind\'ustria (MECAI)} 
}{
% MPMECAIe ==========================================================================
\ifthenelse{\equal{#1}{MPMECAIe}}{
	\renewcommand{\areaname}{Concentration area:}
	\tipotrabalho{Disserta\c{c}\~ao (Mestrado em Ci\^encias)}
	\tipotrabalhoabs{Dissertation (Master in Science)}
	\area{Mathematics, Statistics and Computing}
	\areaadic{\'Area de concentra\c{c}\~ao: Matem\'atica, Estat\'istica e Computa\c{c}\~ao}
	%\opcao{Nome da Opção em inglês}
	%\opcaoadic{Nome da Opção em português}
	% O preambulo deve conter o tipo do trabalho, o objetivo, 
	% o nome da instituição, a área de concentração e opção quando houver
	\preambulo{Dissertation submitted to the Instituto de Ci\^encias Matem\'aticas e de Computa\c{c}\~ao, Universidade de S\~ao Paulo - ICMC/USP, in partial fulfillment of the requirements for the degree of the Master in Science - Professional Masters in Mathematics, Statistics and Computing Applied to Industry.}		
	\preambuloadic{Disserta\c{c}\~ao apresentada ao Instituto de Ci\^encias Matem\'aticas e de Computa\c{c}\~ao, Universidade de S\~ao Paulo - ICMC/USP, como parte dos requisitos para obten\c{c}\~ao do t\'itulo de Mestre em Ci\^encias - Mestrado Profissional em Matem\'atica, Estat\'istica e Computa\c{c}\~ao Aplicadas \`a Ind\'ustria.}
	\notaficha{Dissertation (Master - Professional Master's Program in Mathematics, Statistics and Computing Applied to Industry)}
	\notacapaicmc{Master's Dissertation of the Professional Master's Program in Mathematics, \\Statistics and Computing Applied to Industry (MECAI)}
}{    
% DMAp ==========================================================================
\ifthenelse{\equal{#1}{DMAp}}{
    \tipotrabalho{Tese (Doutorado em Ci\^encias)}
    \tipotrabalhoabs{Thesis (Doctorate in Science)}
    \area{Matem\'atica}
    \areaadic{Concentration area: Mathematics}
	%\opcao{Nome da Opção em português}
	%\opcaoadic{Nome da Opção em inglês}
    % O preambulo deve conter o tipo do trabalho, o objetivo, 
	% o nome da instituição, a área de concentração e opção quando houver
	\preambulo{Tese apresentada ao Instituto de Ci\^encias Matem\'aticas e de Computa\c{c}\~ao, Universidade de S\~ao Paulo - ICMC/USP, como parte dos requisitos para obten\c{c}\~ao do t\'itulo de Doutor em Ci\^encias - Matem\'atica.}	
	\preambuloadic{Thesis submitted to the Instituto de Ci\^encias Matem\'aticas e de Computa\c{c}\~ao, Universidade de S\~ao Paulo - ICMC/USP, in partial fulfillment of the requirements for the degree of the Doctor in Science - Mathematics.}
	\notaficha{Tese (Doutorado - Programa de P\'os-Gradua\c{c}\~ao em Matem\'atica)}
	\notacapaicmc{Tese de Doutorado do Programa de P\'os-Gradua\c{c}\~ao em \\Matem\'atica (PPG-Mat)}
    }{
% DMAe ==========================================================================
\ifthenelse{\equal{#1}{DMAe}}{
	\tipotrabalho{Tese (Doutorado em Ci\^encias)}
    \tipotrabalhoabs{Thesis (Doctorate in Science)}
	\renewcommand{\areaname}{Concentration area:}
    \area{Mathematics}
    \areaadic{\'Area de concentra\c{c}\~ao: Matem\'atica}
	%\opcao{Nome da Opção em inglês}
	%\opcaoadic{Nome da Opção em português}
    % O preambulo deve conter o tipo do trabalho, o objetivo, 
	% o nome da instituição, a área de concentração e opção quando houver
	\preambulo{Thesis submitted to the Instituto de Ci\^encias Matem\'aticas e de Computa\c{c}\~ao, Universidade de S\~ao Paulo - ICMC/USP, in partial fulfillment of the requirements for the degree of the Doctor in Science - Mathematics.}
	\preambuloadic{Tese apresentada ao Instituto de Ci\^encias Matem\'aticas e de Computa\c{c}\~ao, Universidade de S\~ao Paulo - ICMC/USP, como parte dos requisitos para obten\c{c}\~ao do t\'itulo de Doutor em Ci\^encias - Matem\'atica.}
	\notaficha{Thesis (Doctorate - Program in Mathematics)}
	\notacapaicmc{Doctoral Thesis of the Postgraduate Program in Mathematics (PPG-Mat)}
    }{
% MMAp ==========================================================================
\ifthenelse{\equal{#1}{MMAp}}{
    \tipotrabalho{Disserta\c{c}\~ao (Mestrado em Ci\^encias)}
    \tipotrabalhoabs{Dissertation (Master in Science)}
    \area{Matem\'atica}
    \areaadic{Concentration area: Mathematics}
	%\opcao{Nome da Opção em português}
	%\opcaoadic{Nome da Opção em inglês}
    % O preambulo deve conter o tipo do trabalho, o objetivo, 
	% o nome da instituição, a área de concentração e opção quando houver
	\preambulo{Disserta\c{c}\~ao apresentada ao Instituto de Ci\^encias Matem\'aticas e de Computa\c{c}\~ao, Universidade de S\~ao Paulo - ICMC/USP, como parte dos requisitos para obten\c{c}\~ao do t\'itulo de Mestre em Ci\^encias - Matem\'atica.}	
	\preambuloadic{Dissertation submitted to the Instituto de Ci\^encias Matem\'aticas e de Computa\c{c}\~ao, Universidade de S\~ao Paulo - ICMC/USP, in partial fulfillment of the requirements for the degree of the Master in Science - Mathematics.}
	\notaficha{Disserta\c{c}\~ao (Mestrado - Programa de P\'os-Gradua\c{c}\~ao em Matem\'atica)}
	\notacapaicmc{Disserta\c{c}\~ao de Mestrado do Programa de P\'os-Gradua\c{c}\~ao em \\Matem\'atica (PPG-Mat)}
    }{
% MMAe ==========================================================================
\ifthenelse{\equal{#1}{MMAe}}{
	\tipotrabalho{Disserta\c{c}\~ao (Mestrado em Ci\^encias)}
    \tipotrabalhoabs{Dissertation (Master in Science)}
	\renewcommand{\areaname}{Concentration area:}
    \area{Mathematics}
    \areaadic{\'Area de concentra\c{c}\~ao: Matem\'atica}
	%\opcao{Nome da Opção em inglês}
	%\opcaoadic{Nome da Opção em português}
    % O preambulo deve conter o tipo do trabalho, o objetivo, 
	% o nome da instituição, a área de concentração e opção quando houver
	\preambulo{Dissertation submitted to the Instituto de Ci\^encias Matem\'aticas e de Computa\c{c}\~ao, Universidade de S\~ao Paulo - ICMC/USP, in partial fulfillment of the requirements for the degree of the Master in Science - Mathematics.}
	\preambuloadic{Disserta\c{c}\~ao apresentada ao Instituto de Ci\^encias Matem\'aticas e de Computa\c{c}\~ao, Universidade de S\~ao Paulo - ICMC/USP, como parte dos requisitos para obten\c{c}\~ao do t\'itulo de Mestre em Ci\^encias - Matem\'atica.}
	\notaficha{Dissertation (Master - Program in Mathematics)}
	\notacapaicmc{Master\'{}s Dissertation of the Postgraduate Program in Mathematics (PPG-Mat)}
    }{
% DESp ==========================================================================
\ifthenelse{\equal{#1}{DESp}}{
    \tipotrabalho{Tese (Doutorado em Estat\'istica)}
    \tipotrabalhoabs{Thesis (Doctorate in Statistics)}
    \area{Estat\'istica}
    \areaadic{Concentration area: Statistics}
    \instituicao{Instituto de Ci\^encias Matem\'aticas e de Computa\c{c}\~ao, Universidade de S\~ao Paulo; Departamento de Estat\'istica, Universidade Federal de S\~ao Carlos}
	%\opcao{Nome da Opção em português}
	%\opcaoadic{Nome da Opção em inglês}
    % O preambulo deve conter o tipo do trabalho, o objetivo, 
	% o nome da instituição, a área de concentração e opção quando houver
	\preambulo{Tese apresentada ao Instituto de Ci\^encias Matem\'aticas e de Computa\c{c}\~ao, Universidade de S\~ao Paulo - ICMC/USP e ao Departamento de Estat\'istica, Universidade Federal de S\~ao Carlos - DEs/UFSCar, como parte dos requisitos para obten\c{c}\~ao do t\'itulo de Doutor em Estat\'istica - Interinstitucional de P\'os-Gradua\c{c}\~ao em Estat\'istica.}
	\preambuloadic{Thesis submitted to the Instituto de Ci\^encias Matem\'aticas e de Computa\c{c}\~ao, Universidade de S\~ao Paulo - ICMC/USP and to the Departamento de Estat\'istica, Universidade Federal de S\~ao Carlos - DEs/UFSCar, in partial fulfillment of the requirements for the degree of the Doctor in Statistics - Interagency Program Graduate in Statistics.}
	\notaficha{Tese (Doutorado - Interinstitucional de P\'os-Gradua\c{c}\~ao em Estat\'istica)}
	\notacapaicmc{Tese de Doutorado do Programa Interinstitucional de P\'os-Gradua\c{c}\~ao em \\Estat\'istica (PIPGEs)}
    }{
% DESe ==========================================================================
\ifthenelse{\equal{#1}{DESe}}{
	\tipotrabalho{Tese (Doutorado em Estat\'istica)}
    \tipotrabalhoabs{Thesis (Doctorate in Statistics)}
	\renewcommand{\areaname}{Concentration area:}
    \area{Statistics}
    \areaadic{\'Area de concentra\c{c}\~ao: Estat\'istica}
    \instituicao{Instituto de Ci\^encias Matem\'aticas e de Computa\c{c}\~ao, Universidade de S\~ao Paulo; Departamento de Estat\'istica, Universidade Federal de S\~ao Carlos}
	%\opcao{Nome da Opção em inglês}
	%\opcaoadic{Nome da Opção em português}
    % O preambulo deve conter o tipo do trabalho, o objetivo, 
	% o nome da instituição, a área de concentração e opção quando houver
	\preambulo{Thesis submitted to the Instituto de Ci\^encias Matem\'aticas e de Computa\c{c}\~ao, Universidade de S\~ao Paulo - ICMC/USP and to the Departamento
	de Estat\'istica, Universidade Federal de S\~ao Carlos - DEs/UFSCar, in partial fulfillment of the requirements for the degree of the Doctor in Statistics - Interagency Program Graduate in Statistics.}
	\preambuloadic{Tese apresentada ao Instituto de Ci\^encias Matem\'aticas e de Computa\c{c}\~ao, Universidade de S\~ao Paulo - ICMC/USP e ao Departamento de Estat\'istica, Universidade Federal de S\~ao Carlos - DEs/UFSCar, como parte dos requisitos para obten\c{c}\~ao do t\'itulo de Doutor em Estat\'istica - Interinstitucional de P\'os-Gradua\c{c}\~ao em Estat\'istica.}
	\notaficha{Thesis (Doctorate - Joint Graduate Program in Statistics)}
	\notacapaicmc{Doctoral Thesis of the Interagency Postgraduate Program in Statistics (PIPGEs)}
    }{     
% MESp ==========================================================================
\ifthenelse{\equal{#1}{MESp}}{
    \tipotrabalho{Disserta\c{c}\~ao (Mestrado em Estat\'istica)}
    \tipotrabalhoabs{Dissertation (Master in Statistics)}
    \renewcommand{\areaname}{Concentration area:}
    \area{Estat\'istica}
    \areaadic{Concentration area: Statistics}
    \instituicao{Instituto de Ci\^encias Matem\'aticas e de Computa\c{c}\~ao, Universidade de S\~ao Paulo; Departamento de Estat\'istica, Universidade Federal de S\~ao Carlos}
	%\opcao{Nome da Opção em português}
	%\opcaoadic{Nome da Opção em inglês}
    % O preambulo deve conter o tipo do trabalho, o objetivo, 
	% o nome da instituição, a área de concentração e opção quando houver
	\preambulo{Disserta\c{c}\~ao apresentada ao Instituto de Ci\^encias Matem\'aticas e de Computa\c{c}\~ao, Universidade de S\~ao Paulo - ICMC/USP e ao Departamento de Estat\'istica, Universidade Federal de S\~ao Carlos - DEs/UFSCar, como parte dos requisitos para obten\c{c}\~ao do t\'itulo de Mestre em Estat\'istica - Interinstitucional de P\'os-Gradua\c{c}\~ao em Estat\'istica.}
	\preambuloadic{Dissertation submitted to the Instituto de Ci\^encias Matem\'aticas e de Computa\c{c}\~ao, Universidade de S\~ao Paulo - ICMC/USP and to the Departamento de Estat\'istica- DEs, Universidade Federal de S\~ao Carlos - DEs/UFSCar, in partial fulfillment of the requirements for the degree of the Master in Statistics - Joint Graduate Program in Statistics.}
	\notaficha{Disserta\c{c}\~ao (Mestrado - Interinstitucional de P\'os-Gradua\c{c}\~ao em Estat\'istica)}
	\notacapaicmc{Disserta\c{c}\~ao de Mestrado do Programa Interinstitucional de \\P\'os-Gradua\c{c}\~ao em Estat\'istica (PIPGEs)}
    }{
% MESe ==========================================================================
\ifthenelse{\equal{#1}{MESe}}{
	\tipotrabalho{Disserta\c{c}\~ao (Mestrado em Estat\'istica)}
    \tipotrabalhoabs{Dissertation (Master in Statistics)}
    \renewcommand{\areaname}{Concentration area:}
	\area{Statistics}
    \areaadic{\'Area de concentra\c{c}\~ao: Estat\'istica}
    \instituicao{Instituto de Ci\^encias Matem\'aticas e de Computa\c{c}\~ao, Universidade de S\~ao Paulo; Departamento de Estat\'istica, Universidade Federal de S\~ao Carlos}
	%\opcao{Nome da Opção em inglês}
	%\opcaoadic{Nome da Opção em português}
    % O preambulo deve conter o tipo do trabalho, o objetivo, 
	% o nome da instituição, a área de concentração e opção quando houver
	\preambulo{Dissertation submitted to the Instituto de Ci\^encias Matem\'aticas e de Computa\c{c}\~ao, Universidade de S\~ao Paulo - ICMC/USP and to the Departamento de Estat\'istica - DEs, Universidade Federal de S\~ao Carlos - DEs/UFSCar, in partial fulfillment of the requirements for the degree of the Master in Statistics - Interagency Program Graduate in Statistics.} 
	\preambuloadic{Disserta\c{c}\~ao apresentada ao Instituto de Ci\^encias Matem\'aticas e de Computa\c{c}\~ao, Universidade de S\~ao Paulo - ICMC/USP e ao Departamento de Estat\'istica, Universidade Federal de S\~ao Carlos - DEs/UFSCar, como parte dos requisitos para obten\c{c}\~ao do t\'itulo de Mestre em Estat\'istica - Interinstitucional de P\'os-Gradua\c{c}\~ao em Estat\'istica.}
	\notaficha{Dissertation (Master - Joint Graduate Program in Statistics)}
	\notacapaicmc{Master\'{}s Dissertation of the Interagency Postgraduate Program in\\ Statistics (PIPGEs)}
	}{  
% DCCp ==========================================================================
\ifthenelse{\equal{#1}{DCCp}}{
    \tipotrabalho{Tese (Doutorado em Ci\^encias)}
    \tipotrabalhoabs{Thesis (Doctorate in Science)}
    \area{Ci\^encias de Computa\c{c}\~ao e Matem\'atica Computacional}
    \areaadic{Concentration area: Computer Science and Computational Mathematics} 
	%\opcao{Nome da Opção em português}
	%\opcaoadic{Nome da Opção em inglês}
    % O preambulo deve conter o tipo do trabalho, o objetivo, 
	% o nome da instituição, a área de concentração e opção quando houver
	\preambulo{Tese apresentada ao Instituto de Ci\^encias Matem\'aticas e de Computa\c{c}\~ao, Universidade de S\~ao Paulo - ICMC/USP, como parte dos requisitos para obten\c{c}\~ao do t\'itulo de Doutor em Ci\^encias - Ci\^encias de Computa\c{c}\~ao e Matem\'atica Computacional.}
	\preambuloadic{Thesis submitted to the Instituto de Ci\^encias Matem\'aticas e de Computa\c{c}\~ao, Universidade de S\~ao Paulo - ICMC/USP, in partial fulfillment of the requirements for the degree of the Doctor in Science - Program in Computer Science and Computational Mathematics.}
	\notaficha{Tese (Doutorado - Programa de P\'os-Gradua\c{c}\~ao em Ci\^encias de Computa\c{c}\~ao e Matem\'atica Computacional)}	 
	\notacapaicmc{Tese de Doutorado do Programa de P\'os-Gradua\c{c}\~ao em Ci\^encias de\\ Computa\c{c}\~ao e Matem\'atica Computacional (PPG-CCMC)}		
    }{
% DCCe ==========================================================================
\ifthenelse{\equal{#1}{DCCe}}{
    \tipotrabalho{Tese (Doutorado em Ci\^encias)}
    \tipotrabalhoabs{Thesis (Doctorate in Science)}
	\renewcommand{\areaname}{Concentration area:}
    \area{Computer Science and Computational Mathematics}
    \areaadic{\'Area de concentra\c{c}\~ao: Ci\^encias de Computa\c{c}\~ao e Matem\'atica Computacional}
	%\opcao{Nome da Opção em inglês}
	%\opcaoadic{Nome da Opção em português}
    % O preambulo deve conter o tipo do trabalho, o objetivo, 
	% o nome da instituição, a área de concentração e opção quando houver
	\preambulo{Thesis submitted to the Instituto de Ci\^encias Matem\'aticas e de Computa\c{c}\~ao, Universidade de S\~ao Paulo - ICMC/USP, in partial fulfillment of the requirements for the degree of the Doctor in Science - Program in Computer Science and Computational Mathematics.}
	\preambuloadic{Tese apresentada ao Instituto de Ci\^encias Matem\'aticas e de Computa\c{c}\~ao, Universidade de S\~ao Paulo - ICMC/USP, como parte dos requisitos para obten\c{c}\~ao do t\'itulo de Doutor em Ci\^encias - Ci\^encias de Computa\c{c}\~ao e Matem\'atica Computacional.}
	\notaficha{Thesis (Doctorate - Program in Computer Science and Computational Mathematics)}
	\notacapaicmc{Doctoral Thesis of the Postgraduate Program in Computer Science and\\ Computational Mathematics (PPG-CCMC)}	
    }{			
% MCCp ==========================================================================
\ifthenelse{\equal{#1}{MCCp}}{
    \tipotrabalho{Disserta\c{c}\~ao (Mestrado em Ci\^encias)}
    \tipotrabalhoabs{Dissertation (Master in Science)}
    \area{Ci\^encias de Computa\c{c}\~ao e Matem\'atica Computacional}
    \areaadic{Concentration area: Computer Science and Computational Mathematics}         
	%\opcao{Nome da Opção em português}
	%\opcaoadic{Nome da Opção em inglês}
    % O preambulo deve conter o tipo do trabalho, o objetivo, 
	% o nome da instituição, a área de concentração e opção quando houver
	\preambulo{Disserta\c{c}\~ao apresentada ao Instituto de Ci\^encias Matem\'aticas e de Computa\c{c}\~ao, Universidade de S\~ao Paulo - ICMC/USP, como parte dos requisitos para obten\c{c}\~ao do t\'itulo de Mestre em Ci\^encias - Ci\^encias de Computa\c{c}\~ao e Matem\'atica Computacional.}
	\preambuloadic{Dissertation submitted to the Instituto de Ci\^encias Matem\'aticas e de Computa\c{c}\~ao, Universidade de S\~ao Paulo - ICMC/USP, in partial fulfillment of the requirements for the degree of the Master in Science - Program in Computer Science and Computational Mathematics.}
	\notaficha{Disserta\c{c}\~ao (Mestrado - Programa de P\'os-Gradua\c{c}\~ao em Ci\^encias de Computa\c{c}\~ao e Matem\'atica Computacional)}
	\notacapaicmc{Disserta\c{c}\~ao de Mestrado do Programa de P\'os-Gradua\c{c}\~ao em Ci\^encias de\\ Computa\c{c}\~ao e Matem\'atica Computacional (PPG-CCMC)}	
    }{
% MCCe ==========================================================================
\ifthenelse{\equal{#1}{MCCe}}{
    \tipotrabalho{Disserta\c{c}\~ao (Mestrado em Ci\^encias)}
    \tipotrabalhoabs{Dissertation (Master in Science)}
	\renewcommand{\areaname}{Concentration area:}
    \area{Computer Science and Computational Mathematics}
    \areaadic{\'Area de concentra\c{c}\~ao: Ci\^encias de Computa\c{c}\~ao e Matem\'atica Computacional}
	%\opcao{Nome da Opção em inglês}
	%\opcaoadic{Nome da Opção em português}
    % O preambulo deve conter o tipo do trabalho, o objetivo, 
	% o nome da instituição, a área de concentração e opção quando houver
	\preambulo{Dissertation submitted to the Instituto de Ci\^encias Matem\'aticas e de Computa\c{c}\~ao, Universidade de S\~ao Paulo - ICMC/USP, in partial fulfillment of the requirements for the degree of the Master in Science - Program in Computer Science and Computational Mathematics.}
	\preambuloadic{Disserta\c{c}\~ao apresentada ao Instituto de Ci\^encias Matem\'aticas e de Computa\c{c}\~ao, Universidade de S\~ao Paulo - ICMC/USP, como parte dos requisitos para obten\c{c}\~ao do t\'itulo de Mestre em Ci\^encias - Ci\^encias de Computa\c{c}\~ao e Matem\'atica Computacional.}
	\notaficha{Dissertation (Master - Program in Computer Science and Computational Mathematics)}
	\notacapaicmc{Master's Dissertation of the Postgraduate Program in Computer Science and\\ Computational Mathematics (PPG-CCMC)}
    }{		
% MBACDp ==========================================================================
\ifthenelse{\equal{#1}{MBACDp}}{
	\tipotrabalho{Monografia (MBA em Ci\^encias de Dados)}
	\tipotrabalhoabs{Monograph (MBA in Data Sciences)}
	\area{Ci\^encias de Dados}
	\areaadic{Concentration area: Data Science} 
	%\opcao{Nome da Opção em português}
	%\opcaoadic{Nome da Opção em inglês}
	% O preambulo deve conter o tipo do trabalho, o objetivo, 
	% o nome da instituição, a área de concentração e opção quando houver
	\preambulo{Monografia apresentada ao Centro de Ci\^encias Matem\'aticas Aplicadas \`a Ind\'ustria do Instituto de Ci\^encias Matem\'aticas e de Computa\c{c}\~ao, Universidade de S\~ao Paulo - ICMC/USP, como parte dos requisitos para obten\c{c}\~ao do t\'itulo de Especialista em Ci\^encias de Dados.}
	\preambuloadic{Monograph presented to the Centro de Ci\^encias Matem\'aticas Aplicadas \`a Ind\'ustria do Instituto de Ci\^encias Matem\'aticas e de Computa\c{c}\~ao, Universidade de S\~ao Paulo - ICMC/USP, as part of the requirements for obtaining the title of Specialist in Data Science.}
	\instituicao{Centro de Ci\^encias Matem\'aticas Aplicadas \`a Ind\'ustria, Instituto de Ci\^encias Matem\'aticas e de Computa\c{c}\~ao, Universidade de S\~ao Paulo}
	\notaficha{Monografia (MBA em Ci\^encias de Dados)}
	\notacapaicmc{Monografia - MBA em Ci\^encia de Dados (CEMEAI)}
    }{    		
% MBACDe ==========================================================================
\ifthenelse{\equal{#1}{MBACDe}}{
	\tipotrabalho{Monografia (MBA em Ci\^encias de Dados)}
	\tipotrabalhoabs{Monograph (MBA in Data Sciences)}
	\renewcommand{\areaname}{Concentration area:}
	\area{Data Sciences}
	\areaadic{\'Area de concentra\c{c}\~ao: Ci\^encias de Dados}
 	%\opcao{Nome da Opção em inglês}
 	%\opcaoadic{Nome da Opção em português}
	% O preambulo deve conter o tipo do trabalho, o objetivo, 
	% o nome da instituição, a área de concentração e opção quando houver
	\preambulo{Monograph presented to the Centro de Ci\^encias Matem\'aticas Aplicadas \`a Ind\'ustria do Instituto de Ci\^encias Matem\'aticas e de Computa\c{c}\~ao, Universidade de S\~ao Paulo - ICMC/USP, as part of the requirements for obtaining the title of Specialist in Data Science.}
	\preambuloadic{Monografia apresentada ao Centro de Ci\^encias Matem\'aticas Aplicadas \`a Ind\'ustria do Instituto de Ci\^encias Matem\'aticas e de Computa\c{c}\~ao, Universidade de S\~ao Paulo - ICMC/USP, como parte dos requisitos para obten\c{c}\~ao do t\'itulo de Especialista em Ci\^encias de Dados.}
	\instituicao{Centro de Ci\^encias Matem\'aticas Aplicadas \`a Ind\'ustria, Instituto de Ci\^encias Matem\'aticas e de Computa\c{c}\~ao, Universidade de S\~ao Paulo}
	\notaficha{Monograph (MBA in Data Sciences)}
	\notacapaicmc{Monograph - MBA in Data Science (CEMEAI)}
    }{   
% MBAIAp ==========================================================================
\ifthenelse{\equal{#1}{MBAIAp}}{
	\tipotrabalho{Monografia (MBA em Intelig\^encia Artificial e Big Data)}
	\tipotrabalhoabs{Monograph (MBA in Artificial Intelligence and Big Data)}
	\area{Intelig\^encia Artificial}
	\areaadic{Concentration area: Artificial Intelligence} 
	%\opcao{Nome da Opção em português}
	%\opcaoadic{Nome da Opção em inglês}
	% O preambulo deve conter o tipo do trabalho, o objetivo, 
	% o nome da instituição, a área de concentração e opção quando houver
	\preambulo{Monografia apresentada ao Departamento de Ci\^encias de Computa\c{c}\~ao do Instituto de Ci\^encias Matem\'aticas e de Computa\c{c}\~ao, Universidade de S\~ao Paulo - ICMC/USP, como parte dos requisitos para obten\c{c}\~ao do t\'itulo de Especialista em Intelig\^encia Artificial e Big Data.}
	\preambuloadic{Monograph presented to the Departamento de Ci\^encias de Computa\c{c}\~ao do Instituto de Ci\^encias Matem\'aticas e de Computa\c{c}\~ao, Universidade de S\~ao Paulo - ICMC/USP, as part of the requirements for obtaining the title of Specialist in Artificial Intelligence and Big Data.}
	\notaficha{Monografia (MBA em Intelig\^encia Artificial e Big Data)}
	\notacapaicmc{Monografia - MBA em Intelig\^encia Artificial e Big Data}
    }{    		
% MBAIAe ==========================================================================
\ifthenelse{\equal{#1}{MBAIAe}}{
	\tipotrabalho{Monografia (MBA em Intelig\^encia Artificial e Big Data)}
	\tipotrabalhoabs{Monograph (MBA in Artificial Intelligence and Big Data)}
	\renewcommand{\areaname}{Concentration area:}
	\area{Artificial Intelligence and Big Data}
	\areaadic{\'Area de concentra\c{c}\~ao: Intelig\^encia Artificial e Big Data}
	%\opcao{Nome da Opção em inglês}
	%\opcaoadic{Nome da Opção em português}
	% O preambulo deve conter o tipo do trabalho, o objetivo, 
	% o nome da instituição, a área de concentração e opção quando houver
	\preambulo{Monograph presented to the Departamento de Ci\^encias de Computa\c{c}\~ao do Instituto de Ci\^encias Matem\'aticas e de Computa\c{c}\~ao, Universidade de S\~ao Paulo - ICMC/USP, as part of the requirements for obtaining the title of Specialist in Artificial Intelligence and Big Data.}
	\preambuloadic{Monografia apresentada ao Departamento de Ci\^encias de Computa\c{c}\~ao do Instituto de Ci\^encias Matem\'aticas e de Computa\c{c}\~ao, Universidade de S\~ao Paulo - ICMC/USP, como parte dos requisitos para obten\c{c}\~ao do t\'itulo de Especialista em Intelig\^encia Artificial e Big Data.}
	\notaficha{Monograph (MBA in Artificial Intelligence and Big Data)}
	\notacapaicmc{Monograph - MBA in Artificial Intelligence and Big Data}
    }{  
% MBASDp ==========================================================================
\ifthenelse{\equal{#1}{MBASDp}}{
	\tipotrabalho{Monografia (MBA em Seguran\c{c}a de Dados)}
	\tipotrabalhoabs{Monograph (MBA in Data Security)}
	\area{Seguran\c{c}a de Dados}
	\areaadic{Concentration area: Data Security} 
	%\opcao{Nome da Opção em português}
	%\opcaoadic{Nome da Opção em inglês}
	% O preambulo deve conter o tipo do trabalho, o objetivo, 
	% o nome da instituição, a área de concentração e opção quando houver
	\preambulo{Monografia apresentada ao Centro de Ci\^encias Matem\'aticas Aplicadas \`a Ind\'ustria do Instituto de Ci\^encias Matem\'aticas e de Computa\c{c}\~ao, Universidade de S\~ao Paulo - ICMC/USP, como parte dos requisitos para obten\c{c}\~ao do t\'itulo de Especialista em Seguran\c{c}a de Dados.}
	\preambuloadic{Monograph presented to the Centro de Ci\^encias Matem\'aticas Aplicadas \`a Ind\'ustria do Instituto de Ci\^encias Matem\'aticas e de Computa\c{c}\~ao, Universidade de S\~ao Paulo - ICMC/USP, as part of the requirements for obtaining the title of Specialist in Artificial Intelligence and Big Data.}
	\notaficha{Monografia (MBA em Seguran\c{c}a de Dados)}
	\notacapaicmc{Monografia - MBA em Seguran\c{c}a de Dados (CEMEAI)}
    }{    		
% MBASDe ==========================================================================
\ifthenelse{\equal{#1}{MBASDe}}{
	\tipotrabalho{Monografia (MBA em Seguran\c{c}a de Dados)}
	\tipotrabalhoabs{Monograph (MBA in Data Security)}
	\renewcommand{\areaname}{Concentration area:}
	\area{Data Security}
	\areaadic{\'Area de concentra\c{c}\~ao: Seguran\c{c}a de Dados}
	%\opcao{Nome da Opção em inglês}
	%\opcaoadic{Nome da Opção em português}
	% O preambulo deve conter o tipo do trabalho, o objetivo, 
	% o nome da instituição, a área de concentração e opção quando houver
	\preambulo{Monograph presented to the Centro de Ci\^encias Matem\'aticas Aplicadas \`a Ind\'ustria do Instituto de Ci\^encias Matem\'aticas e de Computa\c{c}\~ao, Universidade de S\~ao Paulo - ICMC/USP, as part of the requirements for obtaining the title of Specialist in Artificial Intelligence and Big Data.}
	\preambuloadic{Monografia apresentada ao Centro de Ci\^encias Matem\'aticas Aplicadas \`a Ind\'ustria do Instituto de Ci\^encias Matem\'aticas e de Computa\c{c}\~ao, Universidade de S\~ao Paulo - ICMC/USP, como parte dos requisitos para obten\c{c}\~ao do t\'itulo de Especialista em Seguran\c{c}a de Dados.}
	\notaficha{Monograph (MBA in Data Security)}
	\notacapaicmc{Monograph - MBA in Data Security (CEMEAI)}
    }{  
% Outros
	\tipotrabalho{Disserta\c{c}\~ao/Tese (Mestrado/Doutorado)}
	\tipotrabalhoabs{Dissertation/Thesis (Master/Doctor)}
	\area{Nome da \'Area}
	\opcao{Nome da Op\c{c}\~ao}
	\areaadic{Additional area}
	%\opcao{Nome da Opção em português}
	%\opcaoadic{Nome da Opção em inglês}
    % O preambulo deve conter o tipo do trabalho, o objetivo, 
	% o nome da instituição, a área de concentração e opção quando houver				
	\preambulo{Disserta\c{c}\~ao/Tese apresentada ao Instituto de Ci\^encias Matem\'aticas e de Computa\c{c}\~ao, Universidade de S\~ao Paulo - ICMC/USP, como parte dos requisitos para obten\c{c}\~ao do t\'itulo de Mestre/Doutor em Ci\^encias - Programa.}
	\preambuloadic{Dissertation/Thesis submitted to the Instituto de Ci\^encias Matem\'aticas e de Computa\c{c}\~ao, Universidade de S\~ao Paulo - ICMC/USP, in partial fulfillment of the requirements for the degree of the Master/Doctor in Science - Program.}
	\notaficha{Disserta\c{c}\~ao/Tese (Mestrado/Doutorado - Programa)}
	\notacapaicmc{Master's Dissertation/Doctoral Thesis of the Postgraduate Program in ...}
        }}}}}}}}}}}}}}}}}}}}}}}		







