A notable difference from our method to others in MAPF is that our algorithm remains consistent across any grid dimension. That is, the algorithm for a 2D grid obtains solutions in the same way as for a 3D or 4D grid.

This fact becomes clear when we see that both our methods, the heuristic and the exact model, are based solely on the topology of the graph. There is no spatial dependency. When we increase the dimension to 3D, we are simply increasing the neighborhood of each node. In some sense, each vertex $v$ in 3D space will have two additional edges in its neighborhood (up and down).

Again, using the notation in Table \ref{tab:notation}, we can create a temporal graph $\mathcal{G}$ that naturally represents our problem in three dimensional space. The set of vertices $\mathcal{V} := (i,j,k,t)$ represents every possible position $(i,j,k)$ at any given time $t$ in our problem. Our definition is almost the same, the difference is we add one more index $k$ and consequently the possibility of the drone going up($k+1$) and down($k-1$). We achieve this by defining the set of outgoing edges for a vertex $v^* = (i^*,j^*,k^*,t^*)$ as $\mathcal{E}_{v^*} = \{ 
(i^*+1,j^*,k^*,t^*+1), 
(i^*,j^*+1,k^*,t^*+1), 
(i^*-1,j^*,k^*,t^*+1), 
(i^*,j^*-1,k^*,t^*+1), 
(i^*,j^*,k^*+1,t^*+1),
(i^*,j^*,k^*-1,t^*+1),
(i^*,j^*,k^*,t^*+1)
\}$.

