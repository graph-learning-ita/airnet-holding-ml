% ----------------------------------------------------------
% Introdução
% ----------------------------------------------------------
\chapter[Introduction]{Introduction}
\label{Introdução}

\noindent The aviation industry increasingly relies on data-driven
approaches to improve operational efficiency and reduce delays. Among
the pressing challenges in air traffic management is the prediction of
`holding' maneuvers, where aircraft are instructed to delay landing
due to factors such as airport congestion, adverse weather, or
airspace limitations. While holding patterns are necessary for safety,
they contribute to increased fuel consumption, emissions, and
passenger dissatisfaction. This project aims to enhance the accuracy
of holding predictions using machine learning (ML) models based on
graph-structured data, specifically employing advanced methodologies
in Graph Machine Learning (Graph ML) and Graph Neural Networks (GNNs).

Traditional machine learning applications in aviation have primarily
focused on flight delay prediction and air traffic flow
management. For instance, delay predictions based on weather
conditions, airport congestion, and flight schedules have been widely
studied \cite{lambelho2020assessing, gui2019flight}. However, these
models often rely on tabular data representations, which limit their
ability to capture complex relational patterns among airports and
other influencing factors. Additionally, research specifically focused
on holding maneuvers is limited and generally lacks machine learning
and network-based approaches that can model the spatial and temporal
dependencies intrinsic to air traffic data \cite{lee2020development,
  smith2008management}.

The use of graph-based machine learning methods is rapidly advancing
in the field of intelligent transportation, where graph structures
effectively capture complex spatial and temporal relationships across
networks. A recent survey on GNNs in intelligent transportation
systems, \citeonline{gnnSurveyTransportation}, highlights their
application across a variety of domains, including traffic
forecasting, demand prediction, and urban planning. This survey
underscores the power of GNNs in applications where data is inherently
interconnected, such as autonomous vehicle routing and intersection
management. By organizing studies within these domains, they identify
distinct opportunities and challenges, particularly in multi-modal
models and reinforcement learning applications. Similarly,
\citeonline{zhao2020tgcn} demonstrate the value of GNNs in the
specific context of real-time traffic forecasting with their T-GCN
(Temporal Graph Convolutional Network) model. By combining Graph
Convolutional Networks (GCN) and Gated Recurrent Units (GRU), the
T-GCN model captures both spatial and temporal dependencies, achieving
state-of-the-art accuracy in urban traffic prediction tasks. These
studies exemplify the increasing role of GNNs in
transportation-related decision-making, showing potential for improved
accuracy and efficiency in complex, dynamic systems.

The project employs two main approaches:

\begin{itemize}
\item \textbf{Tabular-based approach:} We use the CatBoost model,
  leveraging graph features—such as centrality and connectivity
  metrics—that capture the significance of directed edges (flights)
  within the network \cite{prokhorenkova2018catboost}.
\item \textbf{Graph Attention Network (GAT) approach:} We compare
  with GATs that have proven effective in applications where
  relational data is essential, making them a promising choice for
  capturing the interconnected nature of air traffic
  \cite{velickovic2017graph}.
\end{itemize}

The contributions of this project are twofold. First, we introduce a
novel application of graph-based ML models for predicting holding
events, providing a refined view of airport interdependencies. Second,
by comparing CatBoost with the GAT model, we evaluate which approach
better captures the graph topology of the air traffic and have a
better predictive performance. This research has potential
implications for improved fuel efficiency, reduced delays, and
enhanced passenger experiences by refining model selection and feature
engineering strategies tailored to aviation applications. For the best
of our knowledge there is no previous work that approached the problem
of flight delay due to holding maneuver using Graph ML.

The structure of this work is as follows. In Section
\ref{TheoreticalFramework}, we review relevant literature and build
the theoretical background needed. Section \ref{Materials} details the
dataset and the modelling. Section \ref{Results} discusses the
experimental setup, with results and comparison analysis presented and
model deployment. Finally, Section \ref{Conclusion} provides
concluding remarks about my Computer Science Bachelor's.
