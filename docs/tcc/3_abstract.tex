\begin{resumo}[Abstract]
  \begin{otherlanguage*}{english}
	\begin{flushleft}
      \setlength{\absparsep}{0pt} % ajusta o espaçamento da
      referência \SingleSpacing
      \imprimirautorabr~~\textbf{\imprimirtituloresumo}. \imprimirdata. \pageref{LastPage}p.
      % Substitua p. por f. quando utilizar oneside em
      % \documentclass \pageref{LastPage}f.
      \imprimirtipotrabalho~-~\imprimirinstituicao,
      \imprimirlocal, \imprimirdata.
 	\end{flushleft} \OnehalfSpacing

    This project models the prediction of flight delays due to holding
    maneuvers as a graph problem, leveraging advanced Graph Machine
    Learning (Graph ML) techniques to capture complex interdependencies in
    air traffic networks. Holding maneuvers, while crucial for safety,
    cause increased fuel usage, emissions, and passenger dissatisfaction,
    making accurate prediction essential for operational
    efficiency. Traditional machine learning models, typically using
    tabular data, often overlook spatial-temporal relations within air
    traffic data. To address this, we model the problem of predicting
    holding as edge feature prediction in a directed (multi)graph where we
    apply both CatBoost, enriched with graph features capturing network
    centrality and connectivity, and Graph Attention Networks (GATs),
    which excel in relational data contexts. Our results indicate that
    CatBoost outperforms GAT in this imbalanced dataset, effectively
    predicting holding events and offering interpretability through
    graph-based feature importance. Additionally, a web-based tool,
    Airdelay, allows users to simulate real-time delay predictions,
    demonstrating the model's potential operational impact. This research
    underscores the viability of graph-based approaches for predictive
    analysis in aviation, with implications for enhancing fuel efficiency,
    reducing delays, and improving passenger experience.

    \vspace{\onelineskip}

    \noindent \textbf{Keywords}: Graph Neural Networks. Graphs. Machine
    Learning. Complex Networks.
  \end{otherlanguage*}
\end{resumo}
