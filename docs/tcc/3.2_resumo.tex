%% USPSC-Resumo.tex
\setlength{\absparsep}{18pt} % ajusta o espaçamento dos parágrafos do
resumo
\begin{resumo}
  \begin{flushleft}
    \setlength{\absparsep}{0pt} % ajusta o espaçamento da
    referência \SingleSpacing
    \imprimirautorabr~~\textbf{\imprimirtituloresumo}. \imprimirdata. \pageref{LastPage}p.
    % Substitua p. por f. quando utilizar oneside em
    % \documentclass \pageref{LastPage}f.
    \imprimirtipotrabalho~-~\imprimirinstituicao,
    \imprimirlocal, \imprimirdata.
  \end{flushleft} \OnehalfSpacing

  Este projeto modela a previsão de atrasos de voo devido a manobras de
  espera como um problema de grafo, aproveitando técnicas avançadas de
  Aprendizado de Máquina com Grafos (Graph ML) para capturar
  interdependências complexas em redes de tráfego aéreo. Manobras de
  espera, embora essenciais para a segurança, causam aumento no uso de
  combustível, emissões e insatisfação dos passageiros, tornando a
  previsão precisa essencial para a eficiência operacional. Modelos
  tradicionais de aprendizado de máquina, que geralmente utilizam dados
  tabulares, muitas vezes não consideram as relações espaço-temporais
  nos dados de tráfego aéreo. Para abordar essa questão, modelamos o
  problema da previsão de espera como uma predição de características de
  arestas em um (multi)grafo direcionado, onde aplicamos tanto o
  CatBoost, enriquecido com características de grafo que capturam
  centralidade e conectividade da rede, quanto Graph Attention Networks
  (GATs), que são eficazes em contextos de dados relacionais. Nossos
  resultados indicam que o CatBoost supera o GAT nesse conjunto de dados
  desbalanceado, prevendo eficazmente eventos de espera e oferecendo
  interpretabilidade por meio da importância dos recursos baseados em
  grafo. Além disso, uma ferramenta web chamada Airdelay permite que os
  usuários simulem previsões de atraso em tempo real, demonstrando o
  impacto operacional potencial do modelo. Esta pesquisa destaca a
  viabilidade de abordagens baseadas em grafos para análise preditiva na
  aviação, com implicações para aumentar a eficiência do combustível,
  reduzir atrasos e melhorar a experiência do passageiro.

  \vspace{\onelineskip}

  \noindent \textbf{Palavras-chave}: Redes Neurais de
  Grafos. Grafos. Aprendizado de Máquina. Redes Complexas.


\end{resumo}
