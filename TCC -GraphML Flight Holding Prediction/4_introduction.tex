%% USPSC-Introducao.tex

% ----------------------------------------------------------
% Introdução (exemplo de capítulo sem numeração, mas presente no Sumário)
% ----------------------------------------------------------
\chapter[Introduction]{Introduction}
\label{Introdução}

% Efficient last-mile delivery logistics is vital for accurate and timely delivery of goods, which influences customer satisfaction and overall business success \cite{boysen2021lastmile}. The use of drones, also called Unmanned Aerial Vehicles (UAVs), in Last-Mile Delivery improves efficiency by overcoming traffic constraints, reducing delivery times, and lowering operational costs.  For this reason, the literature on Delivery Drones increased significantly in the last few years, as analyzed in \citeonline{DUKKANCI2023}.

% The literature related to Last-Mile Delivery Drones is highly heterogeneous, encompassing a diverse array of techniques and problem formulations. This diversity spans from studies involving both drones and trucks in delivery scenarios, linear integer modeling for logistic problem-solving, applications of fuzzy logic to address uncertainties, multi-objective optimization to consider multiple criteria simultaneously, to exclusive investigations focusing solely on drones without the presence of other vehicles, including decentralized models.

% In \citeonline{FREITAS2023100094} it is studied the Truck-Drone Delivery Problem, which can be addressed as a variation of TSP (Traveling Salesman Problem) for two agents, using a Mixed Integer Programming (MIP) formulation and a heuristic based on a TSP solver and two metaheuristics: General Variable Neighborhood Search (GVNS) and Tabu Search (TS). At the same time, \citeonline{MOSHREFJAVADI2020290} addresses a near problem, but with multiple drones, using Mixed Integer Linear Programming (MILP) and a heuristic based on Adaptive Large Neighborhood Search (ALNS) and Simulated Annealing (SA). A similar MILP modeling was used in \citeonline{drones7070407} and two hybrid genetic algorithms were evaluated.



% When addressing multiple criteria simultaneously in collaborative Truck-Drone Delivery, researchers have employed multi-objective optimization techniques.  \citeonline{9580555} consider flexible time windows and diverse objectives, including minimizing the routing costs of drones and trucks and maximizing customer satisfaction. They improved NSGA-II(Non-dominated Sorting Genetic Algorithm) and used a posterior heuristic for Pareto Local Search(PLS).   Complementing this, \citeonline{ZHANG2022108679} introduces a novel multi-objective optimization model for drone delivery, optimizing economic and environmental goals concurrently. The model addresses dynamic drone flight endurance based on loading rates, using an extended NSGA-II algorithm that proves to be effective in generating high-quality solutions, marking advancements in this field.

% %Now, article fuzzy, sentiment and energy.

% In another branch, works such as \citeonline{farjana2020last} introduce a systematic multi-criterion, multipersonnel decision-making approach called interval-valued inferential fuzzy TOPSIS, handling fuzziness in decision-making and providing quality drone selection decisions. Another study \citeonline{9672856} considers sustainability in optimizing the vehicle routing problem with drones, utilizing sentiment analysis based on Twitter to determine customer sentiments on environmental protection. The net promoter score index calculated from the sentiment analysis is considered as a coefficient in the penalty function added to the base model, aiming to help logistics companies improve the sustainability of the supply chain. Addressing the aspect of energy efficiency in drone-based last mile delivery, another study \citeonline{10049551} focuses on tactical decisions about the selection of shared fulfillment centers used as launch and recovery locations for drones, fleet size plans, and operational drone route decisions. 

% %até aqui sem collision

% A new perspective in Last Mile Delivery Drones was introduced in \citeonline{Verri}. The study addressed the last-mile delivery problem from a complex system viewpoint, where the collective performance of the drones is investigated. The delivery system incorporates a \textit{tradable permit model} \cite{AKAMATSU2017178}, based on a financial decentralized market-inspired approach, for airspace use, requiring drones to compete for airspace permits in a distributed manner. The simulation evaluates how different parameters, such as arrival rate and airspace dimensions, impact system behavior in terms of cost, time required for drones to acquire flight permits and airspace utilization. Although the simulation employs a simplified model with naïve agents and disregards drone flight dynamics, it captures interesting properties in the agents' collective behavior, demonstrating satisfactory system performance even under high traffic conditions. Simultaneously, \citeonline{lee2022last} contributes a novel combinatorial double auction bi-objective winner determination problem for last-mile drone delivery. 

% As we can see, the Last Mile Delivery Drone problem is very hard and complex. In \citeonline{faical}, the authors characterize the problem through a cyber-physical lens, highlighting challenges related to the physical aspects of airspace in the context of drone operations for last-mile delivery. Their analysis delves into considerations such as air traffic management systems, focusing on the evolving landscape of airspace control. This review underscores the need for optimal usage of airspace, then ensuring the need for collision avoidance. However, collision avoidance is a problem that has not been addressed by most of the cited papers until now. \citeonline{Verri} addresses this systematically within the decentralized approach. Indeed, \citeonline{DUKKANCI2023} highlights collision avoidance and air traffic management as a crucial aspect that needs attention in Last-mile delivery drone problems, as much of the literature tends to overlook this when modeling the problem. 


% With this problem of collision avoidance and efficient airspace control in mind, the main idea of our work is solve the Last Mile Delivery Drone problem using a MAPF (Multi-Agent Path Finding) approach, thus not ignoring spatial characteristics natural from the airspace. The problem addressed in this work can easily be seen as multi-objective drone path planning for search and rescue \cite{hayat2020multi}, thus the Last Mile Delivery Drone is per se a MAPF.  As stated in \citeonline{lavalle} and proved in \citeonline{Nebel_2020} the MAPF problem is NP-hard. Then our approach is to use elements of prioritized planning \cite{7138650} and conflict-based search \cite{SHARON201540} to face the computational complexity inherited from MAPF while satisfying the constraints of the reduced problem. Although our algorithm does not guarantee global optimum, it always converges in a finite and bounded polynomial time, which is an improvement compared to the previous algorithms described ( MILPs, MOEAs, Metaheuristics).



% Centralized control for airspace management, particularly under the Unmanned Aircraft System Traffic Management (UTM) framework developed by the Federal Aviation Administration (FAA) and NASA, exemplifies the necessity of organized, legislative-backed airspace control \cite{nasa}. The UTM facilitates multiple drone operations beyond visual line of sight (BVLOS) by enabling cooperative interaction between drone operators and the FAA, determining and communicating real-time airspace status. While decentralized models, such as those based on blockchain technology \cite{Verri}, offer novel approaches, they introduce complexities in scalability, regulatory compliance, and operational efficiency that can hinder real-time airspace control. The centralized UTM system, supported by regulatory compliance and legislative backing, ensures effective airspace management, safety, and reliability, fostering the integration of UAV deliveries within existing air traffic systems. The FAA's UTM Implementation Plan \cite{9256745} and ongoing evaluations highlight the continual refinement of this centralized approach, emphasizing its critical role in future UAV operations. 


% In this study, we propose a novel approach to tackle the Last Mile Delivery Drone problem by employing a Multi-Agent Path Finding (MAPF) strategy. Leveraging elements of prioritized planning and conflict-based search, our algorithm aims to address the computational complexity inherent in the MAPF problem. Unlike previous algorithms such as MILPs and MOEAs, our heuristic guarantees convergence in finite and bounded polynomial time, representing a significant advancement. Our graph-based approach, employing Breadth-First Search (BFS) on temporal grids, enhances the efficiency of our solution. Also, we propose a hybrid method, that combines the heuristic and the MILP. We further present experimental results, comparing our approach with a MILP-based solution and evaluating its performance in the hybrid approach that combines the time found by the heuristic and the MILP solution. Finally, a qualitative comparison between our centralized approach and the decentralized, proposed in \citeonline{Verri}, is presented. The use of a graph-based heuristic offers distinctive advantages, and our experiments shed light on its efficacy in solving the complex Last Mile Delivery Drones problem.

% Therefore, in what follows, we describe the main contributions
% of this work:
% \begin{itemize}
%     \item We develop a MILP model to solve the LMDD problem exactly. This model offers a novel method to adapt the MAPF solution to the LMDD context, allowing for drone take-offs, waiting in airspace, and correct landings.
%     \item We propose a heuristic algorithm to address the LMDD problem. The primary advantage of our heuristic lies in its graph-based structure, which can be adapted to higher dimensions and weighted sets in the LMDD context. Additionally, our heuristic is polynomially bounded with a low computational cost, representing a novel contribution to the current literature on drone delivery solutions.
% \end{itemize}




% This monograph is organized as follows. In Section 2 we present the problem definition and the new property. The proposed solution approach is described in Section 3, while the results of the computational experiments are presented in Section 4. Finally, in Section 5 conclusions are drawn.

