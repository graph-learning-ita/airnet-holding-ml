\chapter[Problem Description]{Problem Description}
\label{ProblemDescription}


% In this section details about the formulation of the Last Mile Delivery Drones (LMDD) as a MAPF in a grid and a mathematical model of the problem are presented. At the core of MAPF lies the objective of crafting collision-free paths for a set of agents, each navigating from individual starting points to, respectively, goal destinations. As delineated in \citeonline{ijcai2022p0783}, tackling this issue generally involves the use of compilation, a notable computing technique. This process is defined by the conversion of an input instance from its original formalism into a different, usually well-established formalism, thereby creating a reduced problem for which an efficient solver exists. This reduction is crucial, as it essentially simplifies the original problem into a more manageable form, optimizing the problem-solving process.

%  The Last Mile Delivery Drones problem can be reduced as a set of tasks for each agent(drone), where each task is described by two tuples  $task_i \coloneq \{(x_{\text{start}_i}, y_{\text{start}_i}) , (x_{\text{goal}_i}, y_{\text{goal}_i})\} ,$ $\forall $ drone $d_i$, with $x,y \in \mathbb{Z}_{\geq 0} $, describing the start position of the drone and the final objective position. In our discretization we make the grid finite, where each axis has bounds, i.e. $x \leq X, y \leq Y$, with $X$ and $Y$ being the bounds of the grid. Our goal is to find paths $P_{d_{i}} \coloneq \{(x_{\text{start}}, y_{\text{start}}, t_{\text{begin}} ), (x_{\text{decision}_1}, y_{\text{decision}_1}, t_{\text{begin}}+1),(x_{\text{decision}_2}, y_{\text{decision}_2}, t_{\text{begin}}+2) \dots,  (x_{\text{goal}}, y_{\text{goal}} , t_{\text{begin}} + | P_{d_i} | ) \} $ for each drone $d_i$ making the path length as short as possible.

% Navigational constraints dictate that each drone's movement is restricted to parallel advancements along the grid axes, thus excluding the possibility of diagonal traversals to streamline pathfinding complexity. Consequently, each drone is limited to four principal movements: upward $(x, y+1)$, downward $(x, y-1)$, rightward $(x+1, y)$, and leftward $(x-1, y)$ shifts. Formally, a position within the grid $(x_1,y_1)$ is adjacent to $(x_2,y_2)$  $\iff |x_2-x_1| + |y_2-y_1| = 1$ . 

% It is clear how close this formulation is to the MAPF problem. In fact, the only difference is now we have a new decision variable, the time that each drone starts their flight($t_{\text{begin}}$), i.e, the arrival time when the drone takeoff. The analogy with MAPF enhances the understanding of the problem and leverages the literature of the LMDD problem to a new paradigm. 

% There are currently approximates, compilation and AI search based solutions to the MAPF problem \cite{ma2019searching}. But there is a lack of MILP based formulations for the problem \cite{stern2019multi}. Branch Cut and Price \cite{BCPijcai2019} is a MILP solution that hybridizes these approaches, using a decomposition framework.

% The cited solutions to MAPF are not quite intuitive and leave a hard work for generalization. However, in \citeonline{lavalle} it is shown the isomorphism(equivalence) between the MAPF and multi-commodity minimum cost maximum flow problem. This work let an excellent and easy visualization of MAPF problems. In fact, they can all be seem as a Network Flow problem. That being said, we propose a network-based formulation of the problem.

% The goal of our LMDD formulation is: given an available time $T$, minimize the sum of distances of all drones, while allowing drones to wait in cells of the grid and choose when they enter at the airspace(the grid). Actually, these two additions are what differentiates our formulation from standardly MAPFs as \citeonline{BCPijcai2019} and \citeonline{lavalle}. Also, the proposed LMDD problem has non-weighted distances, what make the problem easier then the MAPF itself. As evidenced in \ref{graph-modeling} this formulation could be easily addressed if modeled as a graph, where our goal is to find the paths that generates the minimum flow in the network/graph. 

% \section{MAPF as a multi-commodity Network Flow Problem}

% As shown in \citeonline{yu_timeoptimal}, the Multi-Agent Pathfinding (MAPF) problem can be transformed into a network flow problem using a time-expanded network and multi-commodity flow approach. This transformation involves several key steps and concepts, illustrated by the figures below.

% \subsection{Time-Expanded Network}

% The first step is to construct a time-expanded network. In this representation, each node in the original grid is duplicated for each time step, creating a layered network where each layer corresponds to a specific time instance. Edges are added to represent possible moves between nodes over time, including waiting in place. This transformation allows us to model the time dimension explicitly within the network.

% \begin{figure}[H]
%     \centering
%     \includegraphics[width=0.8\textwidth]{img/time_extended_net_image.pdf}
%     \caption[Time-expanded network representation]{A time-expanded network representation. Picture sourced from \citeonline{time_optimal_slides}.}
%     \label{fig:time_expanded_network}
% \end{figure}


% \subsection{Multi-Commodity Flow Formulation}

% In this transformed network, the MAPF problem is equivalent to a multi-commodity flow problem. Each drone is considered a separate commodity that needs to flow from its start node to its goal node through the network. The key constraints include ensuring that each drone reaches its goal within a given time horizon $T$.

% \begin{figure}[H]
%     \centering
%     \includegraphics[width=0.8\textwidth]{img/equi_mapf_flow_image.pdf}
%     \caption{Equivalence of MAPF to multi-commodity network flow (sourced from \citeonline{time_optimal_slides})}
%     \label{fig:equi_mapf_flow}
% \end{figure}

% The equivalence between MAPF and multi-commodity network flow is fundamental because it allows the application of well-established network flow algorithms to solve MAPF problems. By leveraging this equivalence, it is possible to find paths for all drones in a computationally efficient manner, especially when combined with integer linear programming (ILP) techniques.

% This transformation not only provides a clearer visualization of the problem but also enhances the ability to apply optimization techniques to find solutions. Consequently, it supports the development of more efficient algorithms for last-mile delivery drones and other applications requiring coordinated multi-agent pathfinding.
