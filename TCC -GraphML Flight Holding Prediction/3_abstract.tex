\begin{resumo}[Abstract]
\begin{otherlanguage*}{english}
	\begin{flushleft} 
			\setlength{\absparsep}{0pt} % ajusta o espaçamento da referência	
			\SingleSpacing 
			\imprimirautorabr~~\textbf{\imprimirtituloresumo}.	\imprimirdata. \pageref{LastPage}p. 
			%Substitua p. por f. quando utilizar oneside em \documentclass
			%\pageref{LastPage}f.
			\imprimirtipotrabalho~-~\imprimirinstituicao, \imprimirlocal, \imprimirdata. 
 	\end{flushleft}
\OnehalfSpacing
	
The growing demand for efficient last-mile delivery solutions, driven by the rise of e-commerce and urbanization, necessitates innovative approaches to manage the complexities of urban logistics. This study investigates the use of drones for last-mile delivery through a graph-based approach, focusing on the Multi-Agent Pathfinding (MAPF) problem, which involves routing multiple drones to deliver packages efficiently. Most algorithms proposed for the Last Mile Delivery Drones (LMDD) problem tend to overlook the inherent complexities of the MAPF problem. This research adopts a graph-based representation of the delivery area, transforming the problem into a network flow optimization. The proposed graph-based heuristic method is evaluated against a purely Mixed Integer Linear Programming (MILP)-based solution. Experimental results demonstrate that the heuristic approach not only enhances computational efficiency but also maintains high solution quality. Specifically, the heuristic outperforms MILP in terms of runtime while achieving comparable accuracy in path optimization. Additionally, a hybrid implementation combining heuristic and MILP is proposed. The exact MILP model has exponential complexity time as the number of drones is increased, which is expected since the MAPF paradigm is NP-Complete. The complexity of the heuristic is bounded by \(\mathcal{O}(N^3 K \max{(\log N, \log K)})\), where $K$ is the number of drones and $N$ is the dimension, if the grid is square. The study highlights the potential of centralized control systems for managing a fleet of delivery drones. The findings from extensive simulations indicate that the graph-based heuristic effectively balances computational efficiency and operational reliability, making it a viable solution for real-world last-mile delivery applications. This research contributes to the broader field of drone logistics by offering a scalable and robust method for optimizing drone delivery paths, thereby supporting the integration of drones into commercial delivery systems.  


   \vspace{\onelineskip}
 
   \noindent 
   \textbf{Keywords}: Optimization. Graphs. Last Mile Delivery Drones.
 \end{otherlanguage*}
\end{resumo}
