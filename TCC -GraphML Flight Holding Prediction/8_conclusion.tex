%% USPSC-Cap3-Conclusao.tex
% ---
% Conclusão
% ---
\chapter{Conclusion}
\label{Conclusion}
This chapter summarizes the findings and contributions of this research, highlights the relationship between the undergraduate course and the project, discusses the implications and limitations of the study, and suggests directions for future research.

\section{Summary of the Project and Objectives}
The primary objective of this research was to develop and analyze an exact model and a heuristic for optimizing drone routing in centralized last-mile delivery scenarios. This involved creating a heuristic algorithm implemented in C\texttt{++} and comparing its performance against the exact model implemented in Julia. The study focused on evaluating the heuristic's efficiency and effectiveness under high-traffic conditions and comparing its results to the optimal paths determined by the exact model.

\section{Main Findings and Results}
The experiments demonstrated that the heuristic algorithm, while not always achieving the optimal solution, provided satisfactory results with significantly lower computational time compared to the exact model. Key findings include:
\begin{itemize}
    \item The heuristic algorithm achieved optimal results for scenarios with a single drone, as BFS guarantees optimality in such cases.
    \item As the number of drones increased, the heuristic's performance decreased, which is expected due to higher traffic in the fixed grid size.
    \item The heuristic maintained sub-linear performance concerning the number of drones, indicating its scalability for larger problems.
    \item The exact model's computational time increased exponentially with the number of drones, consistent with its NP-complete nature.
    \item The use of a hybrid methodology, as described in \ref{Hybrid_Methodology}, would be a good choice.
\end{itemize}

\section{Contributions to the Field and Implications}
This research contributes to the field of drone logistics and optimization by:
\begin{itemize}
    \item Providing a practical and efficient heuristic for drone routing that can be applied in real-world scenarios with multiple drones.
    \item Highlighting the trade-offs between computational efficiency and optimality in heuristic algorithms.
    \item Demonstrating the applicability of heuristic methods in high-traffic conditions, offering a scalable solution for last-mile delivery challenges.
\end{itemize}

The implications of these findings suggest that heuristic algorithms can effectively complement exact models in scenarios where computational resources are limited or quick solutions are required.


\section{Decentralized vs Centralized}

Decentralized systems, like the one proposed in \cite{Verri}, operate as dynamic processes where individual naive agents make localized decisions. However, this decentralized decision-making can limit global optimization, leading to inefficiencies in airspace management and route planning.

In contrast, our centralized approach demonstrated superior efficiency through global optimization, enabling more effective airspace management and route planning. By leveraging centralized control, we achieved better performance in optimizing drone routes and minimizing delivery times.

This comparison is easily observed when visualizing the experiments in \citeonline{Verri} versus the ones developed here. It is evident that the increase in the number of drones (arrival time in \citeonline{Verri}) results in significantly longer paths and wait times compared to the exact model and heuristic proposed in our study.

While decentralized systems offer flexibility, they may struggle to achieve the level of optimization and coordination provided by centralized approaches. Future research should focus on integrating the strengths of both approaches to develop hybrid models that combine the agility of decentralized decision-making with the efficiency of centralized optimization. This integration could involve incorporating intelligence into the agents in the decentralized approach, since \textbf{decentralized is the future}.







\section{Limitations and Future Research Directions}



This study has several limitations that need to be addressed in future research:

\subsection{Limitations}

\begin{itemize}
    \item The experiments conducted in this study were confined to fixed and relatively small grid sizes. To comprehensively evaluate the efficacy of the heuristic and exact model, future studies should extend their analyses to larger grid sizes to capture the complexities and challenges inherent in more extensive environments.
    \item The utilization of the exact model solely with the $T$ value obtained from the heuristic represents a simplification in the experimental methodology. To enhance the accuracy and robustness of the approach, future investigations could employ advanced techniques, such as binary search in $\mathcal{O}( \log( T)$), to efficiently determine the optimal $T$ value while ensuring the model's feasibility.
    \item Both the heuristic and exact model proposed in this study operate under simplified assumptions regarding airspace conditions, neglecting factors such as drone speed variations and environmental conditions. Future research should strive to incorporate these real-world complexities into the modeling framework to enhance the accuracy and applicability of the solutions.
    \item The comparative analysis in this study focused exclusively on contrasting the heuristic against the exact model, overlooking the potential benefits of exploring alternative hybrid, heuristic, and metaheuristic approaches.
\end{itemize}



\subsection{Future Research}

Here we focus on advancements for the decentralized approach.

\begin{itemize}
    \item Advancing research on the decentralized model by exploring techniques from Dynamical/Stochastic Processes and Sequential Decision Optimization. These approaches can enhance the adaptability and robustness of decentralized systems in dynamic environments.
    \item Investigating reinforcement learning (RL) techniques tailored for decentralized drone logistics. RL algorithms can learn optimal policies for drone routing and airspace management, improving efficiency and adaptability.
    \item Leveraging Graph Neural Networks (GNNs) and Complex Networks approaches to develop policies for decentralized systems \cite{gnn_policyXIAO2023119074}. By incorporating insights from centrality measures and influence maximization, RL agents can prioritize actions that have the greatest impact on system dynamics.
    \item Enhancing exploration efficiency in decentralized systems by leveraging knowledge of the network's topology. By understanding node importance and connectivity, RL agents can navigate the environment more effectively, leading to improved learning and optimization of strategies.
\end{itemize}

\section{Project contributions to the student}

The project contributed to me because it was an excellent way to reinforce part of the knowledge passed on in the undergraduate course as algorithms and Graph applications. Furthermore, it was a great way to start studying \emph{optimization} and become aware of its techniques and methodologies because I had the opportunity to do lots of experiments and coding.

\section{Relation between the undergraduate course and the project}

The undergraduate course is heavily connected with the developed project. In particular, one can highlight the subjects of Algorithms and Data Structures, Discrete Mathematics, Computational Modeling in Graphs, Mathematical Programming, Theory of Computation and Formal Languages and Advanced Algorithms and Applications. These subjects present a wide range of knowledge and learning related to data structures, mixed integer linear programming, algorithm complexity and development. Overall, the undergraduate course provides a solid theoretical foundation in Computer Science.

Another fundamental point of the course was the opportunities for research development, which improved my ability to communicate ideas and solutions. In addition, the presence of study and extension groups further complements the contents of the bachelor's degree.
