\section[Heuristic]{Heuristic}
\label{Heuristic}

In this section, we present our heuristic approach for tackling the problem, focusing on the utilization of a distance measure as a heuristic metric to organize drones in ascending order(prioritized planning) based on their start and end points. This prioritized order is then employed in an iterative Breadth-First Search (BFS) on the temporal graph. The BFS process dynamically updates the constraints of occupied positions(conflict-based search), creating a sequential refinement of the temporal graph with each newly generated minimum path. By combining heuristic sorting and iterative BFS, our approach aims to efficiently navigate the solution space, providing a balance between effective path planning and adaptability to evolving constraints in the context of drone delivery.


MAPF algorithms are recognized for their versatility with various distance metrics \cite{Weise2023}. In our specific scenario, we chose the Euclidean Distance as our metric of choice. This decision stems from its efficacy as a tiebreaker for drones that traverse the same number of cells in the minimum path, as observed with the Manhattan Distance. Notably, drones may share the same Manhattan Distance but diverge in Euclidean Distance, particularly when paths involve changes in direction. For instance, drones following a straight-line trajectory in the minimum path encounter fewer potential movement positions, while paths with changes in axis may introduce additional possibilities, making Euclidean Distance a valuable discriminant in such cases. This strategic use of the Euclidean Distance enhances the overall performance and adaptability of our algorithm in navigating complex drone delivery scenarios, as exemplified in \ref{ssec:example}, especially in scenarios with high traffic congestion where having more possibilities is crucial for finding local optimal solutions.

While conflict-based search has exponential complexity \cite{DBLP:conf/socs/GordonFS21}, our algorithm has polynomial complexity in function of size of the grid and number of drones. However, we cannot guarantee optimality, and this is expected since, even in the 2D grid case, MAPF where each agent has three or more possible directions of movement is also NP-hard\cite{geft2023finegrained}.


\begin{table}[H]
  \centering
  \caption{Notation used in the Algorithm.}
  \label{tab:notation}
  \begin{tabular}{ll}
    \toprule
    Notation & Definition \\
    \midrule
    $\mathcal{V}$ &Set of vertices in the graph $: (i,j,t)$ \\
    $\mathcal{E}$ & Set of edges  \\
    $\mathcal{D}$ & Set of drones \\
    $\mathcal{S}$ & Set of already scheduled vertices \\
    $\mathcal{P}_d \subseteq \mathcal{V} $ & Path of drone $d$ \\
    $\mathcal{G} = (\mathcal{V},\mathcal{E}) $ & Temporal Graph \\
    \bottomrule
  \end{tabular}
\end{table}

Formally, using the notation in Table \ref{tab:notation}, we can create a temporal graph $\mathcal{G}$ that naturally represents our problem. Since the set of vertex $\mathcal{V}:= (i,j,t)$ represents every possible position $(i,j)$ at any given time $t$ in our problem.  This modeling approach streamlines the search for optimal solutions in an efficient manner. The representation allows us to capture the possibility of a drone waiting at its position at a specific time $t_{wait}$ through the graph edges. We achieve this by defining the set of outgoing edges for a vertex $v^*=(i^*,j^*,t^*$ as $\mathcal{E}_{v^*} = \{ (i^*+1,j^*,t^*+1), (i^*,j^*+1,t^*+1), (i^*-1,j^*,t^*+1), (i^*,j^* -1,t^*+1), (i^*,j^*,t^*+1) \}$. The edge $(i^*,j^*,t^*+1)$ means that the drone chooses a path in which it decides to wait a time step. We define our algorithm sequentially as follows:

\begin{enumerate}

   \item \textbf{Drones Sorting}: We begin by ascending sorting the drones using the Euclidean Distance as a heuristic, given by \( \sqrt{(x_{\text{begin}} - x_{\text{end}})^2 + (y_{\text{begin}} - y_{\text{end}})^2} \), \begin{equation}
        \mathcal{D}_{\text{sorted}} = \text{sort}(\mathcal{D}, \text{heuristic}) \; \text{.}
    \end{equation}
    
\item \textbf{Path for Each Drone}: For each drone \(d\) in the sorted order, we compute the path \(P_d\) using Breadth-First Search (BFS) on the graph \(\mathcal{G}\). The BFS is performed with the constraints imposed by the set \(\mathcal{S}\), \begin{equation}
        \forall d \in \mathcal{D}_{\text{sorted}}: \quad \mathcal{P}_d = \text{BFS}(\mathcal{G}, d) \; \text{.}
    \end{equation}
    
    \item \textbf{Constraints Update}: After determining paths for the respective drone, we update the set of already scheduled vertices \(\mathcal{S}\) immediately following the execution of each BFS sequentially. This update involves incorporating the vertices covered by the path into the existing set of constraints (previously scheduled paths), \begin{equation}
        \mathcal{S} = \mathcal{S} \cup \bigcup_{d \in \mathcal{D}_{\text{sorted}}} \mathcal{P}_d \; \text{.}
    \end{equation}

   Indeed, the \textbf{constraints} ensure that no two drones occupy the same vertex at the same time, \begin{equation}
    \forall d, d' \in \mathcal{D}, d \neq d', \forall (i, j, t) \in \mathcal{P}_d, (i, j, t) \notin \mathcal{P}_{d'} \; \text{.}
\end{equation}

    
\end{enumerate}

In the computational aspect, our implementation follows the standard implementation of BFS in 3D grids, since the time $t$ can be simply thought as a third dimension. The main difference of common implementations is the addition of a \textit{map}, that is a Red Black Tree, to manage the set of scheduled positions. This addition adds $\mathcal{O}(\log n)$ in the complexity of our algorithm. The pseudo algorithm of the heuristic is described in Algorithm \ref{alg:grid-bfs}.


\begin{algorithm}[H]
    \label{alg:grid-bfs}
    \DontPrintSemicolon
    \KwData{Grid dimensions $N$ (rows) and $M$ (columns), Number of drones $K$, List of drones \textit{drones}}
    \KwResult{Paths for each drone considering constraints}

    \BlankLine
    \ForEach{\textit{drone} in \textit{drones}}{
        Initialize data structures for BFS: \textit{bfs\_queue}, \textit{parent}, \textit{visited}\;
        Set \textit{flight\_time} of \textit{drone} to -1\;
        \While{No valid path found for \textit{drone}}{
            Increment \textit{flight\_time} of \textit{drone}\;
            Enqueue \textit{drone\_begin} with \textit{flight\_time} into \textit{bfs\_queue}\;
            \While{\textit{bfs\_queue} is not empty}{
                Dequeue a position and time \;
                \If{Position is the destination of \textit{drone}}{
                    Reconstruct and schedule path for \textit{drone}\;
                    \Return path\;
                }
                \If{Position is scheduled}{
                \textit{\textbf{Continue}}
                    \;
                }
                \For{Neighbor positions}{
                    \If{Position is valid and not visited}{
                        \If{Position is scheduled and no schedule discovered in neighbors yet}{
                            
                            Enqueue current position and \textit{flight\_time + 1} \;
                            
                            \textit{\textbf{Continue}} \;
                        }
                        Mark position as visited\;
                        Store parent information\;
                        Enqueue the neighbor position into \textit{bfs\_queue}\;
                    }
                }
            }
        }
    }
    \caption{Path Planning for Drones using BFS}
\end{algorithm}


\subsection{Example} \label{ssec:example}

\begin{figure}[h]
    \centering
    \begin{tikzpicture}[scale=.8, every node/.style={minimum size=1cm}]

        % Define quadcopter style
        \tikzset{
            quadcopter/.style={
                quadcopter side,
                fill=white,
                draw=gray,
                minimum width=1cm,
                below=of second-scope, 
            },
        }
        
        % Add another drone
        \node (quadcopterGreen) [quadcopter side, fill=white, draw=greenMW, minimum width=1cm, rotate=0] at (5,1) {};

        \node (quadcopterBrown) [quadcopter side, fill=white, draw=brown, minimum width=1cm, rotate=0] at (5,-11.3) {};

        \node (quadcopterOrange) [quadcopter side, fill=white, draw=orange, minimum width=1cm, rotate=0] at (4.4,-14.3) {};

        \node (quadcopterBlue) [quadcopter side, fill=white, draw=blue, minimum width=1cm, rotate=0] at (-5,-8.2) {};
        
        % Último Escopo (t=6)
        \begin{scope}[yshift=-423, every node/.append style={yslant=0.5, xslant=-1}, yslant=0.5, xslant=-1]

            

            \fill[white, fill opacity=0.9] (0,0) rectangle (5,5);
            
            \draw[step=4mm, black] (0,0) grid (5,5);
            \draw[black, thick] (0,0) rectangle (5,5); % Borders
            \fill[orange] (2.05,2.05) rectangle (2.35,2.35); % center pixel
            \node (centerBegint2) at (2.2,2.2) {B}; % place 'B' in the center
            % \fill[greenMW] (1.65,2.05) rectangle (1.95,2.35); % left
            % \fill[greenMW] (2.45,2.05) rectangle (2.75,2.35); % right
            \fill[orange] (2.05,1.95) rectangle (2.35,1.65); % bottom
            \fill[orange,yshift=-11.2] (2.05,1.95) rectangle (2.35,1.65); % bottom
            \fill[orange,yshift=-22.5] (2.05,1.95) rectangle (2.35,1.65); % bottom
            \fill[orange,yshift=-33.5] (2.05,1.95) rectangle (2.35,1.65); % bottom
            
            %\fill[greenMW] (2.05,2.45) rectangle (2.35,2.75); % top
            % 8 -pixel setting
           
            % \fill[greenMW] (2.75,1.95) rectangle (2.45,1.65); % bottom-right
            % \fill[greenMW] (1.65,1.95) rectangle (1.95,1.65); % bottom-left
            
            \node (landedOrange) at (1.5,0.8) {};

            \node(pathdestinyt6) [black, font=\bfseries] at ([yshift=-45.3]centerBegint2) {E};
        \end{scope}



        % Penúltimo Escopo (t=5)
        \begin{scope}[yshift=-340, every node/.append style={yslant=0.5, xslant=-1}, yslant=0.5, xslant=-1]
            \fill[white, fill] (0,0) rectangle (5,5);
            \draw[step=4mm, black] (0,0) grid (5,5);
            \draw[black, thick] (0,0) rectangle (5,5); % Borders
            \fill[orange!55] (2.05,2.05) rectangle (2.35,2.35); % center pixel
            \node (centerBegint3) at (2.2,2.2) {B}; % place 'B' in the center

            \fill[orange!55,xshift=-22.6] (2.45,2.05) rectangle (2.75,2.35); % most left
            

            \node (landedBrown) at (1.5,0.6) {};

            %\fill[blue] (1.65,2.05) rectangle (1.95,2.35); % left
            % left X
            \draw[orange] (1.65,2.05) -- (1.95,2.35);
            \draw[orange] (1.65,2.35) -- (1.95,2.05);
            % left bottom X
            \draw[orange, yshift=-11.3] (1.65,2.05) -- (1.95,2.35);
            \draw[orange, yshift=-11.3] (1.65,2.35) -- (1.95,2.05);   
            % most left X
            \draw[orange, xshift=-11.3] (1.65,2.05) -- (1.95,2.35);
            \draw[orange, xshift= -11.3] (1.65,2.35) -- (1.95,2.05);   

            \fill[orange!55,xshift=11.3] (2.45,2.05) rectangle (2.75,2.35); % most right
            % most right X
            \draw[orange, xshift=34.4] (1.65,2.05) -- (1.95,2.35);
            \draw[orange, xshift=34.4] (1.65,2.35) -- (1.95,2.05);

            % bottom center right X
            \draw[orange, xshift=34.4, yshift= -11.3] (1.65,2.05) -- (1.95,2.35);
            \draw[orange, xshift=34.4, yshift= -11.3] (1.65,2.35) -- (1.95,2.05);

            %  top center right X
            \draw[orange, xshift=34.4, yshift= 11.3] (1.65,2.05) -- (1.95,2.35);
            \draw[orange, xshift=34.4, yshift= 11.3] (1.65,2.35) -- (1.95,2.05);
            
            % most most right X
            \draw[orange, xshift=34.4 + 11.3] (1.65,2.05) -- (1.95,2.35);
            \draw[orange, xshift=34.4 + 11.3] (1.65,2.35) -- (1.95,2.05);

            

            \fill[orange!55] (2.45,2.05) rectangle (2.75,2.35); % right
            
            \fill[orange!55] (2.05,2.45) rectangle (2.35,2.75); % top
            \fill[orange!55, xshift=11.3] (2.05,2.45) rectangle (2.35,2.75); % top right
            \fill[orange!55, xshift=-11.3] (2.05,2.45) rectangle (2.35,2.75); % top left
            \fill[orange!55, yshift=11.3] (2.05,2.45) rectangle (2.35,2.75); % most top
            \fill[orange!55] (2.05,1.95) rectangle (2.35,1.65); % bottom

            \fill[orange!55, xshift= 11.3] (2.05,1.95) rectangle (2.35,1.65); % bottom right

            \fill[orange!55,yshift=-11.3] (2.05,1.95) rectangle (2.35,1.65); % most bottom
            
            %\fill[greenMW] (1.65,2.45) rectangle (1.95,2.75); % top-left 
            % top left X
            \draw[orange] (1.65,2.75) -- (1.95,2.45);
            \draw[orange] (1.65,2.45) -- (1.95,2.75);
            
            % top most left X
            \draw[orange, xshift=-11.3] (1.65,2.75) -- (1.95,2.45);
            \draw[orange, xshift=-11.3] (1.65,2.45) -- (1.95,2.75);
            
            % \draw[orange,xshift=-11.3] (1.65,2.75) -- (1.95,2.45);
            % \draw[orange,xshift=-11.3] (1.65,2.45) -- (1.95,2.75);   

            %\fill[greenMW] (2.45,2.45) rectangle (2.75,2.75); % top-right
            \draw[orange] (2.45,2.45) -- (2.75,2.75);
            \draw[orange] (2.45,2.75) -- (2.75,2.45);

            %\fill[greenMW] (2.75,1.95) rectangle (2.45,1.65); % bottom-right
            \draw[orange] (2.45,1.95) -- (2.75,1.65);
            \draw[orange] (2.45,1.65) -- (2.75,1.95);

            % top most X 
            \draw[orange,xshift= -34 +45.4,yshift=11] (1.65,2.75) -- (1.95,2.45);
            \draw[orange,xshift=-34 +45.4,yshift=11] (1.65,2.45) -- (1.95,2.75);

            % top most right X 
            \draw[orange,xshift= -34 +45.4 + 11.2,yshift=11] (1.65,2.75) -- (1.95,2.45);
            \draw[orange,xshift=-34 +45.4 + 11.2,yshift=11] (1.65,2.45) -- (1.95,2.75);

            % top most left X 
            \draw[orange,xshift= -34 +45.4 - 11.2,yshift=11] (1.65,2.75) -- (1.95,2.45);
            \draw[orange,xshift=-34 +45.4 - 11.2,yshift=11] (1.65,2.45) -- (1.95,2.75);

            % top most last X 
            \draw[orange,xshift= -34 +45.4 ,yshift=22.2] (1.65,2.75) -- (1.95,2.45);
            \draw[orange,xshift=-34 +45.4 ,yshift=22.2] (1.65,2.45) -- (1.95,2.75);

            % bottom most X
            \draw[orange,xshift= -34 +45.4,yshift=-34.4] (1.65,2.75) -- (1.95,2.45);
            \draw[orange,xshift=-34 +45.4,yshift=-34.4] (1.65,2.45) -- (1.95,2.75);

            % bottom most right X
            \draw[orange,xshift= -34 +45.4 +11.2 ,yshift=-34.4] (1.65,2.75) -- (1.95,2.45);
            \draw[orange,xshift=-34 +45.4 + 11.2,yshift=-34.4] (1.65,2.45) -- (1.95,2.75);

            % bottom most left X
            \draw[orange,xshift= -34 +45.4 -11.2 ,yshift=-34.4] (1.65,2.75) -- (1.95,2.45);
            \draw[orange,xshift=-34 +45.4  -11.2,yshift=-34.4] (1.65,2.45) -- (1.95,2.75);

            % bottom most last X
            \draw[orange,xshift= -34 +45.4,yshift=-45.8] (1.65,2.75) -- (1.95,2.45);
            \draw[orange,xshift=-34 +45.4,yshift=-45.8] (1.65,2.45) -- (1.95,2.75);

                    

            %\fill[blue] (1.65,1.95) rectangle (1.95,1.65); % bottom-left
            % 2. ring
            %\fill[greenMW] (1.25,1.55) rectangle (1.55,1.25); % bottom-left
            %\fill[greenMW] (0.85,1.55) rectangle (1.15,1.25); % bottom-left
            %\fill[greenMW] (0.85,1.15) rectangle (1.15,0.85); % bottom-left

            \fill[brown, xshift=22.7, yshift=-11.2] (1.25,0.75) rectangle (1.55,0.45); % bottom-left

            \node(pathdestinyt3) [black, font=\bfseries] at ([yshift=-45.3]centerBegint3) {E};
        \end{scope}


        % Escopo (t=4)
        \begin{scope}[yshift=-250, every node/.append style={yslant=0.5, xslant=-1}, yslant=0.5, xslant=-1]
            \fill[white, fill] (0,0) rectangle (5,5);
            \draw[step=4mm, black] (0,0) grid (5,5);
            \draw[black, thick] (0,0) rectangle (5,5); % Borders
            \fill[orange!55] (2.05,2.05) rectangle (2.35,2.35); % center pixel
            \node (centerBegint3) at (2.2,2.2) {B}; % place 'B' in the center
            
            %\fill[blue] (1.65,2.05) rectangle (1.95,2.35); % left
            \draw[orange] (1.65,2.05) -- (1.95,2.35);
            \draw[orange] (1.65,2.35) -- (1.95,2.05);   

            \node (landedBlue) at (1.1,2.3) {};

            % most right X
            \draw[orange, xshift=34.4] (1.65,2.05) -- (1.95,2.35);
            \draw[orange, xshift=34.4] (1.65,2.35) -- (1.95,2.05);  

            \fill[orange!55] (2.45,2.05) rectangle (2.75,2.35); % right
            \fill[orange!55] (2.05,2.45) rectangle (2.35,2.75); % top
            \fill[orange!55] (2.05,1.95) rectangle (2.35,1.65); % bottom
            
            %\fill[greenMW] (1.65,2.45) rectangle (1.95,2.75); % top-left
            \draw[orange] (1.65,2.75) -- (1.95,2.45);
            \draw[orange] (1.65,2.45) -- (1.95,2.75);   
            
            % \draw[orange,xshift=-11.3] (1.65,2.75) -- (1.95,2.45);
            % \draw[orange,xshift=-11.3] (1.65,2.45) -- (1.95,2.75);   

            %\fill[greenMW] (2.45,2.45) rectangle (2.75,2.75); % top-right
            \draw[orange] (2.45,2.45) -- (2.75,2.75);
            \draw[orange] (2.45,2.75) -- (2.75,2.45);

            %\fill[greenMW] (2.75,1.95) rectangle (2.45,1.65); % bottom-right
            \draw[orange] (2.45,1.95) -- (2.75,1.65);
            \draw[orange] (2.45,1.65) -- (2.75,1.95);

            % top most X 
            \draw[orange,xshift= -34 +45.4,yshift=11] (1.65,2.75) -- (1.95,2.45);
            \draw[orange,xshift=-34 +45.4,yshift=11] (1.65,2.45) -- (1.95,2.75);

            % bottom most X
            \draw[orange,xshift= -34 +45.4,yshift=-34.4] (1.65,2.75) -- (1.95,2.45);
            \draw[orange,xshift=-34 +45.4,yshift=-34.4] (1.65,2.45) -- (1.95,2.75);

            \fill[blue] (1.65,1.95) rectangle (1.95,1.65); % bottom-left
            % 2. ring
            %\fill[greenMW] (1.25,1.55) rectangle (1.55,1.25); % bottom-left
            %\fill[greenMW] (0.85,1.55) rectangle (1.15,1.25); % bottom-left
            %\fill[greenMW] (0.85,1.15) rectangle (1.15,0.85); % bottom-left

            \fill[brown, xshift=22.6] (1.25,0.75) rectangle (1.55,0.45); % bottom-left

            \node(pathdestinyt3) [black, font=\bfseries] at ([yshift=-45.3]centerBegint3) {E};
        \end{scope}

        % Terceiro Escopo (t=3)
        \begin{scope}[yshift=-166, every node/.append style={yslant=0.5, xslant=-1}, yslant=0.5, xslant=-1]
            \fill[white, fill opacity=0.9] (0,0) rectangle (5,5);
            \draw[step=4mm, black] (0,0) grid (5,5);
            \draw[black, thick] (0,0) rectangle (5,5); % Borders
            \fill[orange!55] (2.05,2.05) rectangle (2.35,2.35); % lighter orange for center pixel
            \node (centerBegint3) at (2.2,2.2) {B}; % place 'B' in the center
            \fill[blue] (1.65,2.05) rectangle (1.95,2.35); % left
            \draw[orange] (2.45,2.05) -- (2.75,2.35);
            \draw[orange] (2.45,2.35) -- (2.75,2.05);
            %\fill[greenMW] (2.05,2.45) rectangle (2.35,2.75); % top
            \draw[orange] (2.05,2.45) -- (2.35,2.75);
            \draw[orange] (2.05,2.75) -- (2.35,2.45);

            %\fill[greenMW] (2.05,1.95) rectangle (2.35,1.65); % bottom
            \draw[orange] (2.05,1.95) -- (2.35,1.65);
            \draw[orange] (2.05,1.65) -- (2.35,1.95);

            
            %\fill[greenMW] (1.65,2.45) rectangle (1.95,2.75); % top-left
            %\fill[greenMW] (2.45,2.45) rectangle (2.75,2.75); % top-right
            %\fill[greenMW] (2.75,1.95) rectangle (2.45,1.65); % bottom-right
            %\fill[greenMW] (1.65,1.95) rectangle (1.95,1.65); % bottom-left
            % 2. ring
            %\fill[greenMW] (1.25,1.55) rectangle (1.55,1.25); % bottom-left
            %\fill[greenMW] (0.85,1.55) rectangle (1.15,1.25); % bottom-left
            %\fill[greenMW] (0.85,1.15) rectangle (1.15,0.85); % bottom-left

            \fill[brown, xshift=11.3] (1.25,0.75) rectangle (1.55,0.45); % bottom-left

            \node(pathdestinyt3) [black, font=\bfseries] at ([yshift=-45.3]centerBegint3) {E};

            \node(midpointX) at (2.6, 2.2){};
            
        \end{scope}

        % Escopo do meio (t=2)
        \begin{scope}[yshift=-83, every node/.append style={yslant=0.5, xslant=-1}, yslant=0.5, xslant=-1]

            \node [quadcopter side,fill=white,draw=orange,minimum width=1cm,rotate=10] (quadcoptert2) at (1,7) {};

            \fill[white, fill opacity=0.9] (0,0) rectangle (5,5);

            \node (arrivalOrange) at (1.8,2.1) {};
            
            \draw[step=4mm, black] (0,0) grid (5,5);
            \draw[black, thick] (0,0) rectangle (5,5); % Borders
            \fill[orange] (2.05,2.05) rectangle (2.35,2.35); % center pixel
            \node (centerBegint2) at (2.2,2.2) {B}; % place 'B' in the center
            % \fill[greenMW] (1.65,2.05) rectangle (1.95,2.35); % left
            % \fill[greenMW] (2.45,2.05) rectangle (2.75,2.35); % right
            \fill[greenMW] (2.05,1.95) rectangle (2.35,1.65); % bottom
            \node (landedGreen) at (1.8,1.9) {};
            %\fill[greenMW] (2.05,2.45) rectangle (2.35,2.75); % top
            % 8 -pixel setting
            \fill[blue] (1.65,2.45) rectangle (1.95,2.75); % top-left
            % \fill[greenMW] (2.45,2.45) rectangle (2.75,2.75); % top-right
            % \fill[greenMW] (2.75,1.95) rectangle (2.45,1.65); % bottom-right
            % \fill[greenMW] (1.65,1.95) rectangle (1.95,1.65); % bottom-left
            % 2. ring
            % \fill[greenMW] (1.25,1.55) rectangle (1.55,1.25); % bottom-left
            % \fill[greenMW] (0.85,1.55) rectangle (1.15,1.25); % bottom-left
            % \fill[greenMW] (0.85,1.15) rectangle (1.15,0.85); % bottom-left
            \fill[brown] (1.25,0.75) rectangle (1.55,0.45); % bottom-left
            \node(pathdestinyt2) [black, font=\bfseries] at ([yshift=-45.3]centerBegint2) {E};
        \end{scope}

        % Primeiro Escopo (t=1)
        \begin{scope}[yshift=0, every node/.append style={yslant=0.5, xslant=-1}, yslant=0.5, xslant=-1]

           \node [quadcopter side,fill=white,draw=orange,minimum width=1cm,rotate=10] (quadcoptert1) at (3,7.6) {};
            
           \node (orangeTakeOff) at (2.5,1.8) {};
        
            \fill[white, fill opacity=0.9] (0,0) rectangle (5,5);
            \draw[step=4mm, black] (0,0) grid (5,5); % Grid definition
            \draw[black, thick] (0,0) rectangle (5,5); % Borders
            \fill[greenMW] (2.05,2.05) rectangle (2.35,2.35); % center pixel
            \draw[red, thin] (2.05,2.05) rectangle (2.35,2.35); % border
            \node (centerBegint1) at (2.2,2.2) {B}; % place 'B' in the center
            \fill[blue] (2.05,2.45) rectangle (2.35,2.75); % top
            % \fill[blue] (1.65,2.45) rectangle (1.95,2.75); % top-left
            % \fill[blue] (2.45,2.45) rectangle (2.75,2.75); % top-right
            % \fill[blue] (2.75,1.95) rectangle (2.45,1.65); % bottom-right
            % \fill[blue] (1.65,1.95) rectangle (1.95,1.65); % bottom-left
            % % 2. ring
            % \fill[blue] (1.25,1.55) rectangle (1.55,1.25); % bottom-left
            % \fill[blue] (0.85,1.55) rectangle (1.15,1.25); % bottom-left
            % \fill[blue] (0.85,1.15) rectangle (1.15,0.85); % bottom-left
            % \fill[blue] (1.25,0.75) rectangle (1.55,0.45); % bottom-left

            \fill[brown, xshift= -11.3] (1.25,0.75) rectangle (1.55,0.45); % bottom-left
            
            % Insert 'E' three squares down the center
            \node(pathdestinyt1) [black, font=\bfseries] at ([yshift=-45.3]centerBegint1) {E};
            
        \end{scope}

        % Draw annotations
        \draw[-latex, thick, gray, dashed] (-5.4,-3.5) node[left] {occupied} to[out=0, in=200] (-0.5,-4.1);

        \draw[-latex, thick, red] (quadcoptert1) to[out=35, in=90] node[midway, above] {failed takeoff} (orangeTakeOff);

        \draw[-latex, thick, orange] (quadcoptert2) to[out=0, in=10] node[very near start, above] {arrival} (arrivalOrange);

        %\draw[-latex, thick, orange] ([xshift=-10]centerBegint3) to[out=0, in=10] (midpointX);

        \draw[thick, green!60!black, dashed] (landedGreen) to[out=0, in=10] node[above, pos=0.45 , font=\footnotesize] {landed} (quadcopterGreen);

        \draw[thick, brown!, dashed] (landedBrown) to[out=0, in=10] node[ above, pos=1 , font=\footnotesize] {landed} (quadcopterBrown);

        \draw[thick, orange!, dashed] (landedOrange) to[out=0, in=10] node[ above, pos=0.75 , font=\footnotesize] {path found} (quadcopterOrange);

        \draw[thick, blue!60!white, dashed] (landedBlue) to[out=0, in=13] node[ above, pos=0.85 , font=\footnotesize] {landed} (quadcopterBlue);

        % Annotations
        \draw[thick, gray!60!black] (6,4) node {t=1};
        \draw[thick, gray!60!black] (6,0) node {t=2};
        \draw[thick, gray!60!black] (6,-3) node {t=3};
        \draw[thick, gray!60!black] (6,-6) node {t=4};
        \draw[thick, gray!60!black] (6,-9) node {t=5};
        \draw[thick, gray!60!black] (6.2,-12.4) node {t=6};
    \end{tikzpicture}
    \caption{Algorithm visualization. Source: The authors.}
    \label{fig:visualgo}
\end{figure}


In Figure \ref{fig:visualgo}, the algorithm visualization shows its ability to address corner cases, including take-off in occupied positions and collision avoidance.

The depicted scenario involves determining the route for the orange drone within a grid where blue, brown, and green drones have already established their scheduled routes. These pre-existing paths serve as constraints in the optimization problem.

Upon closer inspection of Figure \ref{fig:visualgo}, it is evident that both the orange and brown drones cover an equal number of positions in the grid. However, an important factor influencing the algorithm's decision-making is a heuristic chosen for prioritization. In this case, the heuristic involves considering the shorter Euclidean distance covered by the brown drone. Consequently, during the sorting process, the algorithm prioritizes the brown drone based on this chosen heuristic, irrespective of the equality in the number of positions traversed.



\subsection{Complexity Analysis and Boundedness}
\label{secc:complexity_analysis}

Given $K$ the number of drones and
$N, M$ the grid sizes, the number of rows, and the number of columns, respectively. We can analyze our algorithm regarding these inputs using asymptotic computational complexity.

The complexity of our algorithm is the worst case of a BFS in the temporal graph, considering the $\mathcal{O}(\log S)$ from the scheduled structure, where $S= N M T$ is the size of the search space. It can be seen that this would be $\mathcal{O}(N M T)\mathcal{O}(\log N M T)$, where $T$ is the maximum time a drone lands. However, $T$ is surely bounded by $\mathcal{O}(K N M)$. In addition, it can be shown that $T$ is bounded by $\mathcal{O}((N+M) K)$. An intuitive view of this is shown in \ref{fig:worst_path}, where we have three drones on a $3\times 4$ grid with the same initial position($s_i$) and final position($g_i$), where $i=1\dots3$. In this case, the shorter path for each drone is $n+m$,$n=3;m=4$, and in the worst case, we will wait for each drone to complete its path, thus $k(n+m)$ is an upper bound for our heuristic since our heuristic always chooses a shorter path than $k(n+m)$ because we do not wait for each drone to complete the path.   Then the complexity is bounded by $$\mathcal{O}((N+M) K N M \log((N+M) K N M))= \mathcal{O}(\max{(N,M)} N M K  \log(\max{(N,M)} N M K)) $$  $\approx \mathcal{O}(N^3  K \log(N^3 K) = \mathcal{O}(N^3  K \max{(\log N , \log K))} $, if the grid is square. 



\begin{figure}[H]
    \centering
    \includegraphics[width=0.8\textwidth]{img/worst_path.drawio.pdf}
    \caption{Worst case path.}
    \label{fig:worst_path}
\end{figure}



\subsection{Adaptability for 3D}

A notable difference from our method to others in MAPF is that our algorithm remains consistent across any grid dimension. That is, the algorithm for a 2D grid obtains solutions in the same way as for a 3D or 4D grid.

This fact becomes clear when we see that both our methods, the heuristic and the exact model, are based solely on the topology of the graph. There is no spatial dependency. When we increase the dimension to 3D, we are simply increasing the neighborhood of each node. In some sense, each vertex $v$ in 3D space will have two additional edges in its neighborhood (up and down).

Again, using the notation in Table \ref{tab:notation}, we can create a temporal graph $\mathcal{G}$ that naturally represents our problem in three dimensional space. The set of vertices $\mathcal{V} := (i,j,k,t)$ represents every possible position $(i,j,k)$ at any given time $t$ in our problem. Our definition is almost the same, the difference is we add one more index $k$ and consequently the possibility of the drone going up($k+1$) and down($k-1$). We achieve this by defining the set of outgoing edges for a vertex $v^* = (i^*,j^*,k^*,t^*)$ as $\mathcal{E}_{v^*} = \{ 
(i^*+1,j^*,k^*,t^*+1), 
(i^*,j^*+1,k^*,t^*+1), 
(i^*-1,j^*,k^*,t^*+1), 
(i^*,j^*-1,k^*,t^*+1), 
(i^*,j^*,k^*+1,t^*+1),
(i^*,j^*,k^*-1,t^*+1),
(i^*,j^*,k^*,t^*+1)
\}$.


